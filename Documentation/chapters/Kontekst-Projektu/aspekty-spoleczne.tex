%! Author = Wiktor Rostkowski, Mateusz Budzisz
%! Date = 05/24/2024

\section{Aspekty Społeczne}
\label{sec:aspekty-spoleczne}

W erze „fake news”, gdzie występuje duża ilość niskiej jakości informacji, trudno znaleźć wiarygodne źródła informacji.
Konieczne jest wielokrotne weryfikowanie źródeł przez każdą grupę społeczną, od młodzieży po seniorów.

\indent Aplikacja spełnia liczne potrzeby turystów pragnących efektywnie zaplanować zwiedzanie miasta bez znacznego nakładu czasowego.
Turystyczny planer miejski stanowi kompleksowy zbiór informacji niezbędnych dla każdego turysty, wolny od niedopowiedzeń.
Przewidywaną grupą docelową są osoby odwiedzające daną miejscowość, posiadające umiarkowaną znajomość technologii, które chcą dostosować podróż zarówno dla siebie, jak i dla większych grup, na przykład wycieczek dla znajomych.

\indent Na potrzeby turystów proponowane jest rozwiązanie z następującymi funkcjami:

\begin{enumerate}
   \item Interaktywna mapa z naniesionymi atrakcjami turystycznymi zawierająca zawsze aktualne informacje o atrakcjach turystycznych
   \item Generowanie planu wycieczki na podstawie wybranych miejsc oraz trasy pomiędzy nimi.
   \item Automatyczna aktualizacja planu zwiedzania na podstawie dynamicznych wydarzeń, takich jak zmiana pogody, rozkład jazdy czy dostępność atrakcji, co umożliwia planowanie w czasie rzeczywistym.
\end{enumerate}

Dzięki tym funkcjonalnościom możliwe jest uniknięcie wielogodzinnego poszukiwania informacji na temat dostępnych miejsc oraz efektywne zaplanowanie podróży.

\indent Przeanalizowano społeczne aspekty wdrożenia proponowanego rozwiązania, w wyniku czego zidentyfikowano zarówno pozytywne, jak i negatywne skutki.

\indent Jednym z założeń projektu jest wspieranie rozwoju turystyki.
Aplikacja promuje niszowe atrakcje, prezentując na mapie tylko wybrane miejsca.
Dzięki mniejszej liczbie znaczników, mniej popularne atrakcje mają większą szansę na znalezienie przez użytkowników.

\indent Przedstawione informacje są dostępne dla każdego odwiedzającego i obejmują nie tylko atrakcje turystyczne, ale także dostępny transport, a w przyszłości również zakwaterowanie i restauracje.
Dzięki temu podróżowanie staje się łatwiejsze i nie wymaga posiadania przewodnika.
Tego rodzaju informacje oszczędzają czas podróżników, którzy mogą skupić się na zwiedzaniu zamiast na poszukiwaniu informacji w Internecie, przeglądaniu forów podróżniczych czy papierowych przewodników.

\indent Eliminacja przetwarzania danych wrażliwych w przyjętym modelu zwiększa poczucie bezpieczeństwa użytkowników.
Korzystanie z systemu nie wiąże się z ryzykiem utraty lub nieuprawnionego wykorzystania danych, co jest kluczowe dla zachowania zaufania.
Misją zespołu projektowego jest zaspokajanie potrzeb użytkowników, uwzględniając ich interesy oraz przestrzeganie obowiązującego prawa, przy jednoczesnym zachowaniu poufności.

\indent Wytworzona aplikacja pozwala użytkownikom łatwo i samodzielnie uzyskać dostęp do trudno dostępnych informacji.
Dzięki wygodnemu planowaniu w formie interfejsu drag-and-drop, aplikacja zmniejsza atrakcyjność płatnych przewodników oferujących gotowe plany zwiedzania
W rezultacie może to zwiększać poczucie zagrożenia wśród przewodników turystycznych działających na wolnym rynku.

\indent Warto również wspomnieć, że podczas implementacji aplikacji, z powodu braku czasu, pominięto kilka funkcji, które obecnie mogą powodować pewne niezadowolenie użytkowników.
Na przykład, brak możliwości rezerwacji terminów w wybranych atrakcjach może być problematyczny dla użytkowników, szczególnie gdy wymagana jest wcześniejsza rezerwacja.
Ponadto, nie zdążono dodać integracji z zakwaterowaniem i lokalami gastronomicznymi.
Niemniej jednak, aplikacja została przystosowana dla osób z wadami wzroku, co stanowi istotny krok w kierunku zwiększenia jej dostępności i inkluzywności.

\indent W związku z powyższym zidentyfikowano obszary wymagające udoskonalenia proponowanego systemu, aby skuteczniej odpowiadał na potrzeby użytkowników oraz wywierał większy wpływ społeczno-etyczny.
Kluczowym aspektem z perspektywy społecznej w przyszłości będą elementy sieci społecznościowej, takie jak recenzje atrakcji turystycznych oraz możliwość dzielenia się planami zwiedzania z innymi użytkownikami.
