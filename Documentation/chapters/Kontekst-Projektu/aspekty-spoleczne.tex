%! Author = Wiktor Rostkowski
%! Date = 05/1/2024


\section{Aspekty Społeczne}
\label{subsec:aspekty-spoleczne}



Niestety, w erze „fake news”, gdzie występuje duża ilość niskiej jakości informacji, trudno znaleźć potrzebne nam kompendium wiedzy. Każda grupa społeczna, od młodzieży do seniorów, jest podatna na tę sytuację[].
Aplikacja odpowiada na wiele potrzeb turystów, którzy chcą poruszać się po nieznanym mieście. Jest swoistym zbiorem wiedzy potrzebnej dla każdego turysty, bez żadnych niedopowiedzeń. Dlatego przewidywaną grupą docelową są osoby odwiedzające daną miejscowość z umiarkowaną znajomością technologii, personalizujące podróż dla siebie lub dla większej grupy osób, na przykład wycieczki dla znajomych.




   Na potrzeby turystów proponujemy rozwiązanie z poniższymi funkcjami:
I.	Generowanie planu wycieczki na podstawie wybranych zainteresowanych nas miejsc drogę pomiędzy nimi
II.	Automatyczna aktualizacja planu zwiedzania na podstawie dynamicznych wydarzeń, takich jak zmiana pogody, rozkład jazdy czy dostępność atrakcji, co umożliwia planowanie w czasie rzeczywistym.
III.	Interaktywna mapa z naniesionymi atrakcjami turystycznymi zawierająca zawsze aktualne informacje o atrakcjach turystycznych

Dzięki takim funkcjonalnościom można uniknąć wielogodzinnego poszukiwania odpowiednich informacji na temat dostępnych miejsc jak i zaplanować podróż.
Społeczne aspekty
Przeprowadzona została analiza społecznych aspektów realizowanego projektu, na podstawie której wyodrębnione zostały zarówno pozytywne, jak i negatywne skutki. Poniżej zaprezentowane zostały wyniki analizy.
Jednym z założeń projektu jest promowanie rozwoju turystyki. Aplikacja pozwala na promowanie niszowych atrakcji.
Z tego powodu przedstawiamy na mapie tylko atrakcje turystyczne, znaczników jest mniej, więc mniej popularne atrakcje mają większą szansę na odnalezienie przez użytkownika.
Przedstawione informacje są jak dostępne dla każdego odwiedzającego, przez znormalizowane źródło informacji potrzebne turystom nie tylko informacje o atrakcjach, ale też dostępny transport a w przyszłości  możliwe zakwaterowanie oraz restauracje.
Dzięki temu każda osoba ma mniejszy próg wejścia w podróżowania i może zrezygnować z potrzeby posiadania przewodnika.
Takie informacje przekładają się na oszczędność czasu dla każdego podróżnika, chcącego korzystać z przyjemności zwiedzenia a nie szukania informacji w Internecie czy przeglądając fora podróżnicze lub papierowe przewodniki.


Ponadto, eliminacja obracania się danymi wrażliwymi w przyjętym modelu przyczynia się do zwiększenia poczucia bezpieczeństwa użytkowników. Korzystanie z systemu nie niesie dla nich ryzyka utraty lub nieuprawnionego wykorzystania danych, co jest istotne dla zachowania zaufania. Misją zespołu projektowego jest zaspokajanie potrzeb użytkowników z uwzględnieniem ich interesów oraz przestrzeganie obowiązującego prawa, przy jednoczesnym unikaniu wszelkich form dyskryminacji.
W związku z powyższym zarysowują się obszary, w których można ulepszyć, wyłaniają się obszary, w których można przyszłościowo ulepszyć proponowany system tak, aby lepiej zaspokajał potrzeby użytkowników oraz miał większy wpływ społeczno-etyczny. Istotnym krokiem byłoby umożliwienie użytkownikowi dostosowania kolorystyki i czcionek w aplikacji, co jest szczególnie istotne dla osób z zaburzeniami widzenia, trudnościami w rozpoznawaniu barw.

Negatywne czynniki społeczne:
Po pierwsze, wytworzona aplikacja umożliwia użytkownikom samodzielny i łatwy dostęp do informacji obecnie trudno dostępnych, zatem w konsekwencji może zwiększać poczucie zagrożenia u przewodników turystycznych funkcjonujących na wolnym rynku. 
Po drugie, istnieje ryzyko utraty danych związanych z nieukończonym procesem wybierania trasy, szczególnie gdy użytkownik korzysta z aplikacji bez posiadania konta użytkownika. Ponadto, dane takie jak trasy i atrakcje są przechowywane w pamięci przeglądarki, co zwiększa ryzyko ich utraty, jeśli użytkownik nie jest zalogowany. Co więcej, użytkownik może stracić dane podczas zmiany urządzenia bez wcześniejszego zalogowania się na obu z nich.
Warto także nadmienić, iż od początku projektowania aplikacji nie uwzględniono kilku funkcjonalności, które obecnie mogą generować pewien niedosyt dla użytkownika. Na przykład, brak możliwości rezerwacji terminów w wybranych atrakcjach może stanowić problem dla niektórych użytkowników, szczególnie, gdy wymaga to wcześniejszej rezerwacji. Ponadto, użytkownik musi samodzielnie zaplanować przerwy, co może być kłopotliwe dla niektórych osób. Brak integracji z zakwaterowaniem w początkowej fazie rozwoju aplikacji może również ograniczać możliwości użytkowników w planowaniu podróży.
Dyskryminacja użytkowników w kontekście integracji komunikacji miejskiej w danej miejscowości z naszą aplikacją może wynikać z różnic w jakości danych oraz dostępności funkcji dla użytkowników w zależności od miejsca ich pobytu.
Jeśli miasto nie zdecyduje się na pełną integrację z naszą aplikacją, dane związane z tym obszarem mogą być mniej kompleksowe i aktualne niż w przypadku miast, które aktywnie uczestniczą w projekcie. W rezultacie użytkownicy korzystający z naszej aplikacji w takich obszarach mogą napotkać na braki informacji, błędne dane czy ograniczone możliwości korzystania z funkcji.

\paragraph{Paragraf}


