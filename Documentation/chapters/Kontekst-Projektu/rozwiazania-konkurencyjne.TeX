\section{Rozwiązania konkurencyjne}
\label{ch:rozwiązania-konkurencyjne}

\paragraph{Wstęp}

Na rynku dostępne są obecnie różne rozwiązania mające na celu ułatwienie planowania podróży i posiadające różny zakres funkcjonalności.
Użytkownik może napotkać rozbudowane aplikacje integrujące wiele funkcjonalności takich jak układanie list atrakcji do zwiedzania,
optymalizacja tras podróży, dokumentowanie poniesionych wydatków, tworzenie list ułatwiających planowanie (listy czynności do wykonania,
listy zakupów, listy ułatwiające organizację pakowania), synchronizacja naszych planów podróży ze znajomymi, integrowanie rezerwacji hotelowych i lotów.
Aplikacje skupiające się na wspomaganiu procesu zwiedzania sugerują użytkownikowi interesujące punkty w jego lokalizacji (interesujące miejsca, hotele,
restauracje, sklepy itp.) i oznaczają je na mapie. Dodatkowo bardzo często są wyposażone w rozbudowany opis danej atrakcji, czasem również w formie
głosowej, co umożliwia zamienienie urządzenia mobilnego w osobisty audio przewodnik. Kolejną funkcjonalność stanowi możliwość dokonywania zakupu biletów
i rezerwowania atrakcji bezpośrednio z poziomu aplikacji turystycznej .Poniżej przedstawiono przykłady najpopularniejszych obecnie rozwiązań
ułatwiających planowanie zwiedzania.

\paragraph{Wanderlog}

Jest to aplikacja reklamująca się jako planer podróży, ze szczególnym naciskiem na organizowanie wakacji oraz wycieczek samochodem.
W wersji podstawowej jest darmowa, posiada również płatną wersję premium  oferującą dodatkowe funkcjonalności.
Aplikacja jest dostępna poprzez przeglądarkę internetową jak również poprzez dedykowaną aplikację na urządzenia mobilne.
Wanderlog umożliwia układanie list z interesującymi użytkownika miejscami i wydarzeniami, które są graficznie przedstawione w
postaci pinezek na mapie google. Po wybraniu lokalizacji i daty podróży, użytkownik może dokonać przeglądu ofert noclegów.
Aplikacja umożliwia tworzenie planów podróży razem z innymi użytkownikami oraz ich synchronizację w czasie rzeczywistym.
Ponadto użytkownik ma dostęp do spersonalizowanych sugestii.

\paragraph{TripIt}

Jest to planer podróży integrujący wiele funkcjonalności w celu maksymalnego ułatwienia użytkownikowi procesu podróżowania.
Stanowi alternatywę dla wspomnianej wcześniej aplikacji Wanderlog i również posiada darmową wersję podstawową oraz wersję płatną
opartą na modelu subskrypcyjnym, która posiada dodatkowe funkcjonalności. W wersji podstawowej użytkownik ma możliwość układania
planów podróży, które są dostępne na wielu urządzeniach jednocześnie. Udostępnia statystyki, wytyczne dotyczące restrykcji COVID-19
oraz umożliwia dodawanie zdjęć, kodów QR oraz plików PDF do planów podróży. Aplikacja zapewnia nawigację między punktami,
mapy lotnisk, sugestie co do intersujących miejsc blisko lokalizacji użytkownika,
jak również informowania o poziomie niebezpieczeństwa danej okolicy.
Podstawowa wersja aplikacji umożliwia również dzielenie się planami z innymi użytkownikami oraz synchronizację kalendarza.
W wersji płatnej użytkownik ma dostęp do szeregu funkcjonalności ułatwiających podróżowanie samolotem, np. informacja o dostępności
lepszych miejsc, przypomnienia o zarezerwowanych lotach, powiadomienia o statusie lotów w czasie rzeczywistym, mapy lotnisk wraz
ze szczegółowymi informacjami o położeniu obiektów, informacje o punkcie odbioru bagażu.

\paragraph{Harmony}

Umożliwia tworzenie planów podróży wraz ze znajomymi w czasie rzeczywistym oraz synchronizację z kalendarzem Google.
Ponadto aplikacja umożliwia śledzenie poniesionych wydatków i podziału kosztów na poszczególne osoby.
Użytkownik ma możliwość otrzymywania sugestii generowanych przez AI dotyczących interesujących miejsc w danej lokalizacji
jak również rezerwowania wycieczek w aplikacji. Aplikacja umożliwia również tworzenie list rzeczy do wykonania w
celu łatwego śledzenia postępów. Obiekty do zwiedzania są zwizualizowane na mapie Google w postaci pinezek.

\paragraph{Rove.me}

Jest to aplikacja, która sugeruje użytkownikowi najlepszy czas na odwiedzenie danego miejsca lub wydarzenia w ciągu roku.
Aplikacja informuje również o typowej pogodzie występującej w interesującym użytkownika miejscu z uwzględnieniem pory roku lub daty.

\paragraph{Roadtrippers}

Umożliwia tworzenie planów podróży ze szczególnym uwzględnieniem podróży samochodem.
Użytkownik oznacza na mapie interesujące go miejsca takie jak atrakcje, hotele, stacje paliw, sklepy itp.
i udostępnia informacje o odległości od danej destynacji oraz szacowany czas dotarcia do niej.
Aplikacja skierowana jest do użytku na terenie Stanów Zjednoczonych.

\paragraph{Visit a City}

Użytkownik ma możliwość tworzenia list obiektów do zwiedzania jak również rezerwowania wycieczek i aktywności w oparciu o dużą bazę dostępnych lokalizacji.
Posiadają one rozbudowane informacje i oceny dodawane przez innych użytkowników, na podstawie których możliwe jest dokonywanie wyborów, ponadto aplikacja
sugeruje interesujące miejsca w pobliżu lokalizacji użytkownika. Aplikacja jest dostępna mobilnie na urządzeniach z systemem iOS i Android.

\paragraph{SmartGuide}

Zamysłem aplikacji jest dostarczenie użytkownikom platformy dzięki której urządzenie osobiste może zostać zamienione w przewodnik turystyczny.
Zwiedzanie odbywa się po zaplanowanych trasach, a dzięki śledzeniu lokalizacji użytkownika za pomocą nawigacji GPS, aplikacja może określić kiedy
osoba zwiedzająca dotarła do interesującego punktu i odtworzyć zapis audio z opisem obiektu. Informacje o oglądanym obiekcie dostępne są również
w formie tekstowej. Aplikacja skierowana jest nie tylko do turystów, ale również do organizatorów wycieczek, którzy mogą tworzyć własne trasy zwiedzania.

\paragraph{Fodor's City Guide}

Aplikacja rekomenduje użytkownikowi najciekawsze miejsca do zwiedzania oraz najlepsze restauracje, hotele i sklepy, jak również umożliwia rezerwację
tych usług.jest dostępna jest na urządzeniach z systemem iOS.

\paragraph{Tripadvisor}

Aplikacja umożliwia planowanie wycieczek, rezerwację usług  w oparciu między innymi o rekomendacje, opinie oraz wskazówki innych użytkowników.
Rekomendowane miejsca znajdujące się w pobliżu lokalizacji użytkownika są przedstawione na mapie w formie pinezek. Aplikacja umożliwia również dostęp
do zakupionych biletów w formie elektronicznej. Jest dostępna na urządzeniach z systemem iOS i Android.

\paragraph{Podsumowanie}

Wymienione aplikacje różnią się poziomem rozbudowania i co za tym idzie oferowanymi funkcjonalnościami.
Wspólnym mianownikiem jest możliwość tworzenia list z obiektami do odwiedzenia. Najbardziej rozbudowane aplikacje umożliwiają rezerwowanie usług
oraz środków transportu, a także ustalanie optymalnych tras podróży. Każda z aplikacji oferuje graficzne oznaczenie obiektów na mapie. Aplikacje
pełniące rolę przewodników wyposażone są w informacje turystyczne na temat obiektów odwiedzanych przez użytkownika, czasami również w formie audio.
Użytkownik może również kierować się opiniami i wskazówkami innych zwiedzających, które zostały przez nich zamieszczone na temat danego obiektu,
czy usługi. Na ten moment wydaje się jednak, że żadna z aplikacji nie dysponuje możliwością dynamicznego i zautomatyzowanego układania planu zwiedzania,
który jest dostosowany do zainteresowań i możliwości czasowych użytkownika odwiedzającego dane miejsce.
