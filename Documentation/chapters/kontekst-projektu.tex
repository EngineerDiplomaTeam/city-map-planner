%! Author = Wiktor Rostkowski, Mateusz Budzisz
%! Date = 05/24/2024

\chapter{Kontekst projektu}
\label{ch:kontekst-projektu}

\section{Aspekty społeczne}
\label{sec:aspekty-spoleczne}

W erze „fake news”, gdzie występuje duża ilość niskiej jakości informacji, trudno znaleźć wiarygodne źródła informacji.
Konieczne jest wielokrotne weryfikowanie źródeł przez każdą grupę społeczną,~od~młodzieży~po~seniorów~\cite{fakenews}.

\indent Aplikacja spełnia liczne potrzeby turystów pragnących efektywnie zaplanować zwiedzanie miasta wykorzystując rzetelne~i~pewne informacje~o~jego atrakcjach,~a~także~bez~znacznego nakładu czasowego.
Turystyczny planer miejski stanowi kompleksowy zbiór informacji niezbędnych~dla~każdego turysty, wolny~od~niedopowiedzeń.
Przewidywaną grupą docelową~są~osoby odwiedzające daną miejscowość, posiadające umiarkowaną znajomość technologii, które chcą dostosować podróż zarówno~dla~siebie,~jak~i~dla~większych grup,~na~przykład wycieczek~dla~znajomych.

\indent Na~potrzeby turystów proponowane jest rozwiązanie~z~następującymi funkcjami:

\begin{enumerate}
    \item Interaktywna mapa~z~naniesionymi atrakcjami turystycznymi zawierająca zawsze aktualne informacje~o~atrakcjach turystycznych.
    \item Generowanie planu wycieczki~na~podstawie wybranych miejsc oraz trasy pomiędzy nimi.
    \item Automatyczna aktualizacja planu zwiedzania~na~podstawie dynamicznych wydarzeń, takich~jak~zmiana pogody, rozkład jazdy~czy~dostępność atrakcji,~co~umożliwia planowanie~w~czasie rzeczywistym.
\end{enumerate}

Dzięki~tym~funkcjonalnościom możliwe jest uniknięcie wielogodzinnego poszukiwania informacji~na~temat dostępnych miejsc oraz efektywne zaplanowanie podróży.

\indent Przeanalizowano społeczne aspekty wdrożenia proponowanego rozwiązania,~w~wyniku czego zidentyfikowano zarówno pozytywne,~jak~i negatywne skutki.

\indent Jednym~z~założeń projektu jest wspieranie rozwoju turystyki.
Aplikacja promuje niszowe atrakcje, prezentując~na~mapie tylko wybrane miejsca.
Dzięki mniejszej liczbie znaczników, mniej popularne atrakcje mają większą szansę~na~znalezienie przez użytkowników.

\indent Przedstawione informacje~są~dostępne~dla~każdego odwiedzającego~i~obejmują~nie~tylko atrakcje turystyczne,~ale~także dostępny transport,~a~w przyszłości również zakwaterowanie~i~restauracje.
Dzięki temu podróżowanie staje~się~łatwiejsze~i~nie wymaga posiadania przewodnika,~a~samodzielna koordynacja~nie~wiąże~się~z nadmiernym obciążeniem poznawczym.
Człowiek podczas zwiedzania miasta skupia~się~na doświadczaniu~i~zwiedzaniu,~a~nie selekcjonowaniu danych, informacji~i~podejmowaniu decyzji.
Aplikacja zapobiega obciążeniu poznawczemu (ang.\ cognitive load), które jest konsekwencją nadmiaru opcji dostępnych~w~danym mieście.
Dzięki ``uwolnionym zasobom'' możliwa jest koncentracja użytkowników~na~faktycznym poznawaniu miasta~\cite{cognitive-biases}.

\indent Dodatkowo, rezygnacja~z~przetwarzania danych wrażliwych~w~przyjętym modelu zwiększa poczucie bezpieczeństwa użytkowników.
Korzystanie~z~systemu~nie~wiąże~się~z ryzykiem utraty~lub~nieuprawnionego wykorzystania danych,~co~jest kluczowe~dla~zachowania zaufania.
Celem jaki towarzyszył zespołowi projektowemu, było zaspokajanie potrzeb użytkowników, uwzględniając~ich~interesy oraz przestrzeganie obowiązującego prawa, przy jednoczesnym zachowaniu poufności.

\indent Wytworzona aplikacja pozwala użytkownikom łatwo~i~samodzielnie uzyskać dostęp~do~trudno dostępnych informacji.
Dzięki wygodnemu planowaniu~w~formie interfejsu drag-and-drop, aplikacja zmniejsza atrakcyjność płatnych przewodników oferujących gotowe plany zwiedzania
W rezultacie może~to~generować negatywne konsekwencje społeczne~w~postaci zwiększonego poczucia zagrożenia wśród przewodników turystycznych działających~na~wolnym rynku.

\indent Warto również wspomnieć,~że~podczas implementacji aplikacji,~z~powodu braku czasu, pominięto kilka funkcji, które~w~aktualnej wersji mogą generować potencjalne niezadowolenie użytkowników.
Na przykład, brak możliwości rezerwacji terminów~w~wybranych atrakcjach może~być~problematyczny~dla~użytkowników, szczególnie~gdy~wymagana jest wcześniejsza rezerwacja.
Ponadto,~nie~zdążono dodać integracji~z~zakwaterowaniem~i~lokalami gastronomicznymi.
Niemniej jednak, aplikacja została przystosowana~dla~osób~z~wadami wzroku,~co~stanowi istotny krok~w~kierunku zwiększenia~jej~dostępności~i~inkluzywności.

\indent W~związku~z~powyższym zidentyfikowano obszary wymagające udoskonalenia proponowanego systemu,~aby~skuteczniej odpowiadał~na~potrzeby użytkowników oraz wywierał większy wpływ społeczny.
Kluczowym aspektem~z~perspektywy społecznej~w~przyszłości będą elementy sieci społecznościowej, takie~jak~recenzje atrakcji turystycznych oraz możliwość dzielenia~się~planami zwiedzania~z~innymi użytkownikami.



\pagebreak
\section{Analiza ryzyka}
\label{sec:analiza-ryzyka}

\begin{longtable}{|p{.1\linewidth}|p{.15\textwidth}|p{.2\textwidth}|p{.25\textwidth}|p{.25\textwidth}|}
    \hline
    Ryzyko & Czynniki ryzyka & Charakterystyka ryzyka & Prawdopodobieństwo wystąpienia ryzyka & Planowane działania \\
    \hline
    \multirow{6}{=}{\parbox[c]{12cm}{\rotatebox[origin=c]{90}{\multirow{6}{=}{\textbf{Nieukończenie projektu w terminie}}}}}& Rozpad grupy & Odejście jednego z członków zespołu & niskie
    & Przejrzysta komunikacja w zespole i ustalenie wersji MVP produktu, która może zostać ukończona w uszczuplonym składzie \\
    \cline{2-5}
    & Czasowa niedostępność / niedyspozycja członka zespołu & Nagła lub zaplanowana nieobecność członka zespołu & średnie
    & Możliwie wczesne sygnalizowanie planowanej nieobecności \\
    \cline{2-5}
    & Zbyt ambitne założenia projektowe & Zaplanowane prace wykraczają poza moce przerobowe członków zespołu & średnie
    & Śledzenie postępu prac pod kątem wykonalności i zmieszczenia się w ograniczeniach czasowych, konsultacja z promotorem \\
    \cline{2-5}
    & Pełzające wymagania & Ciągle powiększana pula wymagań, zwiększająca zakres prac do wykonania & średnie & \\
    \cline{2-5}
    & Nauka nowych technologii / języków programowania & Zwiększenie ilości czasu potrzebnej na ukończenie zadania &
    wysokie & \\
    \cline{2-5}
    & Problem z komunikacją w zespole & Brak właściwego przepływu informacji między członkami zespołu & średnie
    & Cykliczne spotkania członków zespołu w celu omawiania postępu prac i planowania kolejnych działań \\
    \hline
    \pagebreak
    \hline
    \multirow{5}{=}{\parbox[c]{10cm}{\rotatebox[origin=c]{90}{\multirow{5}{=}{\textbf{Produkt nie spełnia wymagań projektowych}}}}} & Błędna analiza rynku & Niewłaściwe rozpoznanie potrzeb potencjalnych użytkowników oraz rozwiązań konkurencyjnych & niskie
    & Konsultacje z promotorem, stworzenie profilu docelowego użytkownika \\
    \cline{2-5}
    & Niespełnienie wymagań klientów & Finalny produkt nie zapewnia zakładanych korzyści dla użytkownika końcowego & średnie
    & Konsultacje z promotorem, cykliczna analiza kierunku rozwoju produktu \\
    \cline{2-5}
    & Wydanie wersji zawierającej krytyczne błędy & Produkt zawiera błędy uniemożliwiające korzystanie z jego podstawowych funkcjonalności & niskie
    & Napisanie odpowiednich testów sprawdzających działanie funkcjonalności \\
    \cline{2-5}
    & Zmiany w usługach zależnych & Wyłączenie lub zmiana warunków na jakich udostępniane są API wpływa na dostarczane funkcjonalności & średnie
    & Ustalenie alternatywnych dostawców potrzebnych usług \\
    \cline{2-5}
    & Zły dobór narzędzi projektowych & Wybrane narzędzia są nieodpowiednie do zaplanowanych prac & niskie
    & Zaznajomienie się z dokumentacją używanych narzędzi, konsultacje w zespole \\
    \hline
    \pagebreak
    \hline
    \multirow{2}{=}{\parbox[c]{3.5cm}{\rotatebox[origin=c]{90}{\multirow{2}{=}{\textbf{Niewłaściwe zaplanowanie przebiegu prac}}}}} & Przyjęcie nieodpowiedniej strategii projektowej & Przyjęta strategia nie pozwala na uzyskanie wymaganych rezultatów przy dostępnych zasobach & wysokie
    & Stałe monitorowanie kierunku prac, konsultacje z promotorem \\
    \cline{2-5}
    & Nierozpatrzenie potencjalnych przypadków brzegowych & Konieczność uwzględnienia nieprzewidzianych prac & wysokie
    & Staranne analizowanie planowanych funkcjonalności i ich wpływu na produkt \\
    \hline
    \multirow{1}{=}{\parbox[c]{3.5cm}{\rotatebox[origin=c]{90}{\multirow{1}{=}{\textbf{Zdarzenia nieprzewidziane}}}}}& Zbyt gwałtowny rozrost liczby użytkowników & Liczba użytkowników przekracza zaplanowane możliwości techniczne produktu & niskie
    & Projektowanie produktu z możliwością jego skalowania \\
    \hline

\end{longtable}
\pagebreak
\section{Rozwiązania konkurencyjne}
\label{sec:rozwiazania-konkurencyjne}
Na rynku dostępne~są~obecnie różne rozwiązania mające~na~celu ułatwienie planowania podróży~i~posiadające różny zakres funkcjonalności.
Użytkownik może napotkać rozbudowane aplikacje integrujące wiele funkcjonalności takich~jak~układanie list atrakcji~do~zwiedzania, optymalizacja tras podróży, dokumentowanie poniesionych wydatków, tworzenie list ułatwiających planowanie (listy czynności~do~wykonania, listy zakupów, listy ułatwiające organizację pakowania), synchronizacja naszych planów podróży~ze~znajomymi, integrowanie rezerwacji hotelowych~i~lotów.
Aplikacje skupiające~się~na wspomaganiu procesu zwiedzania sugerują użytkownikowi interesujące punkty~w~jego lokalizacji (interesujące miejsca, hotele, restauracje, sklepy itp.)~i~oznaczają~je~na mapie.
Dodatkowo bardzo często~są~wyposażone~w~rozbudowany opis danej atrakcji, czasem również~w~formie głosowej,~co~umożliwia zamienienie urządzenia mobilnego~w~osobisty audio przewodnik.
Kolejną funkcjonalność stanowi możliwość dokonywania zakupu biletów~i~rezerwowania atrakcji bezpośrednio~z~poziomu aplikacji turystycznej.
Poniżej przedstawiono przykłady najpopularniejszych obecnie rozwiązań ułatwiających planowanie zwiedzania.

\subsection{Wanderlog}
\label{subsec:wanderlog}
Jest~to~aplikacja reklamująca~się~jako planer podróży,~ze~szczególnym naciskiem~na~organizowanie wakacji oraz wycieczek samochodem.
W wersji podstawowej jest darmowa, posiada również płatną wersję premium  oferującą dodatkowe funkcjonalności.
Aplikacja jest dostępna poprzez przeglądarkę internetową~jak~również poprzez dedykowaną aplikację~na~urządzenia mobilne.
Wanderlog umożliwia układanie list~z~interesującymi użytkownika miejscami~i~wydarzeniami, które~są~graficznie przedstawione~w~postaci pinezek~na~mapie google.
Po wybraniu lokalizacji~i~daty podróży, użytkownik może dokonać przeglądu ofert noclegów.
Aplikacja umożliwia tworzenie planów podróży razem~z~innymi użytkownikami oraz~ich~synchronizację~w~czasie rzeczywistym.
Ponadto użytkownik~ma~dostęp~do~spersonalizowanych sugestii.

\subsection{TripIt}
\label{subsec:tripit}
Jest~to~planer podróży integrujący wiele funkcjonalności~w~celu maksymalnego ułatwienia użytkownikowi procesu podróżowania.
Stanowi alternatywę~dla~wspomnianej wcześniej aplikacji Wanderlog~i~również posiada darmową wersję podstawową oraz wersję płatną opartą~na~modelu subskrypcyjnym, która posiada dodatkowe funkcjonalności.
W wersji podstawowej użytkownik~ma~możliwość układania planów podróży, które~są~dostępne~na~wielu urządzeniach jednocześnie.
Udostępnia statystyki, wytyczne dotyczące restrykcji COVID-19 oraz umożliwia dodawanie zdjęć, kodów~QR~oraz plików~PDF~do planów podróży.
Aplikacja zapewnia nawigację między punktami, mapy lotnisk, sugestie~co~do interesujących miejsc blisko lokalizacji użytkownika,~jak~również informowania~o~poziomie niebezpieczeństwa danej okolicy.
Podstawowa wersja aplikacji umożliwia również dzielenie~się~planami~z~innymi użytkownikami oraz synchronizację kalendarza.
W wersji płatnej użytkownik~ma~dostęp~do~szeregu funkcjonalności ułatwiających podróżowanie samolotem, np.~informacja~o~dostępności lepszych miejsc, przypomnienia~o~zarezerwowanych lotach, powiadomienia~o~statusie lotów~w~czasie rzeczywistym, mapy lotnisk wraz~ze~szczegółowymi informacjami~o~położeniu obiektów, informacje~o~punkcie odbioru bagażu.

\subsection{Harmony}
\label{subsec:harmony}
Umożliwia tworzenie planów podróży wraz~ze~znajomymi~w~czasie rzeczywistym oraz synchronizację~z~kalendarzem Google.
Ponadto aplikacja umożliwia śledzenie poniesionych wydatków~i~podziału kosztów~na~poszczególne osoby.
Użytkownik~ma~możliwość otrzymywania sugestii generowanych przez~AI~dotyczących interesujących miejsc~w~danej lokalizacji,~jak~również rezerwowania wycieczek~w~aplikacji.
Aplikacja umożliwia również tworzenie list rzeczy~do~wykonania~w~celu łatwego śledzenia postępów.
Obiekty~do~zwiedzania~są~zwizualizowane~na~mapie Google~w~postaci pinezek.

\subsection{Rove.me}
\label{subsec:rove.me}
Jest~to~aplikacja, która sugeruje użytkownikowi najlepszy czas~na~odwiedzenie danego miejsca~lub~wydarzenia~w~ciągu roku.
Aplikacja informuje również~o~typowej pogodzie występującej~w~interesującym użytkownika miejscu~z~uwzględnieniem pory roku~lub~daty.

\subsection{Roadtrippers}
\label{subsec:roadtrippers}
Umożliwia tworzenie planów podróży~ze~szczególnym uwzględnieniem podróży samochodem.
Użytkownik oznacza~na~mapie interesujące~go~miejsca takie~jak~atrakcje, hotele, stacje paliw, sklepy itp.~i udostępnia informacje~o~odległości~od~danej celu podróży oraz szacowany czas dotarcia~do~niej.
Aplikacja skierowana jest~do~użytku~na~terenie Stanów Zjednoczonych.

\subsection{Visit a City}
\label{subsec:visit-a-city}
Użytkownik~ma~możliwość tworzenia list obiektów~do~zwiedzania,~jak~również rezerwowania wycieczek~i~aktywności~w~oparciu~o~dużą bazę dostępnych lokalizacji.
Posiadają~one~rozbudowane informacje~i~oceny dodawane przez innych użytkowników,~na~podstawie których możliwe jest dokonywanie wyborów, ponadto aplikacja sugeruje interesujące miejsca~w~pobliżu lokalizacji użytkownika.
Aplikacja jest dostępna mobilnie~na~urządzeniach~z~systemem~iOS~i Android.

\subsection{SmartGuide}
\label{subsec:smartguide}
Zamysłem aplikacji jest dostarczenie użytkownikom platformy dzięki, której urządzenie osobiste może zostać zamienione~w~przewodnik turystyczny.
Zwiedzanie odbywa~się~po zaplanowanych trasach,~a~dzięki śledzeniu lokalizacji użytkownika~za~pomocą nawigacji GPS, aplikacja może określić, kiedy osoba zwiedzająca dotarła~do~interesującego punktu~i~odtworzyć zapis audio~z~opisem obiektu.
Informacje~o~oglądanym obiekcie dostępne~są~również~w~formie tekstowej.
Aplikacja skierowana jest~nie~tylko~do~turystów,~ale~również~do~organizatorów wycieczek, którzy mogą tworzyć własne trasy zwiedzania.

\subsection{Fodor's City Guide}
\label{subsec:fodor's-city-guide}
Aplikacja rekomenduje użytkownikowi najciekawsze miejsca~do~zwiedzania oraz najlepsze restauracje, hotele~i~sklepy,~jak~również umożliwia rezerwację tych usług jest dostępna jest~na~urządzeniach~z~systemem iOS\@.

\subsection{Tripadvisor}
\label{subsec:tripadvisor}
Aplikacja umożliwia planowanie wycieczek, rezerwację usług ~w~oparciu między innymi~o~rekomendacje, opinie oraz wskazówki innych użytkowników.
Rekomendowane miejsca znajdujące~się~w~pobliżu lokalizacji użytkownika~są~przedstawione~na~mapie~w~formie pinezek.
Aplikacja umożliwia również dostęp~do~zakupionych biletów~w~formie elektronicznej.
Jest dostępna~na~urządzeniach~z~systemem~iOS~i~Android.

\subsection{Podsumowanie}
\label{subsec:podsumowanie}
W tabeli poniżej przedstawiono szczegółowe porównanie wszystkich konkurencyjnych rozwiązań, uwzględniając ich główne funkcje.

\newgeometry{top=0.5cm,bottom=0.5cm,left=2.7cm,right=0.5cm}
\begin{landscape}

    \begin{longtable}{|>{\raggedright\arraybackslash}p{5cm}|>{\centering\arraybackslash}p{1.5cm}|>{\centering\arraybackslash}p{1.5cm}|>{\centering\arraybackslash}p{1.5cm}|>{\centering\arraybackslash}p{1.5cm}|>{\centering\arraybackslash}p{1.5cm}|>{\centering\arraybackslash}p{1.5cm}|>{\centering\arraybackslash}p{1.5cm}|>{\centering\arraybackslash}p{1.5cm}|>{\centering\arraybackslash}p{1.5cm}|}
        \hline
        \textbf{Funkcjonalności} & \textbf{Wanderlog} & \textbf{TripIt} & \textbf{Harmony} & \textbf{Rove.me} & \textbf{Roadtrippers} & \textbf{Visit a City} & \textbf{SmartGuide} & \textbf{Fodor’s City Guide} & \textbf{Tripadvisor} \\
        \hline
        Układanie list atrakcji & Tak & Tak & Tak & Nie & Tak & Tak & Nie & Tak & Tak \\
        \hline
        Optymalizacja tras podróży & Tak & Tak & Nie & Nie & Tak & Nie & Tak & Nie & Nie \\
        \hline
        Dokumentowanie wydatków & Nie & Tak & Tak & Nie & Nie & Nie & Nie & Nie & Nie \\
        \hline
        Tworzenie list czynności & Tak & Tak & Tak & Nie & Nie & Nie & Nie & Nie & Nie \\
        \hline
        Synchronizacja planów z innymi użytkownikami & Tak & Tak & Tak & Nie & Nie & Nie & Nie & Nie & Tak \\
        \hline
        Rezerwacje hotelowe i lotów & Tak & Tak & Tak & Nie & Nie & Tak & Nie & Tak & Tak \\
        \hline
        Sugestie interesujących miejsc & Tak & Tak & Tak & Tak & Nie & Tak & Tak & Tak & Tak \\
        \hline
        Informacje turystyczne (tekstowe) & Tak & Tak & Tak & Tak & Tak & Tak & Tak & Tak & Tak \\
        \hline
        Informacje turystyczne (audio) & Nie & Nie & Nie & Nie & Nie & Nie & Tak & Nie & Nie \\
        \hline
        Zakup biletów i rezerwacja atrakcji & Nie & Tak & Tak & Nie & Nie & Tak & Nie & Tak & Tak \\
        \hline
        Mapy lotnisk, nawigacja między punktami & Nie & Tak & Nie & Nie & Nie & Nie & Nie & Nie & Nie \\
        \hline
        Informacje o restrykcjach COVID-19 & Nie & Tak & Nie & Nie & Nie & Nie & Nie & Nie & Nie \\
        \hline
        Informacje o pogodzie & Nie & Nie & Nie & Tak & Nie & Nie & Nie & Nie & Nie \\
        \hline
        Synchronizacja z kalendarzem & Nie & Tak & Tak & Nie & Nie & Nie & Nie & Nie & Nie \\
        \hline
        Sugestie dotyczące czasu wizyty & Nie & Nie & Nie & Tak & Nie & Nie & Nie & Nie & Nie \\
        \hline
        Powiadomienia o statusie lotów & Nie & Tak (wersja płatna) & Nie & Nie & Nie & Nie & Nie & Nie & Nie \\
        \hline
        Oznaczenie obiektów na mapie & Tak & Tak & Tak & Nie & Tak & Tak & Tak & Tak & Tak \\
        \hline
        Wersja mobilna (iOS, Android) & Tak & Tak & Tak & Tak & Tak & Tak & Tak & Tak & Tak \\
        \hline
    \end{longtable}

\end{landscape}
\restoregeometry

Wymienione aplikacje różnią się poziomem rozbudowania i co za tym idzie oferowanymi funkcjonalnościami.
Wspólnym mianownikiem jest możliwość tworzenia list z obiektami do odwiedzenia.
Najbardziej rozbudowane aplikacje umożliwiają rezerwowanie usług oraz środków transportu, a także ustalanie optymalnych tras podróży.
Każda z aplikacji oferuje graficzne oznaczenie obiektów na mapie.
Aplikacje pełniące funkcję przewodników wyposażone są w informacje turystyczne na temat obiektów odwiedzanych przez użytkownika, czasami również w formie audio.
Użytkownik może również kierować się opiniami i wskazówkami innych zwiedzających, które zostały przez nich zamieszczone na temat danego obiektu, czy usługi.
Na ten moment wydaje się jednak, że żadna z aplikacji nie dysponuje możliwością dynamicznego i zautomatyzowanego układania planu zwiedzania, który jest dostosowany do zainteresowań i możliwości czasowych użytkownika odwiedzającego dane miejsce.

\section{Aspekty biznesowe}
\label{sec:aspekty-biznesowe}
\subsection{Przegląd rynku}
\label{sec:przeglad-rynku}
Założeniem zespołu projektowego było stworzenie produktu unikalnego na skalę światową, aby wprowadzić nowatorskie rozwiązanie na rynek polski oraz międzynarodowy. W związku z tym przeprowadziliśmy analizę konkurencyjnych rozwiązań. Po przeszukaniu istniejącego oprogramowania doszliśmy do wniosku, że nikt wcześniej nie stworzył podobnego produktu.
Nasze rozwiązanie umożliwia użytkownikowi dynamiczne dostosowanie planu podróży na podstawie aktualnej pogody, zgodnie z jego dostępnością czasową. Dodatkowo użytkownik zawsze ma dostęp do informacji o dostępności wybranych atrakcji.
\subsection{Model biznesowy}
\label{sec:aspekty-biznesowy}

