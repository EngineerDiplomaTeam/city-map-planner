%! Author = Wiktor Rostkowski
%! Date = 28/05/2024
\newgeometry{} % Ustawienie mniejszych marginesów górnych i dolnych

\chapter{Analiza wymagań}
\label{ch:analiza-wymagan}

\section{Sposób gromadzenia wymagań}
\label{sec:sposob-gromadzenia-wymagan}

Wymagania dla projektu zostały zebrane w pierwszym semestrze. Zostały określone na podstawie analizy funkcjonowania

Na podstawie opracowanego dokumentu SWS, przygotowano wymagania dotyczące funkcjonalności projektu. \newline Wykorzystana została metodyka priorytetyzacji wymagań MoSCoW\footnote{Metoda MoSCoW – Must, Should, Could, Won't}[]:
\begin{itemize}
    \item \textbf{M – must (musi być)} – wymaganie bezkompromisowo musi zostać zrealizowane, bez niego projekt nie zostanie ukończony;
    \item \textbf{S – should (powinno być)} – wymaganie powinno być zrealizowane, jeżeli tylko jest taka możliwość;
    \item \textbf{C – could (może być)} – wymaganie powinno być zawarte w projekcie, jeśli wystarczy na nie czasu;
    \item \textbf{W – won't (nie będzie)} – wymaganie nie powinno być zawarte w tym wydaniu projektu, ale może być zrealizowane w przyszłości.
\end{itemize}


\section{Aktorzy}
\label{sec:aktorzy}

\begin{stakeholder}[label={tab:stakeholder:someholder},caption={opis udzialowca}]
    \id{UOB 01}
    \name{Podstawowy użytkownik aplikacji (np. Turysta) }
    \descr{Osoba korzystająca z aplikacji w celu poznawania POI i optymalnego przemieszczania się między nimi }
    \type{Ożywiony bezpośredni}
    \viewpoint{Użytkownik - operator}
    \limitations{Wymaga umiarkowanych umiejętności technicznych}
    \requ{-----------------}
\end{stakeholder}

\begin{stakeholder}[label={tab:stakeholder:someholder},caption={Opis udzialowca}]
    \id{UOB 02}
    \name{Zespół projektowy}
    \descr{Pomysłodawcy i autorzy rozwiązania}
    \type{Ożywiony bezpośredni}
    \viewpoint{Techniczny}
    \limitations{Ograniczony czas realizacji projektu}
    \requ{-----------------}
\end{stakeholder}

\begin{stakeholder}[label={tab:stakeholder:someholder},caption={Opis udzialowca}]
    \id{UOB 03}
    \name{Firmy o charakterze najmu krótko-terminowym }
    \descr{Firmy o charakterze najmu krótko-terminowego mające bezpośredni kontakt z turystami}
    \type{Ożywiony niebezpośredni }
    \viewpoint{Współpraca z turystami}
    \limitations{Brak}
    \requ{-----------------}
\end{stakeholder}

\section{Wymagania ogólne i dziedzinowe}
\label{sec:wymagania-ogolne-i-dziedzinowe}


\begin{requirementstab}[label={tab:requirements:general1},caption={Karta wymagania: Obsługa użytkowników}]
    \id{WO1}
    \priority{M}
    \name{Obsługa użytkowników}
    \descr{System powinien działać bez większych przestojów i przerw w funkcjonowaniu, nawet przy dużej liczbie użytkowników. }
    \sholder{UOB2}
    \reqrelated{}
\end{requirementstab}
\begin{requirementstab}[label={tab:requirements:general2},caption={Karta wymagania: Działanie w czasie rzeczywistym Offline}]
    \id{WO2}
    \priority{C}
    \name{Działanie w czasie rzeczywistym offline lub online }
    \descr{Aplikacja musi działać i synchronizować dane do 15 minut. 
    Wersja offline wykorzystuje ostatnie zapisane informacje takie jak Umożliwia podgląd wcześniej zaplanowanych wycieczek}
    \sholder{UOB2}
    \reqrelated{F19}
\end{requirementstab}
\begin{requirementstab}[label={tab:requirements:general3},caption={Karta wymagania: Panel administracyjny}]
    \id{WO3}
    \priority{S}
    \name{Zarządzanie danymi z poziomu aplikacji}
    \descr{Możliwość sprawdzanie aktualnych danych POI poprzez panel aplikacji internetowej (Panel administracyjny) }
    \sholder{UOB2}
    \reqrelated{WO3}
\end{requirementstab}
\begin{requirementstab}[label={tab:requirements:general4},caption={Karta wymagania: Licencje i prawa}]
    \id{WO4}
    \priority{M}
    \name{Licencje i prawa}
    \descr{Wszystkie technologie, narzędzia, wzorce oraz inne elementy niezbędne do realizacji systemu muszą być licencjonowane i pochodzić bezpośrednio od głównych dostawców, a nie od pośredników. }
    \sholder{UOB2}
    \reqrelated{}
\end{requirementstab}
\begin{requirementstab}[label={tab:requirements:general5},caption={Karta wymagania: Responsywny interfejs}]
    \id{WO5}
    \priority{M}
    \name{Responsywny interfejs}
    \descr{Interfejs użytkownika powinien być wykonany zgodnie z praktyką programowania responsywnych stron internetowych.  }
    \sholder{UOB2}
    \reqrelated{}
\end{requirementstab}
\begin{requirementstab}[label={tab:requirements:general6},caption={Karta wymagania: Rejestracja użytkowników}]
    \id{WO6}
    \priority{S}
    \name{Rejestracja użytkowników  }
    \descr{Rejestracja powinna przebiegać sprawnie, zgodnie ze wszystkimi standardami i przepisami prawa, a przede wszystkim musi być zgodna z RODO.  }
    \sholder{UOB2, UOB1}
    \reqrelated{}
\end{requirementstab}
\begin{requirementstab}[label={tab:requirements:general7},caption={Karta wymagania: Komercjalizacja}]
    \id{WO7}
    \priority{S}
    \name{Komercjalizacja }
    \descr{Aplikacja ma umożliwiać reklamowanie swoich usług partnerom biznesowym w sposób spójny z interfejsem.  }
    \sholder{UOB01, UOB02, UOB03}
    \reqrelated{}
\end{requirementstab}

\section{Wymagania funkcjonalne}
\label{sec:wymagania-funkcjonalne}

\begin{requirementstab}[label={tab:requirements:func1},caption={Karta wymagania: Interaktywna mapa}]
    \id{FO1}
    \priority{M}
    \name{Interaktywna mapa}
    \descr{Jako turysta chciałbym zobaczyć mapę wszystkich atrakcji dostępnych w mieście z lotu ptaka 
    }
    \acceptcrit{Użytkownik może zobaczyć aktualne dostępne atrakcje dostępne w danym mieście }
    \inputdata{Użytkownik włączająć aplikacje widzi przedstawioną mapę}
    \preconditions{ dostępność systemów partnerów, głównie Stadia maps   }
    \postconditions{ Wyświetlenie atrakcji. }
    \exceptions{ Wyświetlanie komunikatu w przypadku niedostępności atrakcji w aktualnej chwili oraz przedstawienia propozycji zmiany punktu docelowego. }
    \implementation{ Użytkownik, korzystając z przeglądarki lub aplikacji, widzi interaktywną mapę z atrakcjami w danym mieście.}
    \sholder{UOB01, UOB02 }
    \reqrelated{}
\end{requirementstab}
\begin{requirementstab}[label={tab:requirements:func2},caption={Karta wymagania: Dane czasu rzeczywistego }]
    \id{FO2}
    \priority{S}
    \name{Wyświetlanie powiadomień o zmianach. }
    \descr{\begin{itemize}
        \item Jako Turysta 
        \item chcę dostawać powiadomienia o zmianach w planie,
        \item bo wtedy nie spóźnię się na czas otwarcia atrakcji. 
    \end{itemize}
    }
    \acceptcrit{Powiadomienie będzie zgodnie z aktualnym planem  }
    \inputdata{Aplikacja pamięta jakie użytkownik wybrał atrakcje do odwiedzenia i informuje o zmianach po synchronizacji. }
    \preconditions{ Użytkownik był wcześniej odwiedził stronę i posiada zapisane atrakcje.  }
    \postconditions{ Aplikacja przedstawi nowe zmiany po kliknięciu w powiadomienie.   }
    \exceptions{ Użytkownik nie zezwolił powiadomienia bądź nowe dane nie istnieją. }
    \implementation{ Na planie zajęć wyświetlają się nowe zmiany w innym kolorze  }
    \sholder{UOB01, UOB02 }
    \reqrelated{}
\end{requirementstab}
\begin{requirementstab}[label={tab:requirements:func3},caption={Karta wymagania: Widok listy POI}]
    \id{FO3}
    \priority{M}
    \name{Zakładka z przestawionymi dostępnymi wszystkimi atrakcjami  }
    \descr{\begin{itemize}
        \item Jako użytkownik  
        \item chce korzystać z aplikacji z listą dostępnych atrakcji w wersji pisanej
        \item ponieważ chce zobaczyć dostępne spektrum możlwości, bez szukania informacji na mapie i przeciągania. 
    \end{itemize}
    }
    \acceptcrit{Lista dostępnych POI. }
    \inputdata{Brak}
    \preconditions{ Użytkownik wszedł na strone z dostępnymi atrakcjami.  }
    \postconditions{ Użytkownik może przeczytać lub klinknąć na interesujące go atrakcje.   }
    \exceptions{ brak dostępnych atrakcji w wyznaczonym zakresie czasowym  }
    \implementation{ Aplikacja pobiera w tle plik bazy danych ważący 1MB.  }
    \sholder{UOB01, UOB02, UOB03 }
    \reqrelated{}
\end{requirementstab}
\begin{requirementstab}[label={tab:requirements:func4},caption={Karta wymagania: Panel zarządzania atrakcjami}]
    \id{FO4}
    \priority{M}
    \name{Panel zarządzania atrakcjami. }
    \descr{\begin{itemize}
        \item Jako administrator 
        \item potrzebuję panelu do zarządzania danymi atrakcji turystycznych wspomaganego sztuczną inteligencją
        \item aby szybko reagować na zmiany wprowadzane przez właścicieli atrakcji turystycznych. 
    \end{itemize}
    }
    \acceptcrit{Lista dostępnych punktów. }
    \inputdata{Konto z uprawnieniami administracyjnymi}
    \preconditions{ Użytkownik wszedł na ukrytą strone z panelem.  }
    \postconditions{ wykonywać podstawowe operacje CRUD   }
    \exceptions{ Jest możliwość dodawania kolejnych atrakcji. Sztuczna inteligencja może błędnie wykonać swoje zadanie.   }
    \implementation{ Panel administracyjny do zarządzania opiniami użytkowników (umożliwia cenzurę.).   }
    \sholder{ UOB02 }
    \reqrelated{W03,IO9}
\end{requirementstab}
\begin{requirementstab}[label={tab:requirements:func5},caption={Karta wymagania: Punkty interaktywnej mapy}]
    \id{FO5}
    \priority{M}
    \name{Znaczniki na interaktywnej mapie informujące o POI.}
    \descr{\begin{itemize}
        \item Jako użytkownik 
        \item chce móc wyświetlić wybrane atrakcje, które mnie interesują
        \item ponieważ  umożliwi to spojrzenie na opis POI.
    \end{itemize}
    }
    \acceptcrit{Mapa posiada znaczniki atrakcji dodanych poprzez administratora. Po kliknięciu znacznika pojawią się szczegółowe informacje o POI }
    \inputdata{Otwiera Aplikacje. }
    \preconditions{ Mapa Atrakcji z znacznikami dodanymi przez administratora  aplikacji.}
    \postconditions{ Brak  }
    \exceptions{ Brak dostępności API, aplikacja powinna wtedy pokazać błąd i próbować pobrać dane ponow    }
    \implementation{ Mają być duże znaczniki ze zdjęciami atrakcji które po kliknięciu otwierają szczegółowy widok    }
    \sholder{ UOB01, UOB02, UOB03  }
    \reqrelated{}
\end{requirementstab}
\begin{requirementstab}[label={tab:requirements:func6},caption={Karta wymagania: Oceny użytkowników }]
    \id{FO6}
    \priority{C}
    \name{Oceny użytkowników}
    \descr{\begin{itemize}
        \item Jako użytkownik 
        \item chce mieć możliwość oceniania atrakcjami gwiazdkami w skali 1-5
        \item ponieważ chciałbym podzielić się swoją opinią. 
    \end{itemize}
    }
    \acceptcrit{Każda atrakcja posiada możliwośc dodawania opinii. }
    \inputdata{Posiadanie konta Opinia w skali od 1-5 }
    \preconditions{ Można dodawać opinie liczbową.}
    \postconditions{ Brak  }
    \exceptions{ Użytkownik nie posiada konta albo został zablokowany na platformie.   }
    \implementation{ Przy danym punkcie zainteresowania (POI) dostępny jest przycisk z opcją dodania opinii, a powyżej wyświetlana jest średnia ocen. Każdy użytkownik może dodać jedną opinię w formie gwiazdek od 1 do 5.    }
    \sholder{ UOB01  }
    \reqrelated{F07}
\end{requirementstab}
\begin{requirementstab}[label={tab:requirements:func7},caption={Karta wymagania: Zdjęcia użytkowników + tekst}]
    \id{FO7}
    \priority{C}
    \name{Możliwość pisania opinii wraz ze zdjęciami pod punktami zainteresowania .}
    \descr{\begin{itemize}
        \item Jako użytkownik 
        \item chce możliwość pisania pisemnych opinii z możliwością dodania zdjecia
        \item aby móc przekazać większy kontekst moich opinii. 
    \end{itemize}
    }
    \acceptcrit{Opinia musi zostać zatwierdzona przez administratora, może póżniejszym etapie na podstawie algorytmu, który spełnia jakąś polityke prywatności. }
    \inputdata{Posiadanie konta zdjecia, tekst }
    \preconditions{ Można dodawać opinie (ewentualnie ze zdjeciem).}
    \postconditions{  }
    \exceptions{     brak konta, za duże zdjęcie, nie prawidłowy format zdjęcia      }
    \implementation{ Przy danym POI po kliknięciu jest dostępny przycisk z opcją napisania opinia , każdy użytkownik może dodać jedną opinie.
    Opinia musi zostać zatwierdzona przez moderatora, nowe opinie wyświetlają się w panelu administratora.  Formularz dodawania zdjecia używa kompresji formatu zdjecia. }
    \sholder{ UOB01  }
    \reqrelated{F06,F27}
\end{requirementstab}
\begin{requirementstab}[label={tab:requirements:func8},caption={Karta wymagania: Zapisane atrakcje turystyczne}]
    \id{FO8}
    \priority{S}
    \name{Zapisywanie list wybranych POI pod konkretną nazwą.}
    \descr{\begin{itemize}
        \item Jako użytkownik 
        \item chce możliwość zapisania wybranej trasy
        \item ponieważ umożliwi to przygotowanie trasy na później. 
    \end{itemize}
    }
    \acceptcrit{ Po wybraniu trasy, która zostanie przedstawiona, wyskakuje dymek z zapytaniem czy  chcesz dodać tą trase do ulubionych}
    \inputdata{  }
    \preconditions{ Zaplanowanie trasy w poprzednich widokach aplikacji }
    \postconditions{  Brak }
    \exceptions{  Jeśli użytkownik jest offline lub niezalogowany, pojawia się komunikat o błędzie:
     "Aby dodać trasę, należy być zalogowanym i mieć połączenie z internetem."   }
    \implementation{ Możliwość otworzenia zapisanej, optymalnej trasy bez konieczności ponownego jej obliczania oraz bez połączenia internetowego.}
    \sholder{ UOB01, UOB02 }
    \reqrelated{F09,F28}
\end{requirementstab}

\begin{requirementstab}[label={tab:requirements:func9},caption={Karta wymagania: Udostępnianie zapisanych list POI}]
    \id{FO9}
    \priority{C}
    \name{Udostępnianie zapisanych list POI pomiędzy użytkownikami}
    \descr{\begin{itemize}
        \item Jako użytkownik 
        \item chce mieć możliwość udostępnienia wybranej trasy
        \item ponieważ umożliwi to polecania wybranych trasy. 
    \end{itemize}
        }
    \acceptcrit{ W zakładce zapisanych listy widniej dymek z udostępnię danej trasy}
    \inputdata{Użytkownik klika na przycisk udostępnienia trasy.  }
    \preconditions{ Posiadanie zapisanej trasy  }
    \postconditions{     Użytkownik udostępniający musi przekazać link do udostępnionej trasy, 
     użytkownik odstający link musi zaakceptować udostępnioną trasę  }
    \exceptions{  Nie można przekroczyć 100 zapisanych tras.  }
    \implementation{ Każda trasa ma indywidualny kod w bazie danych. Właściciel trasy może ją edytować. a zobaczyć może każda osoba posiadająca link.}
    \sholder{ UOB01, UOB02, UOB03   }
    \reqrelated{F08,F28}
\end{requirementstab}
\begin{requirementstab}[label={tab:requirements:func10},caption={Karta wymagania: Wyznaczanie trasy pieszej }]
    \id{F10}
    \priority{M}
    \name{Wyznaczanie optymalnej trasy pieszej pomiędzy wybranymi POI}
    \descr{Jako użytkownik chcę w sposób automatyczny wyznaczyć optymalną trasę między atrakcjami aby wiedzieć jak map się przedostać do danej atrakcj
    }
    \acceptcrit{ Algorytm wyznacza trase od wybranego punktu pomiędzy 2 podanymi punktami na mapie.}
    \inputdata{Wybrane 2 atrakcje turystyczne - trasę.  }
    \preconditions{ brak }
    \postconditions{ brak  }
    \exceptions{ Brak dostępnej trasy między podanymi punktami  }
    \implementation{ Implementacja na podstawie BFS i Priority Queue.    }
    \sholder{ UOB01, UOB02   }
    \reqrelated{F09}
\end{requirementstab}
\begin{requirementstab}[label={tab:requirements:func11},caption={Karta wymagania: Wyznaczanie optymalnej trasy z udziałem komunikacji miejskiej}]
    \id{F11}
    \priority{S}
    \name{Wyznaczanie optymalnej trasy z udziałem komunikacji miejskiej}
    \descr{\begin{itemize}
        \item Jako użytkownik 
        \item chce mieć możliwość przejścia wybranej trasy pieszo oraz dostępnej komunikacji miejskiej
        \item aby przyśpieszyć podróż i zaoszczędzić czas
    \end{itemize}
    }
    \acceptcrit{ Możliwość wyboru trasy z udziałem komunikacji miejskiej.}
    \inputdata{Wybrane 2 POI }
    \preconditions{ Przedstawienie trasy podstawie wybranych punktów. }
    \postconditions{ brak albo połączenie z internetem aby aktualizować na bieżąco rozkład jazdy  }
    \exceptions{ Zaproponowanie innego transportu np. pieszo  }
    \implementation{ *************************************************
        }
    \sholder{ UOB01, UOB02   }
    \reqrelated{F09,F10}
\end{requirementstab}
\begin{requirementstab}[label={tab:requirements:func12},caption={Karta wymagania: Wyznacznia optymalnej trasy z udziałem samochodów osobowych}]
    \id{F12}
    \priority{S}
    \name{Wyznaczanie optymalnej trasy z udziałem samochodów osobowych}
    \descr{\begin{itemize}
        \item Jako użytkownik 
        \item chce mieć możliwość odbyć wybraną trase z udziałem samochodu
        \item ponieważ umożliwi to wykorzystanie szybszego transportu z punktu A do B. 
    \end{itemize}
    }
    \acceptcrit{ Algorytm umożliwia planowanie trasy z użyciem dróg przeznaczonych dla samochodów przy zachowaniu przepisów prawa drogowego takich jak drogi jednokierunkowe}
    \inputdata{Wybrane 2 POI }
    \preconditions{ Brak}
    \postconditions{  Algorytm zaprezentował  trasę z wykorzystaniem dróg przeznaczonych dla samochodów     }
    \exceptions{ Brak trasy możliwej do pokonania samochodem     }
    \implementation{ *************************************************
        }
    \sholder{ UOB01, UOB02   }
    \reqrelated{IO1,F09,F10,F11}
\end{requirementstab}
\begin{requirementstab}[label={tab:requirements:func13},caption={Karta wymagania: Podgląd Wygenerowanej optymalnej trasy }]
    \id{F13}
    \priority{M}
    \name{Widok wygenerowanej optymalnej trasy z możliwością podglądu jej na osi czasu}
    \descr{\begin{itemize}
        \item Jako użytkownik 
        \item chce mieć możliwość przejrzenia całej trasy 
        \item ponieważ umożliwi to przejrzenia zaproponowanej trasy, atrakcja po atrakcji. 
    \end{itemize}
    }
    \acceptcrit{ Widok listy zawierającej wszystkie uprzednio wybrane atrakcje turystyczne wraz z podstawowymi informacjiami o nich oraz trasy po między nimi.}
    \inputdata{ Uprzednio wygenerowany plan zwiedzania}
    \preconditions{ Brak }
    \postconditions{ Dostęp do internetu aby pobrać aktualizacje trasdy w przypadku zmian danych. }
    \exceptions{  Brak wybranego planu zwiedzania }
    \implementation{ Z lewej strony widać proponowaną trase a z prawej strony ekranu widać oś czas z wybranymi atrakcjami. Użytkownik może modyfikować trase oraz wybierać konkretne punkty na trasie, 
    w celu bliższego przyjrzenia się nim}
    \sholder{ UOB01, UOB03   }
    \reqrelated{F09,F10,F11,F12,F14, F15, F16}
\end{requirementstab}
\begin{requirementstab}[label={tab:requirements:func14},caption={Karta wymagania: Widok kalendarza podróży }]
    \id{F14}
    \priority{M}
    \name{Widok kalendarza podróży trasy powinien umożliwiać zmianę kolejności odwiedzania na osi czasu POI co w efekcie dostosuje resztę optymalnej trasy}
    \descr{\begin{itemize}
        \item Jako użytkownik 
        \item chce mieć możliwość ustalania godzin zwiedzania
        \item ponieważ kalendarz pokaże mi możliwość edycji trasy wycieczki optymalnej do godzin otwarcia. 
    \end{itemize}
    }
    \acceptcrit{ Użytkownik może wybrać każdy element z proponowanej trasy i go zmienić}
    \inputdata{Wybrane atrakcje turystyczne – trasa np. Zapisana trasa lub stworzenie nowej trasy.  }
    \preconditions{ Z edytowana nowa trasa. }
    \postconditions{ Prezentacja nowej trasy o nowe sugestie w podróży.  }
    \exceptions{  Brak przeładowania trasy przez brak internetu. }
    \implementation{ Poprzez przesuwanie punktów na osi czasu jest możliwość manipulowania trasą.}
    \sholder{ UOB01, UOB02   }
    \reqrelated{F09,F10,F11,F12,F14, F15, F16}
\end{requirementstab}
\begin{requirementstab}[label={tab:requirements:func15},caption={Karta wymagania: POI użytkowników }]
    \id{F15}
    \priority{C}
    \name{Użytkownicy powinni mieć możliwość dodawania na mapę własnych POI}
    \descr{\begin{itemize}
        \item Jako użytkownik 
        \item chce możliwość dodania nowych punktów i usunięcia z aktualnej listy dostępnych punków
        \item ponieważ chciałbym wyznaczyć trasę między punktami, którymi administratorzy nie dodali. 
    \end{itemize}
    }
    \acceptcrit{ Możliwość dodania niestandardowego punktu startowego i końcowego.}
    \inputdata{Wybrane atrakcje turystyczne - trasa. }
    \preconditions{  }
    \postconditions{   }
    \exceptions{  Brak punktów niestandardowych lub punkt nie mieści się w obrębie dostępnej mapy. }
    \implementation{ Na interaktywnej mapie powienien znajdować się przycisk do dodania własnych atrakcji, ale tylko gdy użytkownik jest zalogowany.}
    \sholder{ UOB01, UOB02   }
    \reqrelated{F09,F10,F11,F12,F14, F15, F16}
\end{requirementstab}
\begin{requirementstab}[label={tab:requirements:func18},caption={Karta wymagania:  Aktualna pozycja użytkownika na interaktywnej mapie}]
    \id{F18}
    \priority{C}
    \name{Aktualna pozycja użytkownika na interaktywnej mapie}
    \descr{\begin{itemize}
        \item Jako użytkownik 
        \item chce widzieć swoją pozycję na mapie 
        \item aby móc ocenić postępy w trasie.
    \end{itemize}
    } 
    \acceptcrit{ Dodany Tryb podróży.}
    \inputdata{Wybrane atrakcje turystyczne – inicjacja trasy. }
    \preconditions{Możliwość wykorzystanie oprogramowania jako nawigacja.   }
    \postconditions{ Tryb nawigacja daje wskazówki drogowe. }
    \exceptions{  Za mało wybranych punktów, np. Nie brak udostępnienia lokalizacji urządzenia. }
    \implementation{ Aplikacja na podstawie danych GPS śledzi postęp użytkownika.}
    \sholder{ UOB01 }
    \reqrelated{F17}
\end{requirementstab}
\begin{requirementstab}[label={tab:requirements:func19},caption={Karta wymagania: Trasa Offline}]
    \id{F19}
    \priority{S}
    \name{Trasa Offline}
    \descr{\begin{itemize}
        \item Jako użytkownik 
        \item chce możliwość otworzenia zapisanej trasy bez połączenia internetowego  
        \item ponieważ umożliwi to interaktywne korzystanie z aplikacji cały czas. 
    \end{itemize}
    }
    \acceptcrit{ Widok osi czasu obecnej trasy powinien domyślnie przeskakiwać do etapu na którym obecnie znajduje się użytkownik ( na podstawie czasu i pozycji usera).}
    \inputdata{Wybrane atrakcje turystyczne – trasa. }
    \preconditions{Aplikacja działa na zapisanych danych.   }
    \postconditions{ Aplikacja działająca bez podłączenia do internetu. }
    \exceptions{  Aplikacja została uruchomiona automatycznie po pobraniu bez Internetu (prosi o podłączenie do Internetu). }
    \implementation{ Aplikacja na podstawie danych GPS oraz zapisanych lokalnych informacjami na urządzeniu.}
    \sholder{ UOB01 }
    \reqrelated{F16}
\end{requirementstab}
\begin{requirementstab}[label={tab:requirements:func20},caption={Karta wymagania: Algorytm Plannera}]
    \id{F20}
    \priority{S}
    \name{Algorytm Plannera}
    \descr{\begin{itemize}
        \item Jako użytkownik 
        \item chce żeby planner korzystał z wszystkich możliwości 
        \item ponieważ umożliwi to interaktywne korzystanie cały czas bez dłużych nakładu czasowego. 
    \end{itemize}
    }
    \acceptcrit{ Widok osi czasu obecnej trasy powinien domyślnie przeskakiwać do etapu na którym obecnie znajduje się użytkownik ( na podstawie czasu i pozycji usera).}
    \inputdata{Wybrane atrakcje turystyczne – trasa. }
    \preconditions{Przestawia optymalną trase.   }
    \postconditions{ Trasa zostaje przedstawiona użytkownikowi. }
    \exceptions{ Brak atrakcji brak planu.}
    \implementation{ Algorytm optymalnej trasy powinien brać pod uwagę czas potrzeby na zwiedzenie konkretnej atrakcji.
    Algorytm optymalnej trasy powinien brać pod uwagę czas przeznaczony na zwiedzenia w ciągu danego dnia.
    Algorytm optymalnej trasy powinien brać pod uwagę dni przeznaczone na zwiedzanie.
    Algorytm optymalnej trasy powinien brać pod uwagę status atrakcji (np otwarte tylko w środy o konkretnych godzinach).
    Algorytm optymalnej trasy powinien brać pod uwagę warunki pogodowe i przesuwać POI wymagające odpowiedniej kolejności względem siebie jeśli to możliwe.
    Algorytm optymalnej trasy powinien informować użytkownika o zmianie pogody i powinien zaproponować alternatywny plan zwiedzania.}
    \sholder{ UOB02 }
    \reqrelated{F09,F10,F11,F12,F14, F15}
\end{requirementstab}
\begin{requirementstab}[label={tab:requirements:func21},caption={Karta wymagania: Algorytm Plannera - środek transportu rower}]
    \id{F21}
    \priority{C}
    \name{Algorytm Plannera - środek transportu rower}
    \descr{\begin{itemize}
        \item Jako użytkownik 
        \item chce możliwość  wyznaczenia trasy z wykorzystanie roweru / mevo 
        \item ponieważ chciałbym korzystać z swojego roweru bądź wypożyczonego podczas zwiedzania. 
    \end{itemize}
    }
    \acceptcrit{ Aplikacja ma tryb trasy rowerej.}
    \inputdata{Wybrane atrakcje turystyczne – trasa. }
    \preconditions{Przestawia optymalną trase.   }
    \postconditions{ ATrasa zostaje przedstawiona użytkownikowi. }
    \exceptions{ Brak atrakcji brak planu.}
    \implementation{ Algorytm optymalnej trasy powinien brać pod uwagę czas potrzeby na zwiedzenie konkretnej atrakcji.
    Algorytm optymalnej trasy powinien brać pod uwagę czas przeznaczony na zwiedzenia w ciągu danego dnia.
    Algorytm optymalnej trasy powinien brać pod uwagę dni przeznaczone na zwiedzanie.
    Algorytm optymalnej trasy powinien brać pod uwagę status atrakcji (np otwarte tylko w środy o konkretnych godzinach).
    Algorytm optymalnej trasy powinien brać pod uwagę warunki pogodowe i przesuwać POI wymagające odpowiedniej kolejności względem siebie jeśli to możliwe.
    Algorytm optymalnej trasy powinien informować użytkownika o zmianie pogody i powinien zaproponować alternatywny plan zwiedzania.}
    \sholder{ UOB02 }
    \reqrelated{F20}
\end{requirementstab}
\begin{requirementstab}[label={tab:requirements:func22},caption={Karta wymagania: Druk planu zwiedzania }]
    \id{F22}
    \priority{C}
    \name{Widok Druku - Pdf}
    \descr{\begin{itemize}
        \item Jako użytkownik 
        \item chce możliwość prezentacji trasy z wszystkimi punktami w formacie pdf
        \item ponieważ chciałbym mieć na kartce swój plan. 
    \end{itemize}
    }
    \acceptcrit{ Aplikacja umożliwia zapisanie w pdf widoku osi czasu}
    \inputdata{Wybrane atrakcje turystyczne – Zapisana trasa. }
    \preconditions{ Wybrana trasa nie może mieć błedów   }
    \postconditions{ Przełącza na tryb druku wbudowany w urządzenie. }
    \exceptions{ Jeśli urządzenie nie ma możliwości druku przełącza na pobieranie PDF.}
    \implementation{ Widok optymalnej trasy powinien umożliwiać wydruk trasy w formie przystępnej drukarkom 
    }
    \sholder{UOB01, UOB02,UOB03 }
    \reqrelated{F12, F13,F14, F20,F21}
\end{requirementstab}
\begin{requirementstab}[label={tab:requirements:func23},caption={Karta wymagania: Filtracja POI }]
    \id{F23}
    \priority{C}
    \name{Filtracja atrakcji turystycznych}
    \descr{\begin{itemize}
        \item Jako użytkownik 
        \item chce możliwość filtrowanie atrakcji po kategoriach takie jak ilość osób czy nazwa
        \item ponieważ chciałbym mieć dokładniejszej selekcji POI.
    \end{itemize}
    }
    \acceptcrit{ Możliwości wyboru kategorii POI}
    \inputdata{Wybrane atrakcje turystyczne, okres czasowy jaki nas interesuje dotyczące atrakcji. }
    \preconditions{ Włączona lista z wszystkimi POI }
    \postconditions{ Wyświetlaja się potrzebne POI. }
    \exceptions{ Brak dostępnych POI w zadanyum czasie.}
    \implementation{ Możliwość filtrowania poi na podstawie kategorii takich jak ilość osób, cena itd
    }
    \sholder{UOB01, UOB02 }
    \reqrelated{F20,F21,}
\end{requirementstab}
\begin{requirementstab}[label={tab:requirements:func24},caption={Karta wymagania: Automatyczna aktualizacja planu}]
    \id{F24}
    \priority{C}
    \name{Przeliczanie Trasy o aktualną datę i czas}
    \descr{\begin{itemize}
        \item Jako użytkownik 
        \item chce dostawać aktualne (ciągle odświeżane) dane o mojej trasie
        \item ponieważ chciałbym mieć aktualne dane np.  nie będę mógł spełnić całego planu np.. Pójdę coś zjeść.
    \end{itemize}
    }
    \acceptcrit{ Przycisk odświeżenia planu trasy}
    \inputdata{Wybrana trasa turystyczna. }
    \preconditions{ Wybrana trasa była wcześniej włączona }
    \postconditions{ Wyświetla się aktualny plan trasy. }
    \exceptions{ Wyświetla błąd że aktualna atrakcja jest już zamknięta.}
    \implementation{ Widok optymalnej trasy powinien być zaktualizowany o nowe dane.
    }
    \sholder{UOB01, UOB02, UOB03 }
    \reqrelated{F20,F21,}
\end{requirementstab}
\begin{requirementstab}[label={tab:requirements:func25},caption={Karta wymagania: Przeliczanie Trasy o opóźnienia komunikacji}]
    \id{F25}
    \priority{C}
    \name{Przeliczanie Trasy o opóźnienia komunikacji}
    \descr{\begin{itemize}
        \item Jako użytkownik 
        \item chce dostawać aktualne dane o czasie oczekiwania na środek transportu
        \item ponieważ chciałbym mieć aktualne dane ile będę czekać
    \end{itemize}
    }
    \acceptcrit{ Przycisk odświeżenia planu trasy}
    \inputdata{Wybrana trasa turystyczna. }
    \preconditions{ Wybrana trasa była wcześniej włączona }
    \postconditions{ Wyświetla się aktualny plan trasy. }
    \exceptions{ Wyświetla błąd że aktualna atrakcja jest już zamknięta.}
    \implementation{ Widok optymalnej trasy powinien być zaktualizowany o nowe dane.
    }
    \sholder{UOB01, UOB02, UOB03 }
    \reqrelated{F20,F21,}
\end{requirementstab}
\begin{requirementstab}[label={tab:requirements:func26},caption={Karta wymagania: Dostęp bez autoryzacji}]
    \id{F26}
    \priority{C}
    \name{Możliwość korzystania jako gość}
    \descr{\begin{itemize}
        \item Jako użytkownik 
        \item Nie chce logować się do aplikacji
        \item ponieważ chciałbym przetestować aplikacje bez tworzenia konta.
    \end{itemize}
    }
    \acceptcrit{ Dostęp do podstawowych funkcjonalności bez konieczności posiadania konta}
    \inputdata{ Praca bez danych wejściowych }
    \preconditions{ Aplikacja działa bez dodatkowych danych }
    \postconditions{ Użytkownik ma możliwość skorzystania z wybranej funkcji bez konta. }
    \exceptions{ Użytkownik próbuje wykonać akcję, która wymaga posiadania konta – system powinien powiadomić użytkownika o konieczności posiadania konta do wykonania tej akcji.}
    \implementation{ aplikacja dostępna bez logowania
    }
    \sholder{UOB01, UOB02, UOB03 }
    \reqrelated{F20,F21}
\end{requirementstab}
\begin{requirementstab}[label={tab:requirements:func27},caption={Karta wymagania: Panel Moderacji Opinii}]
    \id{F27}
    \priority{C}
    \name{Panel Moderacji Opinii}
    \descr{
     Jako Administrator chce panel do administracyji opinii aby treści były zgodne obowiązującym prawem.
    }
    \acceptcrit{ Możliwość zarządzania opiniami wyświetlenie - akceptowanie - odrzucaniem  }
    \inputdata{ Nowo dodane opinie użytkowników }
    \preconditions{  }
    \postconditions{  }
    \exceptions{ Użytkownik dodał opinie nieodpowiednią dla dzieci. }
    \implementation{ Dodanie kolejnej strony dostepnej tylko dla administratora, w któy będą wyświetlały się nowo dodane opinie oraz wszystkie zaakceptowane. 
    }
    \sholder{ UOB02 }
    \reqrelated{F07}
\end{requirementstab}
\begin{requirementstab}[label={tab:requirements:func28},caption={Karta wymagania: Widok list atrakcji}]
    \id{F28}
    \priority{C}
    \name{Widok zapisanych list atrakcji}
    \descr{Użytkownik dodaje atrakcje do koszyka, liste takich atrakcji może zapisać do ulubionych i jakąś ją nazwać, w celu późniejszego odtworzenia.
    }
    \acceptcrit{ użytkownik dodał liste atracji do ulubionych }
    \inputdata{ Zapisane listy atrakcji }
    \preconditions{  }
    \postconditions{  }
    \exceptions{ Atrakcje z listy są niedostępne }
    \implementation{ Dodanie widoku dla użytkownika, którym będzie mógł zobaczyć dodane listy atrakcji do ulubionych. Widok ten włącza się z menu konta użytkownika.
    }
    \sholder{ UOB01 }
    \reqrelated{F09}
\end{requirementstab}
\section{Interfejs z otoczeniem}
\label{sec:interfejs-z-otoczeniem}

\begin{requirementstab}[label={tab:requirements:env1},caption={Karta wymagania: Komunikacja z API MZK Gdańsk}]
    \id{IO1}
    \priority{S}
    \name{Komunikacja z API MZK Gdańsk}
    \descr{Interfejs pozyskujący dane o komunikacji publicznej miasta Gdańsk}
    \acceptcrit{Pozyskanie aktualnych danych}
    \inputdata{API MZK Gdańsk}
    \preconditions{ Platforma MZK Gdańsk działa}
    \postconditions{ Informacje zostały poprawnie przekazane do plików}
    \exceptions{ API urzędu transportu miejskiego w Gdańsku nie działa poprawnie, system powinien zapisać w logach informację o błędzie w przeładowaniu danych i ponowić działanie w następnym okresie synchronizacji}
    \implementation{ Dane powinny być strumieniowane w formacje JSON a przechowywane w bazie danych}
    \sholder{UOB01, UOB02, UOB03}
    \reqrelated{F11,F20}
\end{requirementstab}
\begin{requirementstab}[label={tab:requirements:env2},caption={Karta wymagania: Integracja z systemem open Weather API}]
    \id{IO2}
    \priority{S}
    \name{Integracja z systemem open Weather API}
    \descr{Cykliczna aktualizacja bazy pogodowej dla Miast z POI np.Gdańska}
    \acceptcrit{Integracja algorytmu planowania trasy na podstawie pogody}
    \inputdata{Miasto np. Miasto Gdańsk}
    \preconditions{ Okres potrzebowania na podróż a}
    \postconditions{ Informacje zostały poprawnie przekazane do plików}
    \exceptions{ Wyświetla błąd iż plan trasy nie został uzwgledniony o pogodę, gdyż wystąpił błąd np. Brak połączenia internetowego}
    \implementation{ Dodanie Klienta Pogodowego, który przechowuje dane z wiarygodnością do 15minut.}
    \sholder{UOB01, UOB02, UOB03}
    \reqrelated{F20}
\end{requirementstab}
\begin{requirementstab}[label={tab:requirements:env3},caption={Karta wymagania: Integracja z systemem overpass API}]
    \id{IO3}
    \priority{M}
    \name{Integracja z systemem overpass API}
    \descr{cykliczna aktualizacja bazy dróg, ścieżek itd w Gdańsku}
    \acceptcrit{Integracja algorytmu planowania trasy na podstawie danych transportu publicznego}
    \inputdata{Graf połączeń pieszych, samochodowych i transportu miejskiego z punktami docelowymi na mapie, 2 wybrane przez użytkownika punkty - początek i koniec}
    \preconditions{ Okres potrzebowania na podróż a}
    \postconditions{ Baza danych zawiera aktualne informacje o ścieżkach w grafie}
    \exceptions{ Brak połączenia między wybranymi punktami, system powinien zwrócić informację o takiej styacji zamiast zakończyć żądanie niepowodzeniem.}
    \implementation{ Cała mapa Gdańska zajmuje ok 100 MB w formacie JSON, system powinien obliczać optymalną trasę na pełnym grafie połączeń, nie ma potrzeby tworzenia pod grafów..}
    \sholder{UOB01, UOB02, UOB03}
    \reqrelated{F20}
\end{requirementstab}
\begin{requirementstab}[label={tab:requirements:env4},caption={Karta wymagania: Implementacja interaktywnej mapy internetowej w oparciu o Open layers    }]
    \id{IO4}
    \priority{S}
    \name{Implementacja interaktywnej mapy internetowej w oparciu o Open layers}
    \descr{Aplikacja zawiera dwuwymiarową mapę, kolorystycznie dostosowaną do trybu ciemnego / jasnego zgodnie z ustawieniami urządzenia użytkownika.}
    \acceptcrit{Mapa zawiera ikony przedstawiające atrakcje turystycznych, po których kliknięciu otworzy się okienko z informacjami o tej atrakcji. Mapa obsługuje urządzenia obsługiwane przy pomocy myszki jak i urządzenia z ekranami dotykowymi.}
    \inputdata{Lista atrakcji turystycznych wraz z informacjami o nich, schemat kolorystyczny urządzenia (jasny / ciemny)}
    \preconditions{ Załadowane dane z Aplikacji}
    \postconditions{ Baza danych zawiera aktualne informacje o ścieżkach w grafie}
    \exceptions{ Brak atrakcji turystycznych – mapa powinna pokazać informację o braku danych.}
    \implementation{ Kafelki mapy w różnych wersjach kolorystycznych zapewni usługa Stadia Maps.}
    \sholder{ UOB02}
    \reqrelated{IO3,F20}
\end{requirementstab}
\begin{requirementstab}[label={tab:requirements:env4},caption={Karta wymagania: Integracja z systemem otwartych danych o dostępnym transporcie }]
    \id{IO4}
    \priority{C}
    \name{Integracja z systemem otwartych danych o transporcie w Gdańsku}
    \descr{Aplikacja zawiera aktualne dane o komunikacji miejskiej w częstotliwości zmian co do 5 minut, dostępnych trasporcie takich jak rowery i hulajnogi w danej lokalizacji}
    \acceptcrit{Mapa podczas tworzenia proponowanej trasy robi możliwe opcje z wykorzystaniem tych informacji.}
    \inputdata{API urzędu transportu miejskiego w Gdańsku działa poprawnie}
    \preconditions{ Załadowane dane z Aplikacji}
    \postconditions{ Docelowa baza danych działa poprawnie}
    \exceptions{ API urzędu nie działa poprawnie, system powinien zapisać w logach informację o błędzie w przeładowaniu danych i ponowić działanie w następnym okresie synchronizacji.}
    \implementation{ Dane powinny być strumieniowane w formacje JSON a przechowywane w bazie danych.}
    \sholder{UOB01, UOB02}
    \reqrelated{IO3,F20}
\end{requirementstab}
\begin{requirementstab}[label={tab:requirements:env5},caption={Karta wymagania: Integracja z systemem otwartych danych o dostępnym transporcie }]
    \id{IO5}
    \priority{C}
    \name{Integracja z systemem otwartych danych o transporcie w Gdańsku}
    \descr{Aplikacja zawiera aktualne dane o komunikacji miejskiej w częstotliwości zmian co do 5 minut, dostępnych trasporcie takich jak rowery i hulajnogi w danej lokalizacji}
    \acceptcrit{Mapa podczas tworzenia proponowanej trasy robi możliwe opcje z wykorzystaniem tych informacji.}
    \inputdata{API urzędu transportu miejskiego w Gdańsku działa poprawnie}
    \preconditions{ Załadowane dane z Aplikacji}
    \postconditions{ Docelowa baza danych działa poprawnie}
    \exceptions{ API nie działa poprawnie, system powinien zapisać w logach informację o błędzie w przeładowaniu danych i ponowić działanie w następnym okresie synchronizacji.}
    \implementation{ Dane powinny być strumieniowane w formacje JSON a przechowywane w bazie danych.}
    \sholder{UOB01, UOB02}
    \reqrelated{IO3,F20}
\end{requirementstab}
\begin{requirementstab}[label={tab:requirements:env6},caption={Karta wymagania: Opcja  do zakupu biletów dla komunikacji miejskiej }]
    \id{IO6}
    \priority{W}
    \name{Opcja przekierowania do zakupu biletów dla komunikacji miejskiej}
    \descr{Po wybraniu trybu transportu z wykorzystaniem Komunikacji miejskiej pojawia się propozycja skorzystania z jakieś aplikacji do zakupu biletów np. skycash}
    \acceptcrit{Opcja transportu przekazuje informacje w formie linka do polecanej aplikacji.}
    \inputdata{Transport publiczny}
    \preconditions{ Użytkownik wybrał tryb nawigacji poprzez transport publiczny}
    \postconditions{ Przekierowanie działa}
    \exceptions{ Brak internetu informacje przekazuje że bez internetu ta funkcja nie działa.}
    \implementation{ Na osi czasu widać przekierowanie do partnerskiej aplikacji}
    \sholder{UOB01, UOB02}
    \reqrelated{IO3,F20}
\end{requirementstab}
\begin{requirementstab}[label={tab:requirements:env7},caption={Karta wymagania: Opcja  do zakupu biletów dla POI }]
    \id{IO7}
    \priority{C}
    \name{Integracja z system biletów konkretnej atrakcji online}
    \descr{cykliczna aktualizacja danych o cennikach danej atrakcji}
    \acceptcrit{Link przekierowujący do strony z sklepem internetowym danej atrakcji.}
    \inputdata{Użytkownik wybrał szczegółowe informacje dotyczące atrakcji}
    \preconditions{ Miejsce z informacją polecanych platformach kupna}
    \postconditions{ Przekierowanie działa}
    \exceptions{ Brak internetu informacje przekazuje że bez internetu ta funkcja nie działa.}
    \implementation{ Na osi czasu widać przekierowanie do partnerskiej aplikacji}
    \sholder{UOB01}
    \reqrelated{IO3,F20}
\end{requirementstab}
\begin{requirementstab}[label={tab:requirements:env8},caption={Karta wymagania: Cykliczna aktualizacja danych POI Miasta Gdańsk}]
    \id{IO8}
    \priority{S}
    \name{Cykliczna aktualizacja danych POI Miasta Gdańsk}
    \descr{cykliczna aktualizacja danych POI Miasta Gdańsk}
    \acceptcrit{Link przekierowujący do strony z sklepem internetowym danej atrakcji.}
    \inputdata{Nazwa , czasy otwarcia }
    \preconditions{ Dla podstawowych poi system powinien implementować web scrappera który automatycznie będzie aktualizować godziny otwarcia oraz stan atrakcji}
    \postconditions{ Integracje przewiduje 10 ATRAKCJI Najbardziej popularnych w gdańsku}
    \exceptions{Jeśli scapper nie działą przedstawia ostatnią aktualną informacje.}
    \implementation{ Scrapujemy strone z potrzebnymi dla nas danymi i przesyłamy HTML do Chata GPT. 
    Chat GPT generuje Tabele z potrzebnymi dla nas danymi, według ustalonego schematu dla bazy danych.
    Następnie Administrator akceptuje zmiany.
    }
    \sholder{UOB02}
    \reqrelated{IO3,F20}
\end{requirementstab}
\begin{requirementstab}[label={tab:requirements:env9},caption={Karta wymagania: Informacja o aktualnych danych POI }]
    \id{IO9}
    \priority{S}
    \name{Informacja o aktualnych danych POI - potrzebna aktualizacja}
    \descr{Wykorzystując pobieranie strony HTML, (sprawdzamy co godzine zmiany), powiadamy Administratora w Panelu o potrzebie zaaktulizowania danych}
    \acceptcrit{Link przekierowujący do strony z sklepem internetowym danej atrakcji.}
    \inputdata{Nazwa , czasy otwarcia }
    \preconditions{ Dla podstawowych poi system powinien pokazywać potrzebe informacji o zmianach}
    \postconditions{ Integracje przewiduje atrakcje w Gdańsku}
    \exceptions{Jeśli scapper nie działa przedstawia ostatnią aktualną informacje. - Brak zmian}
    \implementation{ Scrapujemy strone z potrzebnymi dla nas danymi i jeśli strona różni się od poprzedniego stanu informujemy o tym}
    \sholder{UOB02}
    \reqrelated{IO3,F20}
\end{requirementstab}
\section{Wymagania niefunkcjonalne}
\label{sec:wymagania-niefunkcjonalne}

\begin{requirementstab}[label={tab:requirements:nonfunc1},caption={Karta wymagania: Dostęp 24/7}]
    \id{NFO1}
    \priority{S}
    \name{Dostęp 24/7}
    \descr{System powinien być dostępny 7 dni w tygodniu, 24 godziny na dobę. Na potrzeby prac utrzymaniowych dopuszcza się niedostępność max 8h w skali miesiąca w godzinach nocnych (SLA 98.905 \%) }
    \acceptcrit{maksymalny czas niedostępności systemu w skali miesiąca to 8h w godzinach 22:00-06:00}
    \sholder{UOB01, UOB02}
    \reqrelated{}
\end{requirementstab}
\begin{requirementstab}[label={tab:requirements:nonfunc2},caption={Karta wymagania: Responsywność Systemu na mobilnych urządzeniach}]
    \id{NFO1}
    \priority{M}
    \name{System musi być intuicyjny oraz responsywny na urządzeniach z małym ekranem}
    \descr{System używany na urządzeniach mobilnych, musi charakteryzować się wysoką responsywnością}
    \acceptcrit{Wynik Lighthouse w kategorii performance na poziomie co najmniej 90\%}
    \sholder{UOB02}
    \reqrelated{WO5}
\end{requirementstab}

\section{Wymagania dotyczące procesu wytwarzania}
\label{sec:wymagania-dotyczace-procesu-wytwarzania}

\begin{requirementstab}[label={tab:requirements:envfunc1},caption={Karta wymagania: Android}]
    \id{ŚDO1}
    \priority{C}
    \name{Android (PWA)}
    \descr{W dniu premiery musi być dostępny na systemy mobilne Android }
    \acceptcrit{System będzie działał sprawnie na systemy Android min. 2 dni przed premierą}
    \sholder{UOB02}
    \reqrelated{}
\end{requirementstab}
\begin{requirementstab}[label={tab:requirements:envfunc2},caption={Karta wymagania: Przeglądarka internetowa}]
    \id{ŚDO2}
    \priority{M}
    \name{Przeglądarka internetowa}
    \descr{W dniu premiery musi być dostępny na urządzenia mobilne  }
    \acceptcrit{System będzie działał sprawnie na wszystkich przeglądarkach internetowych 
    Interaktywna mapa wraz ze znacznikami poi powinna być responsywna i działać prawidłowo na urządzeniach z ekranem dotykowym jak i urządzeniach ze wskaźnikiem analogowym (myszka)
    }
    \sholder{UOB01, UOB02,UOB03}
    \reqrelated{}
\end{requirementstab}



\restoregeometry % Przywrócenie poprzednich marginesów