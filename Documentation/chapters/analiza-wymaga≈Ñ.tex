%! Author = Wiktor Rostkowski
%! Date = 28/05/2024

\chapter{Analiza wymagań}
\label{ch:analiza-wymagan}

\section{Sposób gromadzenia wymagań}
\label{sec:sposob-gromadzenia-wymagan}

Wymagania dla projektu zostały zebrane w pierwszym semestrze. Zostały określone na podstawie analizy funkcjonowania

Na podstawie opracowanego dokumentu SWS, przygotowano wymagania dotyczące funkcjonalności projektu. Wykorzystana została metodyka priorytetyzacji wymagań MoSCoW\footnote{Metoda MoSCoW – Must, Should, Could, Won't}[]:
\begin{itemize}
    \item \textbf{M – must (musi być)} – wymaganie bezkompromisowo musi zostać zrealizowane, bez niego projekt nie zostanie ukończony;
    \item \textbf{S – should (powinno być)} – wymaganie powinno być zrealizowane, jeżeli tylko jest taka możliwość;
    \item \textbf{C – could (może być)} – wymaganie powinno być zawarte w projekcie, jeśli wystarczy na nie czasu;
    \item \textbf{W – won't (nie będzie)} – wymaganie nie powinno być zawarte w tym wydaniu projektu, ale może być zrealizowane w przyszłości.
\end{itemize}


\section{Aktorzy}
\label{sec:aktorzy}

\begin{stakeholder}[label={tab:stakeholder:someholder},caption={opis udzialowca}]
    \id{UOB 01}
    \name{Podstawowy użytkownik aplikacji (np. Turysta) }
    \descr{Osoba korzystająca z aplikacji w celu poznawania PoI i optymalnego przemieszczania się między nimi }
    \type{Ożywiony bezpośredni}
    \viewpoint{Użytkownik - operator}
    \limitations{Widzi tylko widok końcowy}
    \requ{-----------------}
\end{stakeholder}

\begin{stakeholder}[label={tab:stakeholder:someholder},caption={Opis udzialowca}]
    \id{UOB 02}
    \name{Zespół Projektowy}
    \descr{Osoba korzystająca z aplikacji w celu poznawania PoI i optymalnego przemieszczania się między nimi}
    \type{Ożywiony bezpośredni}
    \viewpoint{Techniczny}
    \limitations{Brak}
    \requ{-----------------}
\end{stakeholder}

\begin{stakeholder}[label={tab:stakeholder:someholder},caption={Opis udzialowca}]
    \id{UOB 03}
    \name{Firmy o charakterze najmu krótko-terminowym }
    \descr{Firmy o charakterze najmu krótko-terminowego mają bezpośredni kontakt z turystami}
    \type{Ożywiony niebezpośredni }
    \viewpoint{Współpraca z Turystami}
    \limitations{Brak}
    \requ{-----------------}
\end{stakeholder}

\section{Wymagania ogólne i dziedzinowe}
\label{sec:wymagania-ogolne-i-dziedzinowe}


\begin{requirementstab}[label={tab:requirements:general1},caption={Wymaganie ogólne i dziedzinowe}]
    \id{WO1}
    \priority{M}
    \name{Obsługa użytkowników}
    \descr{System powinien działać bez większych przestojów i przerw w funkcjonowaniu, nawet przy dużej liczbie użytkowników. }
    \sholder{UOB2}
    \reqrelated{}
\end{requirementstab}
\begin{requirementstab}[label={tab:requirements:general2},caption={Wymaganie ogólne i dziedzinowe}]
    \id{WO2}
    \priority{C}
    \name{Działanie w czasie rzeczywistym Offline lub Online }
    \descr{Aplikacja musi działać i synchronizować dane do jak najmniejszej możliwej jednostki czasowej. 
    Wersja offline wykorzystuje ostatnie zapisane informacje }
    \sholder{UOB2}
    \reqrelated{}
\end{requirementstab}
\begin{requirementstab}[label={tab:requirements:general3},caption={Wymaganie ogólne i dziedzinowe}]
    \id{WO3}
    \priority{M}
    \name{Zarządzanie danymi z aplikacji}
    \descr{Możliwość sprawdzanie aktualnych danych poprzez panel aplikacji internetowej (Panel Administracyjny) }
    \sholder{UOB2}
    \reqrelated{}
\end{requirementstab}
\begin{requirementstab}[label={tab:requirements:general4},caption={Wymaganie ogólne i dziedzinowe}]
    \id{WO4}
    \priority{M}
    \name{Licencje i prawa}
    \descr{Wszystkie technologie, narzędzia, wzorce itp. potrzebne do wykonania systemu muszą być licencjonowane oraz od głównych dostawców nie pośredników. }
    \sholder{UOB2}
    \reqrelated{}
\end{requirementstab}
\begin{requirementstab}[label={tab:requirements:general5},caption={Wymaganie ogólne i dziedzinowe}]
    \id{WO5}
    \priority{M}
    \name{Wyglad interface responsywny do standardowych urządzeń }
    \descr{System powinien działać na wielu platformach. 
    System powinien wyglądać estetycznie, poukładanie i działać płynnie, bez większych opóźnień w reakcji na zdarzenie.  }
    \sholder{UOB2}
    \reqrelated{}
\end{requirementstab}
\begin{requirementstab}[label={tab:requirements:general6},caption={Wymaganie ogólne i dziedzinowe}]
    \id{WO6}
    \priority{M}
    \name{Rejestracja użytkowników  }
    \descr{Rejestracja powinna przebiegać sprawnie i wraz z wszystkimi standardami i prawami.  }
    \sholder{UOB2, UOB1}
    \reqrelated{}
\end{requirementstab}
\begin{requirementstab}[label={tab:requirements:general7},caption={Wymaganie ogólne i dziedzinowe}]
    \id{WO7}
    \priority{M}
    \name{Komercjalizacji (reklamy) }
    \descr{W przyszłości system może wykorzystać system płatności abonamentowej w celu całkowitego usunięcia z niego reklam.   }
    \sholder{UOB01, UOB02, UOB03}
    \reqrelated{}
\end{requirementstab}

\section{Wymagania funkcjonalne}
\label{sec:wymagania-funkcjonalne}
\begin{requirementstab}[label={tab:requirements:func1},caption={Tabela z wymaganiami funkcjonalnymi}]
    \id{FO1}
    \priority{M}
    \name{Przeglądanie Widoku mapy z pinezkami }
    \descr{Jako turysta chciałbym zobaczyć Mapę wszystkich atrakcji dostępnych w mieście z lotu ptaka 
    }
    \acceptcrit{Użytkownik może zobaczyć aktualne dostępne atrakcje dostępne w danym mieście }
    \inputdata{Użytkownik logując się widzi przedstawioną Mapę}
    \preconditions{ Po otwarciu aplikacji PWA przestawi się zachęcają Mapa  }
    \postconditions{ Wyświetlenie zaznaczonych atrakcji. }
    \exceptions{ Wyświetlanie komunikatu w przypadku niedostępności atrakcji w aktualnej chwili oraz przedstawienia propozycji zmiany punktu docelowego. }
    \implementation{ Użytkownik wykorzystując przeglądarkę bądź aplikacje widzi Dostępne Atrakcje ogólnie dostępnych w danym mieście. }
    \sholder{UOB01, UOB02 }
    \reqrelated{}
\end{requirementstab}
\begin{requirementstab}[label={tab:requirements:func2},caption={Tabela z wymaganiami funkcjonalnymi}]
    \id{FO2}
    \priority{C}
    \name{Synchronizacja danych i wyświetlanie powiadomień o zmianach. }
    \descr{\begin{itemize}
        \item Jako Turysta 
        \item chcę dostawać powiadomienia o zmianach w planie,
        \item bo wtedy nie spóźnię się na czas otwarcia atrakcji. 
    \end{itemize}
    }
    \acceptcrit{Powiadomienie będzie zgodnie z aktualnym planem  }
    \inputdata{Aplikacja pamięta jakie użytkownik wybrał atrakcje do odwiedzenia i informuje o zmianach po synchronizacji. }
    \preconditions{ Użytkownik był wcześniej odwiedził stronę i posiada zapisane atrakcje.  }
    \postconditions{ Aplikacja przedstawi nowe zmiany po kliknięciu w powiadomienie.   }
    \exceptions{ Użytkownik nie zezwolił powiadomienia bądź nowe dane nie istnieją. }
    \implementation{ Na planie zajęć wyświetlają się nowe zmiany w innym kolorze  }
    \sholder{UOB01, UOB02 }
    \reqrelated{}
\end{requirementstab}
\begin{requirementstab}[label={tab:requirements:func3},caption={Tabela z wymaganiami funkcjonalnymi}]
    \id{FO3}
    \priority{M}
    \name{Zakładka z przestawionymi dostępnymi wszystkimi atrakcjami  }
    \descr{\begin{itemize}
        \item Jako Użytkownik 
        \item Chce korzystać z aplikacji z listą dostępnych atrakcji 
        \item Ponieważ chce zobaczyć dostępne spektrum możlwości. 
    \end{itemize}
    }
    \acceptcrit{Lista dostępnych punktów. }
    \inputdata{Brak}
    \preconditions{ Użytkownik wszedł na strone z dostępnymi atrakcjami lub na Mape z atrakcjami.  }
    \postconditions{ Użytkownik może przeczytać lub klinknąć na interesujące go atrakcje.   }
    \exceptions{ Brak grupy, której szuka użytkownik. Użytkownik musi użyć regularnego planu zajęć.   }
    \implementation{ Aplikacja pobiera w tle plik bazy danych ważący 1MB.  }
    \sholder{UOB01, UOB02, UOB03 }
    \reqrelated{}
\end{requirementstab}
\begin{requirementstab}[label={tab:requirements:func4},caption={Tabela z wymaganiami funkcjonalnymi}]
    \id{FO4}
    \priority{M}
    \name{Panel zarządzania atrakcjami w aplikacji. }
    \descr{\begin{itemize}
        \item Jako Administrator 
        \item mieć panel administracyjny do wymuszania odświeżenia godzin otwarcia, edycja opisu itd 
        \item Ponieważ  potrzebuje mieć wygodny interfejs do edycji atrakcji dostępnych na mapie. 
    \end{itemize}
    }
    \acceptcrit{Lista dostępnych punktów. }
    \inputdata{Konto z uprawnieniami Administracyjnymi}
    \preconditions{ Użytkownik wszedł na strone z dostępnymi atrakcjami lub na Mape z atrakcjami.  }
    \postconditions{ Można edytować/usuwać/ dodawać recznie atrakcje.   }
    \exceptions{ Jest możliwość dodawania kolejnych atrakcji .   }
    \implementation{ Panel administracyjny do zarządzania opiniami użytkowników (umożliwia cenzurę.).   }
    \sholder{ UOB02 }
    \reqrelated{W03}
\end{requirementstab}
\begin{requirementstab}[label={tab:requirements:func5},caption={Tabela z wymaganiami funkcjonalnymi}]
    \id{FO5}
    \priority{M}
    \name{Znaczniki na interaktywnej mapie informujące o punktach zainteresowania.}
    \descr{\begin{itemize}
        \item Jako Użytkownik 
        \item Chce móc wyświetlić wybrane atrakcje, które mnie interesują
        \item Ponieważ umożliwi to spojrzenie na opis POI. 
    \end{itemize}
    }
    \acceptcrit{Mapa posiada znaczniki atrakcji dodanych poprzez Administratora/. }
    \inputdata{Użytkownik wyświetla zakładkę z mapą atrakcji. }
    \preconditions{ Mapa Atrakcji z znacznikami dodanymi przez twórców aplikacji.}
    \postconditions{ Wyświetlenie wszystkie atrakcje turystyczne plus atrakcje dodane przez użytkowników  }
    \exceptions{ Aplikacja wyświetla błąd w przerwie transmisji aplikacji.    }
    \implementation{ Tu musi być wybór pois i podanie czasu na zwiedzanie.    }
    \sholder{ UOB01, UOB02, UOB03  }
    \reqrelated{}
\end{requirementstab}
\begin{requirementstab}[label={tab:requirements:func6},caption={Tabela z wymaganiami funkcjonalnymi}]
    \id{FO6}
    \priority{C}
    \name{Możliwość oceniania atrakcji turystycznych .}
    \descr{\begin{itemize}
        \item Jako Użytkownik 
        \item Chce mieć możliwość oceniania atrakcjami gwiazdkami 1-5
        \item Ponieważ umożliwi to dodawania indywidualnych ocenaniai i ich edytowania. 
    \end{itemize}
    }
    \acceptcrit{Każda atrakcja posiada możliwośc dodawania opinii. }
    \inputdata{Nazwa użytkownika, mail. }
    \preconditions{ Można dodawać opinie liczbową.}
    \postconditions{ Wyświetla tylko średnią z opinii, brak możliwości komentowania.  }
    \exceptions{ Do każdej atrakcji po zalogowaniu użytkownik widzi opcje dodaj opinie.    }
    \implementation{ Przy danym POI po kliknięciu jest dostępny przycisk z opcją opinia oraz powyżej średnia z opini, każdy użytkownik może dodać jedną opinie w gwiazdkach od 1 do 5.    }
    \sholder{ UOB01  }
    \reqrelated{}
\end{requirementstab}
\begin{requirementstab}[label={tab:requirements:func7},caption={Tabela z wymaganiami funkcjonalnymi}]
    \id{FO7}
    \priority{C}
    \name{Możliwość pisania opinii wraz ze zdjęciami pod punktami zainteresowania .}
    \descr{\begin{itemize}
        \item Jako Użytkownik 
        \item Chce mieć możliwość pisania pisemnych opinii z możliwością dodania zdjecia
        \item Ponieważ umożliwi to dodawania indywidualnych opinii atrakcji i ich edytowania. 
    \end{itemize}
    }
    \acceptcrit{Opinia musi zostać zatwierdzona przez administrator, może póżniejszym etapie na podstawie algorytmu, który spełnia jakąś polityke prywatności. }
    \inputdata{Nazwa użytkownika, mail. }
    \preconditions{ Można dodawać opinie (ewentualnie ze zdjeciem).}
    \postconditions{ Brak możliwości komentowania, wyświetla się opinia pod atrakcją. }
    \exceptions{ Do każdej atrakcji po zalogowaniu użytkownik widzi opcje dodaj opinie.    }
    \implementation{ Przy danym POI po kliknięciu jest dostępny przycisk z opcją napisania opinia , każdy użytkownik może dodać jedną opinie.
    Opinia musi zostać zatwierdzona przez moderatora, nowe opinie wyświetlają się w panelu administratora.    }
    \sholder{ UOB01  }
    \reqrelated{}
\end{requirementstab}
\begin{requirementstab}[label={tab:requirements:func8},caption={Tabela z wymaganiami funkcjonalnymi}]
    \id{FO8}
    \priority{S}
    \name{Zapisywanie list wybranych pois pod konkretną nazwą.}
    \descr{\begin{itemize}
        \item Jako Użytkownik 
        \item Chce możliwość zapisania wybranej trasy
        \item Ponieważ umożliwi to przygotowanie trasy na później. 
    \end{itemize}
    }
    \acceptcrit{ Po wybraniu trasy, która zostanie przedstawiona pokazuje się dymek dodaj do swoich tras}
    \inputdata{Wybrane atrakcje turystyczne z mapy bądź listy atrakcji,  }
    \preconditions{ Wpis do zakładki ostatnich wybranych tras. }
    \postconditions{ Odpowiednia zakładka z listą Zapisane Trasy.  }
    \exceptions{  Jeśli użytkownik jest offline(lub nie zalogowany) wyskakuje błąd żeby dodać trase trzeba być zalogowanym online.   }
    \implementation{ Możliwość otworzenia zapisanej optymalnej trasy bez konieczności ponownego jej obliczania.
    Możliwość otworzenia zapisanej trasy bez połączenia internetowego.
    
        }
    \sholder{ UOB01, UOB02   }
    \reqrelated{}
\end{requirementstab}

\begin{requirementstab}[label={tab:requirements:func9},caption={Tabela z wymaganiami funkcjonalnymi}]
    \id{FO9}
    \priority{C}
    \name{Udostępnianie zapisanych list pois pomiędzy użytkownikami}
    \descr{\begin{itemize}
        \item Jako Użytkownik 
        \item Chce mieć możliwość udostępnienia wybranej trasy
        \item Ponieważ umożliwi to polecania wybranych trasy. 
    \end{itemize}
    }
    \acceptcrit{ W zakładce zapisanych listy widniej dymek z udostępnię danej trasy}
    \inputdata{Na podstawie zapisanej trasy jest możliwość wysłania krótko trwałego kodu trasy.  }
    \preconditions{ Link do zapisanej trasy }
    \postconditions{ Przesłania linku bądź kodu dla drugiego użytkownika w celu dodania tej trasy u siebie  }
    \exceptions{  Wszystkie trasy udostępnione są dostępne online dla wszystkich użytkowników Internetu.  }
    \implementation{ Każda trasa ma indywidualny kod w bazie danych
        }
    \sholder{ UOB01, UOB02, UOB03   }
    \reqrelated{F10}
\end{requirementstab}
\begin{requirementstab}[label={tab:requirements:func10},caption={Tabela z wymaganiami funkcjonalnymi}]
    \id{F10}
    \priority{M}
    \name{Udostępnianie zapisanych list pois pomiędzy użytkownikami}
    \descr{\begin{itemize}
        \item Jako Użytkownik 
        \item Chce mieć możliwość przejścia wybranej trasy tylko na pieszo
        \item Ponieważ umożliwi to to wykorzystanie tylko pieszych dróżek. 
    \end{itemize}
    }
    \acceptcrit{ Algorytm wyznacza trase od wybranego punktu pierwszego do punktu ostatniego .}
    \inputdata{Wybrane atrakcje turystyczne - trasę.  }
    \preconditions{ Przedstawienie trasy podstawie wybranych punktów. }
    \postconditions{ Trasa przedstawia w sposób edytowalny dla użytkownika  }
    \exceptions{ Wybrano punkt, który w dostępnym czasie jest nie dostępny wyświetlenia godzin otwarcia danego punktu z sugestią zmiany czasu  }
    \implementation{ *************************************************
        }
    \sholder{ UOB01, UOB02   }
    \reqrelated{F09}
\end{requirementstab}
\begin{requirementstab}[label={tab:requirements:func11},caption={Tabela z wymaganiami funkcjonalnymi}]
    \id{F11}
    \priority{S}
    \name{Wyznaczanie optymalnej trasy z udziałem komunikacji miejskiej (tylko 1 wariant)}
    \descr{\begin{itemize}
        \item Jako Użytkownik 
        \item Chce mieć możliwość przejścia wybranej trasy pieszo oraz aktualnej komunikacji miejskiej
        \item Ponieważ umożliwi to wykorzystanie optymalnego połączenia drogi pieszej jak i uaktualnianej komunikacji miejskiej. 
    \end{itemize}
    }
    \acceptcrit{ Możliwość wyboru trasy z transportem komunikacji miejskiej.}
    \inputdata{Wybrane atrakcje turystyczne - trasę.  }
    \preconditions{ Przedstawienie trasy podstawie wybranych punktów. }
    \postconditions{ Algorytm zaprezentował  trase z wykorzystaniem komunikacji miejskiej  }
    \exceptions{ Zaproponowanie innego transportu np. pieszo  }
    \implementation{ *************************************************
        }
    \sholder{ UOB01, UOB02   }
    \reqrelated{F09,F10}
\end{requirementstab}
\begin{requirementstab}[label={tab:requirements:func11},caption={Tabela z wymaganiami funkcjonalnymi}]
    \id{F11}
    \priority{S}
    \name{Wyznaczanie optymalnej trasy z udziałem komunikacji miejskiej (tylko 1 wariant)}
    \descr{\begin{itemize}
        \item Jako Użytkownik 
        \item Chce mieć możliwość przejścia wybranej trasy pieszo oraz aktualnej komunikacji miejskiej
        \item Ponieważ umożliwi to wykorzystanie optymalnego połączenia drogi pieszej jak i uaktualnianej komunikacji miejskiej. 
    \end{itemize}
    }
    \acceptcrit{ Możliwość wyboru trasy z transportem komunikacji miejskiej.}
    \inputdata{Wybrane atrakcje turystyczne - trasę.  }
    \preconditions{ Przedstawienie trasy podstawie wybranych punktów. }
    \postconditions{ Algorytm zaprezentował  trase z wykorzystaniem komunikacji miejskiej  }
    \exceptions{ Zaproponowanie innego transportu np. pieszo  }
    \implementation{ *************************************************
        }
    \sholder{ UOB01, UOB02   }
    \reqrelated{F09,F10}
\end{requirementstab}











\section{Interfejs z otoczeniem}
\label{sec:interfejs-z-otoczeniem}

TODO

\section{Wymagania niefunkcjonalne}
\label{sec:wymagania-niefunkcjonalne}

TODO

\section{Wymagania dotyczące procesu wytwarzania}
\label{sec:wymagania-dotyczace-procesu-wytwarzania}

TODO

\section{Wymagania jakościowe i inne}
\label{sec:wymagania-jakosciowe-i-inne}
