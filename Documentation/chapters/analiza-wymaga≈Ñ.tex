%! Author = Mateusz Budzisz
%! Date = 26/05/2024

\chapter{Analiza wymagań}
\label{ch:analiza-wymagan}

\section{Sposób gromadzenia wymagań}
\label{sec:sposob-gromadzenia-wymagan}

Wymagania dla projektu zostały zebrane w pierwszym semestrze. Zostały określone na podstawie analizy funkcjonowania

\section{Aktorzy}
\label{sec:aktorzy}

TODO

\section{Wymagania ogólne i dziedzinowe}
\label{sec:wymagania-ogolne-i-dziedzinowe}

\begin{requirementstab}[label={tab:requirements:general},caption={Przykładowe wymaganie ogólne lub dziedzinowe}]
    \id{WO1}
    \priority{M}
    \name{krótki opis}
    \descr{opis szczegółowy, należy dążyć do tego, żeby wszystkie znane na ten moment szczegóły wymagania zostały wydobyte i wyspecyfikowane}
    \sholder{nazwa udziałowca, który podał wymaganie}
    \reqrelated{wymagania zależne i uszczegóławiające – odesłanie poprzez identyfikator}
\end{requirementstab}

\section{Wymagania funkcjonalne}
\label{sec:wymagania-funkcjonalne}

TODO

\section{Interfejs z otoczeniem}
\label{sec:interfejs-z-otoczeniem}

TODO

\section{Wymagania niefunkcjonalne}
\label{sec:wymagania-niefunkcjonalne}

TODO

\section{Wymagania dotyczące procesu wytwarzania}
\label{sec:wymagania-dotyczace-procesu-wytwarzania}

TODO

\section{Wymagania jakościowe i inne}
\label{sec:wymagania-jakosciowe-i-inne}
