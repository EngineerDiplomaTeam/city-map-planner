%! Author = Mateusz Budzisz
%! Date = 26/05/2024

\chapter{Analiza wymagań}
\label{ch:analiza-wymagan}

\section{Sposób gromadzenia wymagań}
\label{sec:sposob-gromadzenia-wymagan}

Wymagania dla projektu zostały zebrane w pierwszym semestrze. Zostały określone na podstawie analizy funkcjonowania

Na podstawie opracowanego dokumentu SWS, przygotowano wymagania dotyczące funkcjonalności projektu. Wykorzystana została metodyka priorytetyzacji wymagań MoSCoW\footnote{Metoda MoSCoW – Must, Should, Could, Won't}[]:
\begin{itemize}
    \item \textbf{M – must (musi być)} – wymaganie bezkompromisowo musi zostać zrealizowane, bez niego projekt nie zostanie ukończony;
    \item \textbf{S – should (powinno być)} – wymaganie powinno być zrealizowane, jeżeli tylko jest taka możliwość;
    \item \textbf{C – could (może być)} – wymaganie powinno być zawarte w projekcie, jeśli wystarczy na nie czasu;
    \item \textbf{W – won't (nie będzie)} – wymaganie nie powinno być zawarte w tym wydaniu projektu, ale może być zrealizowane w przyszłości.
\end{itemize}


\subsection{Aktorzy}
\label{sec:aktorzy}

\begin{stakeholder}[label={tab:stakeholder:someholder},caption={opis udzialowca}]
    \id{UOB 01}
    \name{Podstawowy użytkownik aplikacji (np. Turysta) }
    \descr{Osoba korzystająca z aplikacji w celu poznawania PoI i optymalnego przemieszczania się między nimi }
    \type{Ożywiony bezpośredni}
    \viewpoint{Użytkownik - operator}
    \limitations{Widzi tylko widok końcowy}
    \requ{-----------------}
\end{stakeholder}

\begin{stakeholder}[label={tab:stakeholder:someholder},caption={Opis udzialowca}]
    \id{UOB 02}
    \name{Zespół Projektowy}
    \descr{Osoba korzystająca z aplikacji w celu poznawania PoI i optymalnego przemieszczania się między nimi}
    \type{Ożywiony bezpośredni}
    \viewpoint{Techniczny}
    \limitations{Brak}
    \requ{-----------------}
\end{stakeholder}

\begin{stakeholder}[label={tab:stakeholder:someholder},caption={Opis udzialowca}]
    \id{UOB 03}
    \name{Firmy o charakterze najmu krótko-terminowym }
    \descr{Firmy o charakterze najmu krótko-terminowego mają bezpośredni kontakt z turystami}
    \type{Ożywiony niebezpośredni }
    \viewpoint{Współpraca z Turystami}
    \limitations{Brak}
    \requ{-----------------}
\end{stakeholder}

\subsection{Wymagania ogólne i dziedzinowe}
\label{sec:wymagania-ogolne-i-dziedzinowe}

\begin{requirementstab}[label={tab:requirements:general},caption={Wymaganie ogólne i dziedzinowe}]
    \id{WO1}
    \priority{M}
    \name{Obsługa użytkowników}
    \descr{System powinien działać bez większych przestojów i przerw w funkcjonowaniu, nawet przy dużej liczbie użytkowników. }
    \sholder{UOB2}
    \reqrelated{}
\end{requirementstab}


\begin{requirementstab}[label={tab:requirements:general},caption={Wymaganie ogólne i dziedzinowe}]
    \id{WO2}
    \priority{C}
    \name{Działanie w czasie rzeczywistym Offline lub Online }
    \descr{Aplikacja musi działać i synchronizować dane do jak najmniejszej możliwej jednostki czasowej. 
    Wersja offline wykorzystuje ostatnie zapisane informacje }
    \sholder{UOB2}
    \reqrelated{}
\end{requirementstab}

\begin{requirementstab}[label={tab:requirements:general},caption={Wymaganie ogólne i dziedzinowe}]
    \id{WO3}
    \priority{M}
    \name{Zarządzanie danymi z aplikacji}
    \descr{Możliwość sprawdzanie aktualnych danych poprzez panel aplikacji internetowej (Panel Administracyjny) }
    \sholder{UOB2}
    \reqrelated{}
\end{requirementstab}


\begin{requirementstab}[label={tab:requirements:general},caption={Wymaganie ogólne i dziedzinowe}]
    \id{WO4}
    \priority{M}
    \name{Licencje i prawa}
    \descr{Wszystkie technologie, narzędzia, wzorce itp. potrzebne do wykonania systemu muszą być licencjonowane oraz od głównych dostawców nie pośredników. }
    \sholder{UOB2}
    \reqrelated{}
\end{requirementstab}

\begin{requirementstab}[label={tab:requirements:general},caption={Wymaganie ogólne i dziedzinowe}]
    \id{WO5}
    \priority{M}
    \name{Wyglad interface responsywny do standardowych urządzeń }
    \descr{System powinien działać na wielu platformach. 
    System powinien wyglądać estetycznie, poukładanie i działać płynnie, bez większych opóźnień w reakcji na zdarzenie.  }
    \sholder{UOB2}
    \reqrelated{}
\end{requirementstab}

\begin{requirementstab}[label={tab:requirements:general},caption={Wymaganie ogólne i dziedzinowe}]
    \id{WO6}
    \priority{M}
    \name{Rejestracja użytkowników  }
    \descr{Rejestracja powinna przebiegać sprawnie i wraz z wszystkimi standardami i prawami.  }
    \sholder{UOB2, UOB1}
    \reqrelated{}
\end{requirementstab}


\begin{requirementstab}[label={tab:requirements:general},caption={Wymaganie ogólne i dziedzinowe}]
    \id{WO7}
    \priority{M}
    \name{Komercjalizacji (reklamy) }
    \descr{W przyszłości system może wykorzystać system płatności abonamentowej w celu całkowitego usunięcia z niego reklam.   }
    \sholder{UOB01, UOB02, UOB03}
    \reqrelated{}
\end{requirementstab}










\subsection{Wymagania funkcjonalne}
\label{sec:wymagania-funkcjonalne}

TODO

\subsection{Interfejs z otoczeniem}
\label{sec:interfejs-z-otoczeniem}

TODO

\subsection{Wymagania niefunkcjonalne}
\label{sec:wymagania-niefunkcjonalne}

TODO

\subsection{Wymagania dotyczące procesu wytwarzania}
\label{sec:wymagania-dotyczace-procesu-wytwarzania}

TODO

\subsection{Wymagania jakościowe i inne}
\label{sec:wymagania-jakosciowe-i-inne}
