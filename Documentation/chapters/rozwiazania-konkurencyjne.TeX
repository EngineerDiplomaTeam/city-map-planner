\chapter{Rozwiązania konkurencyjne}
\label{ch:rozwiązania-konkurencyjne}

\paragraph{Wstęp}

Na rynku dostępne są obecnie różne rozwiązania mające na celu ułatwienie planowania podróży i posiadające różny zakres 
funkcjonalności. Użytkownik może napotkać rozbudowane aplikacje integrujące wiele funkcjonalności takich jak układanie list 
atrakcji do zwiedzania, optymalizacja tras podróży, dokumentowanie poniesionych wydatków, tworzenie list ułatwiających planowanie 
(listy czynności do wykonania, listy zakupów, listy ułatwiające organizację pakowania), synchronizacja naszych planów podróży 
ze znajomymi, integrowanie rezerwacji hotelowych i lotów. Poniżej przedstawiono przykłady najpopularniejszych obecnie rozwiązań 
w tej dziedzinie.

\paragraph{Wanderlog}

Jest to aplikacja reklamująca się jako planer podróży, ze szczególnym naciskiem na organizowanie wakacji oraz wycieczek samochodem.
W wersji podstawowej jest darmowa, posiada również płatną wersję premium  oferującą dodatkowe funkcjonalności.
Aplikacja jest dostępna poprzez przeglądarkę internetową jak również poprzez dedykowaną aplikację na urządzenia mobilne. 
Wanderlog umożliwia układanie list z interesującymi użytkownika miejscami i wydarzeniami, które są graficznie przedstawione w
postaci pinezek na mapie google. Po wybraniu lokalizacji i daty podróży, użytkownik może dokonać przeglądu ofert noclegów. 
Aplikacja umożliwia tworzenie planów podróży razem z innymi użytkownikami oraz ich synchronizację w czasie rzeczywistym. 
Ponadto użytkownik ma dostęp do spersonalizowanych sugestii.

\paragraph{TripIt}

Jest to planer podróży integrujący wiele funkcjonalności w celu maksymalnego ułatwienia użytkownikowi procesu podróżowania. 
Stanowi alternatywę dla wspomnianej wcześniej aplikacji Wanderlog i również posiada darmową wersję podstawową oraz wersję płatną 
opartą na modelu subskrypcyjnym, która posiada dodatkowe funkcjonalności. W wersji podstawowej użytkownik ma możliwość układania 
planów podróży, które są dostępne na wielu urządzeniach jednocześnie. Udostępnia statystyki, wytyczne dotyczące restrykcji COVID-19
oraz umożliwia dodawanie zdjęć, kodów QR oraz plików PDF do planów podróży. Aplikacja zapewnia nawigację między punktami,
mapy lotnisk, sugestie co do intersujących miejsc blisko lokalizacji użytkownika,
jak również informowania o poziomie niebezpieczeństwa danej okolicy. 
Podstawowa wersja aplikacji umożliwia również dzielenie się planami z innymi użytkownikami oraz synchronizację kalendarza. 
W wersji płatnej użytkownik ma dostęp do szeregu funkcjonalności ułatwiających podróżowanie samolotem, np. informacja o dostępności
lepszych miejsc, przypomnienia o zarezerwowanych lotach, powiadomienia o statusie lotów w czasie rzeczywistym, mapy lotnisk wraz
ze szczegółowymi informacjami o położeniu obiektów, informacje o punkcie odbioru bagażu.

\paragraph{Harmony}

Umożliwia tworzenie planów podróży wraz ze znajomymi w czasie rzeczywistym oraz synchronizację z kalendarzem Google.
Ponadto aplikacja umożliwia śledzenie poniesionych wydatków i podziału kosztów na poszczególne osoby. 
Użytkownik ma możliwość otrzymywania sugestii generowanych przez AI dotyczących interesujących miejsc w danej lokalizacji 
jak również rezerwowania wycieczek w aplikacji. Aplikacja umożliwia również tworzenie list rzeczy do wykonania w
celu łatwego śledzenia postępów. Obiekty do zwiedzania są zwizualizowane na mapie Google w postaci pinezek.

\paragraph{Rove.me}

Jest to aplikacja, która sugeruje użytkownikowi najlepszy czas na odwiedzenie danego miejsca lub wydarzenia w ciągu roku. 
Aplikacja informuje również o typowej pogodzie występującej w interesującym użytkownika miejscu z uwzględnieniem pory roku lub daty.

\paragraph{Roadtrippers}

Umożliwia tworzenie planów podróży ze szczególnym uwzględnieniem podróży samochodem. 
Użytkownik oznacza na mapie interesujące go miejsca takie jak atrakcje, hotele, stacje paliw, sklepy itp. 
i udostępnia informacje o odległości od danej destynacji oraz szacowany czas dotarcia do niej. 
Aplikacja skierowana jest do użytku na terenie Stanów Zjednoczonych.

\paragraph{Podsumowanie}

Wymienione aplikacje różnią się poziomem rozbudowania i co za tym idzie oferowanymi funkcjonalnościami. 
Wspólnym mianownikiem jest możliwość tworzenia list z obiektami do odwiedzenia. Najbardziej rozbudowane aplikacje 
umożliwiają rezerwowanie usług oraz środków transportu, a także ustalanie optymalnych tras podróży. Każda z aplikacji 
oferuje graficzne oznaczenie obiektów na mapie. Jednak na ten moment wydaje się, że żadna nie dysponuje możliwością dynamicznego 
i zautomatyzowanego układania planu zwiedzania, który jest dostosowany do zainteresowań i możliwości czasowych użytkownika 
odwiedzającego dane miejsce.