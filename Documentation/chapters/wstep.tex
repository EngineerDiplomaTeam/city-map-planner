%! Author = Wiktor Rostkowski, Mateusz Budzisz
%! Date = 29.10.2023

\chapter{Wstęp}
\label{ch:wstep}

Celem projektu jest stworzenie interaktywnej mapy z punktami atrakcji, która umożliwi kompleksowe zaplanowanie optymalnej trasy zwiedzania z uwzględnieniem środków komunikacji miejskiej oraz godzin otwarcia atrakcji turystycznych. W dzisiejszym świecie, gdzie podróże turystyczne stają się coraz popularniejsze, narzędzie to będzie niezwykle przydatne dla turystów, zwłaszcza w dużych miastach, gdzie dostępność środków komunikacji miejskiej i różnorodność atrakcji turystycznych stanowią kluczowe wyzwania.

W miastach o dużej liczbie atrakcji turystycznych i rozbudowanych systemach transportu publicznego, planowanie efektywnej trasy zwiedzania może być skomplikowane. Warto więc opracować narzędzie, które ułatwi turystom eksplorację miasta i zapewni im możliwość zwiedzania miejscowych atrakcji w najbardziej optymalny sposób.

Projekt ten zakłada integrację różnych źródeł danych, takich jak informacje o trasach komunikacji miejskiej, godziny otwarcia atrakcji, informacje turystyczne i dane geograficzne, aby dostarczyć spersonalizowane trasy zwiedzania, uwzględniając preferencje użytkownika, takie jak zainteresowania i dostępność czasu.

W niniejszej pracy omówimy cele projektu, jego znaczenie w kontekście współczesnego ruchu turystycznego oraz zaprezentujemy plan działania i zakres projektu. Ponadto, przedstawimy również strukturę pracy oraz sposób, w jaki kolejne rozdziały będą się rozwijać, aby dokładnie opisać proces tworzenia interaktywnej Mapy z punktami atrakcji.
