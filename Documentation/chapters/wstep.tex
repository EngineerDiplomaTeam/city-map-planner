%! Author = Wiktor Rostkowski, Mateusz Budzisz
%! Date = 29.10.2023

\chapter{Wstęp}
\label{ch:wstep}

W miastach, które oferują wiele atrakcji turystycznych oraz posiadają rozbudowane~i~złożone systemy transportu publicznego, tworzenie optymalnego planu zwiedzania może~być~skomplikowane~i~czasochłonne.
Konieczność uwzględnienia godzin otwarcia poszczególnych miejsc, dostępności różnych środków transportu oraz efektywnego wykorzystania czasu sprawia,~że~planowanie wycieczki wymaga dużej precyzji~i~uwagi.
Co więcej, uwzględnienie indywidualnych preferencji turystów, takich~jak~zainteresowania, tempo zwiedzania~czy~potrzeba przerw~na~odpoczynek, jeszcze bardziej zwiększa stopień trudności tego zadania.

Zespół projektowy wcześniej napotkał dokładnie~ten~problem,~co~doprowadziło~do~powstania pomysłu~na~stworzenie nowatorskiej aplikacji.
Aplikacja~ta~składa~się~z interaktywnej mapy, która prezentuje różnorodne atrakcje turystyczne~w~danym mieście, oraz kalendarza~w~stylu \gls{drag-n-drop}, umożliwiającego łatwe planowanie dnia.
Aplikacja uwzględnia dane~w~czasie rzeczywistym, takie~jak~prognoza pogody, aktualne informacje~o~komunikacji miejskiej oraz godziny otwarcia poszczególnych atrakcji.
Ponadto aplikacja oferuje interaktywny widok, który automatycznie generuje optymalną trasę zwiedzania zgodnie~z~ustalonym planem, wspierając użytkowników~w~maksymalnym wykorzystaniu~ich~czasu~i~zasobów podczas wizyty~w~mieście.

Niniejsza praca przedstawia szczegółowy proces powstawania opisanego rozwiązania, rozpoczynając~od~przedstawienia~i~omówienia problemu, kontekstu oraz zakresu systemu,~a~także omówienia wymagań.
Następnie przedstawione~są~kluczowe decyzje projektowe, które kształtowały architekturę systemu, oraz szczegółowy opis procesu implementacji, obejmujący realizację poszczególnych modułów.
W kolejnej części pracy omawiany jest proces testowania,~w~tym metody~i~narzędzia używane~do~zapewnienia jakości~i~niezawodności systemu.
Zakończenie pracy obejmuje prezentację osiągniętych rezultatów oraz~ich~analizę,~a~także podsumowanie całości projektu.
Warto zaznaczyć,~że~praca początkowo była realizowana przez czteroosobowy zespół, jednak~pod~sam koniec projektu dwie osoby zdecydowały~się~zrezygnować~z~dalszego udziału,~co~wpłynęło~na~ostateczny przebieg~i~realizację projektu.
