%! Author = Mateusz Budzisz
%! Date = 04/11/2023

\chapter{Decyzje projektowe}
\label{ch:decyzje-projektowe}

\section{Narzędzia i technologie}
\label{sec:narzedzia-i-technologie}

\subsection{Języki programowania i biblioteki}
\label{subsec:jezyki-programowania-i-biblioteki}

\subsubsection{Oprogramowanie po stronie użytkownika}
Jednym z kluczowych celów projektu jest stworzenie progresywnej aplikacji internetowej (PWA) dostępnej na większości urządzeń codziennego użytku, takich jak komputery osobiste i smartfony.
Według statystyk prowadzonych przez StatCounter większość konsumenckiego ruchu internetowego jest generowana przez systemy operacyjne Android, Windows i iOS~\cite{statcounter-os-shares}\@.
To zróżnicowanie platform stawia przed zespołem projektowym wyzwanie polegające na opracowaniu aplikacji, która będzie działała efektywnie na każdym z tych systemów.
Nasz zespół składający się z czterech pracujących na etacie studentów nie byłby w stanie stworzyć a co dopiero utrzymywać rozwiązania na 3 różnych platformach na raz.
Istnieją rozwiązania niwelujące ten problem w znacznym stopniu, zespół projektowy przeanalizował część z nich pod kątem naszego doświadczenia.

\paragraph{Flutter}
Flutter to otwartoźródłowy zestaw narzędzi dla programistów stworzony przez firmę Google~\cite{flutter-quick-start}.
Pozwala na tworzenie natywnych, wieloplatformowych aplikacji mobilnych, komputerowych oraz internetowych.
Aplikacja stworzona przy użyciu tej technologii umożliwia napisanie kodu źródłowego w jednej wersji i uruchomienie go na szerokiej gamie urządzeń docelowych.

Zalety:
\begin{itemize}
    \item Jednolity kod źródłowy dla wielu platform
    \item Obsługa natywnych funkcji systemów mobilnych i desktopowych
\end{itemize}

Wady:
\begin{itemize}
    \item Język programowania Dart, który jest stosunkowo niszowy i nieznany zespołowi
    \item Relatywnie niska wydajność dostarczonego oprogramowania\cite{flutter-perf}
\end{itemize}

\paragraph{Electron}
Electron to otwartoźródłowa platforma umożliwiająca tworzenie aplikacji w formie stron internetowych oraz osadzanie ich w minimalistycznej wersji przeglądarki, która jest dostarczana razem z aplikacją~\cite{electron-in-action}.

Zalety:
\begin{itemize}
    \item Możliwość wykorzystania React-a, poznanego przez zespół podczas zajęć akademickich
    \item Wsparcie dla szerokiej gamy platform docelowych
\end{itemize}

Wady:
\begin{itemize}
    \item Duży rozmiar końcowej aplikacji z powodu konieczności dołączenia przeglądarki internetowej
    \item Potencjalne problemy z wydajnością
\end{itemize}

\paragraph{Progresywna Aplikacja Internetowa}
Progresywne aplikacje internetowe (Progressive Web Apps) to aplikacje uruchamiane jak zwykłe strony internetowe, ale zapewniające doświadczenie podobne do natywnych aplikacji mobilnych lub desktopowych.\cite{pwa-book}

Zalety:
\begin{itemize}
    \item Bardzo mały rozmiar końcowej aplikacji
    \item Relatywnie dobra wydajność dzięki optymalizacji przeglądarek pod kątem wielu platform docelowych
\end{itemize}

Wady:
\begin{itemize}
    \item Zależność od przeglądarki użytkownika, która może nie obsługiwać wszystkich funkcjonalności
\end{itemize}

\paragraph{Wybór technologii}
Każde z analizowanych rozwiązań ma swoje unikalne zalety i wady.
Zespół projektowy kierował się następującymi priorytetami podczas wyboru technologii:

\begin{enumerate}
    \item Znajomość technologii w zespole
    \item Wydajność rozwiązania
\end{enumerate}

Na podstawie powyższych czynników zdecydowaliśmy się na Progressive Web Apps.
Dzięki temu rozwiązaniu możliwe jest stworzenie aplikacji internetowej, która działa jak natywna aplikacja mobilna czy desktopowa, a jednocześnie nie wymaga dołączenia przeglądarki do paczki instalacyjnej, jak w przypadku Electron.

Aby wykorzystać komercyjne doświadczenie zespołu, zdecydowaliśmy się zaimplementować aplikację w frameworku Angular, zamiast korzystać z React-a poznanego na uczelni.
Angular oferuje solidne narzędzia do tworzenia wydajnych aplikacji internetowych oraz doskonałą integrację z \acrshort{pwa}.

Wybór technologii Progressive Web Apps umożliwił naszemu zespołowi stworzenie wydajnej, wieloplatformowej aplikacji dostępnej na większości urządzeń codziennego użytku.
Decyzja ta pozwoliła na efektywne wykorzystanie umiejętności zespołu, a także na dostarczenie rozwiązania zgodnego z priorytetami projektu, jakimi są wydajność i znajomość technologii.

\subsubsection{Oprogramowanie niewidoczne dla użytkownika}
Nasz projekt zakłada stworzenie zaawansowanego algorytmu ustalania optymalnej trasy zwiedzania na podstawie danych czasu rzeczywistego.
Ciągłe aktualizacje danych po stronie użytkowników spowodowałoby wysokie zużycie internetu oraz ogromne obciążenie po stronie dostawców danych.
Dlatego postanowiliśmy stworzyć serwis, który będzie odpowiadać za aktualność danych oraz wyznaczanie trasy.
Z uwagi na \acrshort{pwa}, język, w którym stworzymy ten serwis, musi mieć bardzo dobre wsparcie dla protokołu HTTP\@.
Na uczelni poznaliśmy 2 technologie tworzenia takich serwisów.

\paragraph{Java}
Java to język paradygmatu obiektowego, bardzo popularny w systemach bankowych.
W trakcie zajęć na uczelni poznaliśmy sposób tworzenia serwisów \acrshort{http} w technologii Spring.
Części naszego zespołu nie podobają się konwencje narzucone przez tę technologię, jak i sposób komunikacji z bazą danych.
Należy zaznaczyć, że nikt z naszego zespołu nie posada komercyjnego doświadczenia z tą technologią.

\paragraph{.NET}
C\# to jeden z języków z rodziny .NET, który mielimy szanse poznać na uczelni.
Łączy on w sobie paradygmaty obiektowe i funkcyjne, dzięki czemu jest bardziej elastyczny, jeśli chodzi o styl programowania.
Trzy z czterech osób w naszym zespole ma doświadczenie komercyjne w tej technologi.

\paragraph{Wybór}
Z uwagi na doświadczenie zespołu postawiliśmy na C\#.

\subsubsection{Baza danych}
Na uczelni dogłębnie poznaliśmy bazę danych PostgreSql, zespół ma komercyjne doświadczenie z tą bazą, więc nie braliśmy innych rozwiązań pod uwagę.

\subsection{Narzędzia programistyczne}
\label{subsec:narzedzia-programistyczne}
Nasz zespół składa się z osób, które mają znaczne doświadczenie w oprogramowaniu, którego używa na co dzień w pracy.
Wybrany przez nas stos technologiczny składa się z .NET C\#, Angular oraz PostgreSql.
Wszystkie te technologie, pomimo iż są utrzymywane przez ogromne korporacje, są otwarto-źródłowe, co w praktyce pozwala na tworzenie narzędzi deweloperskich przez niezależnych twórców, dzięki czemu nie bylibyśmy uwiązani do konkretnego rozwiązania.
Wybór oprogramowania do tworzenia kodu został podyktowany wybraną technologią a w drugiej kolejności z uwagi na zróżnicowany sprzęt komputerowy, którym dysponujemy  dostępnością oprogramowania na wielu platformach (tj.: Windows, MacOS, Linux).

\subsubsection{.NET/C\#}
.NET to framework stworzony i rozwijany przez Microsoft wspólnie ze społecznością otwartego oprogramowania.
Pierwszym oprogramowaniem, które przychodzi nam do głowy, gdy słyszymy .NET, jest Microsoft Visual Studio.
Oprogramowanie to powstało w 1997 roku i przez wiele lat było konsekwentnie ulepszane.
W 2016 roku VS zostało wydane na komputery Mac, jednakże, po 7 latach, w 2023 roku Microsoft ogłosił zakończenie wsparcia dla wersji MacOS.
Jeśli chodzi o systemy Linux, to VS nigdy nie zostało wydane na ten system.

Zespół projektowy bardzo nie chciał uzależnić pracy nad projektem od systemu Windows, Microsoft poleca użycie programu Visual Studio Code, który jest wspierany na wszystkich systemach operacyjnych, których używamy.
Niestety VSC jest edytorem przeznaczenia ogólnego, co za tym idzie, znaczna większość specjalistycznych dla danej technologii narzędzi jest obsługiwana przy pomocy rozszerzeń.
Rozszerzenia mogą tworzyć wszyscy, jest ich całkiem sporo niestety wszystkie rozszerzenia skupiające się na technologii .NET/C\# są we wczesnej fazie rozwoju i nie wspierają tak podstawowych rzeczy, jak debugowanie kodu.
Nawet gdy dane rozszerzenie wspiera jakąś zaawansowaną funkcjonalność, to przeważnie konfiguracja takiego rozszerzenia bardzo różni się pomiędzy systemami operacyjnymi co, de facto niweczy sens wieloplatformowości.

W 2016 roku został ogłoszony JetBrains Rider niejako odpowiedź na VS na MacOS od Microsoftu.
Rider bardzo dynamicznie się rozwijał, nadganiając, a nawet prześcigając VS pod względem funkcjonalności, do momentu, w wyniku czego duża część środowiska programistycznego zaczęła go używać jako głównego narzędzia do pracy.
Programy firmy JetBrains są znane ze świetnej integracji wieloplatformowej oraz dbania o szczegóły, co zapewnia dobre doświadczenie deweloperskie.

Dzięki PJATK nasz zespół mógł przetestować, każdy z tych programów w ramach licencji edukacyjnej.
Nasze testy doprowadziły nas do decyzji, aby postawić na Ridera od JetBrains-ów, z zachowaniem możliwości uruchomienia projektu w Visual Studio, aby koledzy z dużym doświadczeniem komercyjnym opartym o VS nie musieli zmieniać swoich przyzwyczajeń.

\subsubsection{Angular}
Angular z natury bycia technologią tworzenia stron internetowych nie posiada dedykowanego środowiska programistycznego stworzonego przez jedną firmę.
Zespół tworzący ten framework stworzył własny program wspomagania deweloperów na podstawie protokółu \acrshort{lsp}, co sprowadza się do tego, że każde środowisko programistyczne wspierające ten protokół zapewni ten sam poziom wsparcia dla Angular'a.

Warto tu wyróżnić Visual Studio Code, ponieważ jest to środowisko zalecane przez twórców Angular-a, a nowe projekty tworzone w tej technologii zawierają niezbędną konfigurację potrzebną do pełnego wsparcia dla Angulara.

JetBrains też ma w swojej ofercie program do tworzenia w technologiach internetowych o nazwie WebStorm, który posiada automatyczną konfigurację dla Angular-a, jak i wielu innych technologii, więc nie potrzebuje bezpośredniego wsparcia od autorów technologi.

Mając na uwadzę fakt, że Visual Studio Code domyślnie ma skróty klawiszowe nieprzypominające innych popularnych środowisk programistycznych, których trzeba by było się nauczyć, oraz fakt, że już wybrany Rider jest niemalże identyczny w obsłudze do WebStorm'a, postawiliśmy na WebStorm'a, znowu z możliwością uruchomienia projektu w dowolnym innym środowisku, gdy ktoś będzie tego z jakiegoś powodu potrzebował.

\subsubsection{PostgreSql}
W przypadku baz danych przeważnie jest tak, że twórcy bazy danych, równocześnie do samej bazy danych rozwijają narzędzie pozwalające na administrację tą bazą.
W przypadku PostgreSql narzędzie to nazywa się pgAdmin, umożliwia kompleksową konfigurację bazy danych i administrowanie nią.

Należy jednak wspomnieć tu o integracji Ridera z bazą danych we współpracy z programem DataGrip.
Gdy mamy licencję na DataGrip-a, to wewnątrz Rider-a możemy podłączyć się do bazy danych i otrzymywać dodatkowe wsparcie w kontekście bazy danych.
Wsparcie nie ogranicza się tylko do dodatkowego kolorowania składni w zależności od dialektu wybranej bazy danych, a deweloper jest wspierany przez podpowiadanie nazw kolumn, tabeli oraz poprawność zapytań do bazy jest  weryfikowana w czasie rzeczywistym, a w przypadku wątpliwości Rider pozwala na szczegółową analizę kwerendy w DataGrip'e.

Mając na uwadze powyższe, zespół projektowy zaleca użycie DataGrip-a jako narzędzia do projektowania rozwiązań dookoła bazy danych z uwagi na kompleksową integrację z Riderem.

\subsubsection{Dalekosiężność wyborów}
W przypadku, w którym zespół zdecydowałby się na dalsze utrzymywanie bądź rozwój projektu, należy brać pod uwagę koszta oprogramowania.

Visual Studio jest najdroższym z wymienionych rozwiązań, do tego wspiera tylko wąski fragment stosu technologicznego.

Oprogramowanie JetBrains, na które postawiliśmy, jest dostępne jako poszczególne programy w formie abonamentu, jednakże najkorzystniej jest je kupować w pakietach łączonych.

Warto też wspomnieć o programie wsparcia open source firmy JetBrains, która to oferuje darmowe licencje do wszystkich swoich komercyjnych rozwiązań dla projektów o otwartym kodzie źródłowym, przy czym nasz projekt spełnia tę definicję.

Ponadto dzięki zachowaniu kompatybilności z alternatywnymi środowiskami deweloperskimi, jesteśmy w stanie małym nakładem pracy zmienić oprogramowanie nawet na darmowe.

\subsection{Dostawcy mocy obliczeniowej}
\label{subsec:dostawcy-mocy-obliczeniowej}
Wytworzone oprogramowanie potrzebuje serwerów, na których będzie wdrożone.
Zespół projektowy zna 2 podejścia do tego problemu.

\subsubsection{Oprogramowanie w chmurze}
Dostawcy oprogramowania w chmurze umożliwiają bardzo elastyczny system rozliczania się za moc obliczeniową.

Gdy aplikacja nie wymaga ciągłej pracy oraz w sposób deterministyczny można ustalić warunki czasowe dla aplikacji, możemy jej użyć w stylu \gls{on-demand}.
Sposób ten charakteryzuje bardzo korzystny sposób rozliczania, gdzie płacimy za faktycznie wykorzystany czas pracy maszyn oraz potencjalnie nieskończone możliwości horizontal-scallingu.

Wybór takiego rozwiązania mocno wpłynąłby na architekturę naszego rozwiązania.
Niejako wymusiłby na nas, podział projektu na niezależne od siebie moduły i z uwagi na bardzo skomplikowany \gls{refactoring} przy 4-osobowym zespole de facto scementowałby strukturę projektu.

Nasz projekt teoretycznie można podzielić na niezależne od siebie części o różnym zapotrzebowaniu na moc obliczeniową.
Pierwszym modułem musi być baza danych, w której będziemy trzymać graf możliwych połączeń między atrakcjami oraz informacje o samych atrakcjach, ten moduł musiałby być cały czas dostępny z uwagi na duży koszt wyprodukowania danych, przeważający koszt utrzymania bazy w sposób ciągły.
Drugim modułem mógłby być \gls{job} aktualizujący dane z OSM, urzędu miasta oraz atrakcji turystycznych, taki moduł mógłby uruchamiać się na przykład raz dziennie.
Trzecim modułem byłby serwis implementujący algorytm, dobierania trasy pomiędzy podanymi POI, taki serwis uruchamiałby się na zapytanie użytkownika i kończył swoją pracę wraz z podaniem optymalnie dobranej trasy.
Ostatnim modułem byłaby aplikacja \gls{frontend}-owa implementująca UI, włączana z osobna dla każdego użytkownika, gdy ten podejmie próbę wejścia na adres naszego rozwiązania.

Takie podejście w praktyce zmniejsza koszt utrzymania do kosztu utrzymywania bazy danych, ponieważ gdy zapotrzebowanie na pozostałe moduły drastycznie by zmalało, to nie będą one chodzić i czekać na ruch użytkowników.
Niestety uruchamianie programów na żądanie wiąże się ze znaczącym opóźnieniem wobec żądań użytkownika docelowego.
Gdy użytkownik postanowi skorzystać z naszej aplikacji, dostawca chmury obliczeniowej musi uruchomić co najmniej dwie maszyny, dla modułu 3 i 4, co czasami potrafi zająć kilka sekund.
Gdy dostawca chmury uruchomi odpowiednie maszyny, jeszcze nasze oprogramowanie wymaga czasu na obsłużenie żądania.
W przypadku modułu 3 musimy nawiązać połączenie z bazą danych, odczytać dane, zastosować na nich skomplikowany obliczeniowo algorytm i wysłać odpowiedź, obawiamy się, że w skrajnych przypadkach może to zająć za dużo czasu aby odpowiedź trafiła do użytkownika przed zamknięciem połączenia.
W przypadku modułu 4 możemy odroczyć oczekiwanie na odpowiedź modułu 3 i poinformować użytkownika o wydłużonym czasie oczekiwania, niemniej jednak, aby cokolwiek użytkownikowi pokazać i tak jesteśmy zmuszeni do wykonania kodu odpowiedzialnego za stworzenie \gls{ui}, według doświadczeń zespołu czas potrzebny na \gls{rendering} \gls{hello-world} pozostawia niewiele czasu do przekroczenia średniego time span nowego użytkownika (TODO: source).

Istnieją techniki pozwalające na optymalizację tych rozwiązań, aby wyeliminować przytoczone problemy.
Jednak opierają się one o jeszcze większą fragmentację aplikacji, na którą nie możemy sobie pozwolić z uwagi na liniowo rosnący czas potrzebny na konfigurację infrastruktury potrzebnej do skomunikowana wszystkich modułów w \glslink{hermetyzacja}{hermetyczny} sposób.
Infrastrukturę chmury internetowej w każdym ze znanych nam dostawców to jest: \acrshort{aws}, \acrshort{gcp}, \acrshort{azure} możemy wyklikać w przeglądarce internetowej.
Jednak usługi te znacząco się od siebie różnią.
Możemy tę manualną pracę zautomatyzować przy pomocy podejścia \gls{infra-as-code}, na przykład przy użyciu terraform.
Pomimo iż dostawcy oferują podobne rozwiązania, to różnią się one w sposobie konfiguracji, na przykład 1 usługa w \acrshort{gcp} to 3 usługi w \acrshort{aws}\@.
Terraform daje możliwość skorzystania z dodatkowej warstwy abstrakcji pozwalającej na opisanie architektury dla \acrshort{aws}, jak i zarówno \acrshort{gcp}, jednakże nie jest to pełne wsparcie i niektóre usługi po prostu nie pokrywają się pomiędzy dostawcami.
Ponadto Terraform nie wspiera takiej funkcjonalności dla Azure.

\subsubsection{Oprogramowanie on premise}
Oprogramowanie on premise możemy podzielić na bare metal i virtual.

Bare metal to takie, gdzie mamy fizyczny dostęp do maszyny i płacimy za konkretne podzespoły, mamy pełną kontrolę nad maszyną.
Virtual to oprogramowanie, w którym tak jak w przypadku bare-metal musimy sami zarządzać maszyną.
Różni się jednak tym, że płacimy za parametry podzespołów, nie za konkretne podzespoły.
Nie mamy też fizycznego dostępu do maszyny, ponieważ parametry, jakich potrzebujemy, mogą być mniej wymagające, niż fizyczna maszyna jest w stanie zapewnić, przez to właściciele takich maszyn stosują \glslink{virt}{wirtualizację}.

Bare metal ma taką zaletę, że jesteśmy pewni, że nikt poza nami na takiej maszynie nie będzie pracował.
Minusem jest mała elastyczność podzespołów (musimy wiedzieć, co będzie działać, z czym) oraz wysoki koszt początkowy.
W przypadku wirtualizacji na początku rozwoju aplikacji możemy poprosić o niewielkie parametry a wraz z rozwojem przeskalować parametry do naszych potrzeb.

Pomimo faktu, iż w przypadku \gls{on-prem} musimy sami zarządzać infrastrukturą, to jest to niezależne od dostawcy sprzętu, dzięki czemu można łatwo przenieść całą aplikację do innego dostawcy.
Biorąć pod uwagę ten fakt oraz brak możliwości napisania oprogramowania w sposób pełni agnostyczny względem konkretnych usług chmury obliczeniowej, zespół projektowy zdecydował się na on premise w wariancie virtual.

\subsubsection{Przegląd rynku}
\paragraph{Microsoft Azure Cloud}
Chmura obliczeniowa Microsoftu została nam udostępniona w ramach zajęć z konteneryzacji.
Microsoft udostępnia studentom jednorazowo 200\$ do wykorzystania na okres roku.
W ramach okresu edukacyjnego mamy do dyspozycji najsłabsze i najmniej stabilne maszyny dostępne w ofercie.
W trakcie zajęć natknęliśmy się na wiele problemów ze stabilnością i wydajnością tej chmury.
Jednorazowy roczny okres próbny w połączeniu z ograniczeniem budżetowym całkowicie deklasuje tą chmurę z powodu braku możliwości utrzymania rozwiązania w rozumieniu długoterminowym

\paragraph{Amazon Web Services}
Chmurę obliczeniową Amazon mieliśmy okazję zobaczyć na szkoleniu organizowanym przez Merapar.
W trakcie zajęć chmura nie sprawiła najmniejszych problemów, co pokrywa się z licznymi opiniami rynkowymi.
Niestety pomimo wielu darmowych progów zużycia, Amazon nie oferuje darmowego progu w przypadku usługi EC2 (\gls{on-prem} virtual).

\paragraph{Google Cloud Platform}
GCP zostało nam zaprezentowane w ramach projektu PJATK cloudelina.
Usługi w ramach GCP wydają się znacznie prostsze w konfiguracji oraz bardzo dobrze integrują się z Angular-em.
Niestety podobnie jak w przypadku Amazon-u, GCP nie oferuje darmowego progu dla maszyn wirtualnych.

\paragraph{Oracle Cloud Platform}
Oracle cloud jako jedyne nie posiada usług typowych dla architektury chmurowych.
Oracle daje bardzo hojny darmowy limit dla maszyn wirtualnych opartych o \glslink{arm}{architekturę arm} wynoszący 6 CPU, 24 GB RAM, 200 HDD i 100 Mbs ethernet.
Taki próg pozwoli nie tylko na wytworzenie oprogramowania, przeprowadzenie testów, ale nawet i obsługę średniej wielkości ruchu użytkowników.

\subsection{Przechowywanie kodu oraz potok ciągłego wdrożenia}
\label{subsec:przechowywanie-kodu-oraz-potok-ciagego-wdrozenia}

W trakcie zajęć na uczelni, jak i pracy indywidualnej członków zespołu projektowego poznaliśmy 2 systemy kontroli wersji.
Są to GIT oraz Mercurial.

GIT pod każdym względem jest lepszy od Mercurial, sam Mercurial jest uznany za projekt bez rozwoju.
Zespół projektowy postanowił oprzeć swoją pracę na podstawie GIT, natomiast nie chcemy samodzielnie zarządzać serwerem repozytorium.

\subsubsection{Usługi internetowe ułatwiające pracę wielu współbieżną w repozytorium GIT}
\paragraph{GitLab}
\paragraph{BitBucket}
\paragraph{Microsoft Azure Repositories}
\paragraph{Github}
