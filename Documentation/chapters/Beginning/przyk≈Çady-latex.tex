%! Author = mate
%! Date = 11.11.2023

\chapter{Przykłady LaTeX}
\label{ch:przykady-latex}

\section{Cytaty, źródła}
\label{sec:cytaty-zroda}
Cytat: \cite{bla}

\section{Listingi}
\label{sec:listingi}
Jak załączać kod źródłowy jest pokazane na listingu~\ref{lst:helloworld}

\begin{lstlisting}[language=c,caption={Przykładowy witaj w świecie}, label={lst:helloworld}]
printf("hello");
\end{lstlisting}

\section{Obrazki}
\label{sec:obrazki}

Na ilustracji~\ref{fig:pjatklogo} widzimy oficjalne logo PJATK\@.

\begin{figure}[h]
    \centering
    \inputgraphics[width=0.5\textwidth]{../attachments/pjatk}
    \caption{Logo PJATK załączone jako obrazek}
    \label{fig:pjatklogo}
\end{figure}

albo dla wygody jako makro tak jak na obrazku~\ref{img:pjatklogo2}

\putimage{Obrazek załączony za pomocą makra}{attachments/pjatk}{img:pjatklogo2}{0.5\textwidth}

\section{Karty udziałowca}
\label{sec:karty-udziaowca}

\begin{stakeholder}[label={tab:stakeholder:someholder},caption={Przykładowy opis udzialowca}]
    \id{UOB 01}
    \name{nazwa udziałowca}
    \descr{opis udziałowca}
    \type{ożywiony/nieożywiony, bezpośredni/pośredni}
    \viewpoint{z jakiej perspektywy patrzy udziałowiec np. technicznej, ekonomicznej, operatora systemu itp.}
    \limitations{ograniczenia udziałowca np. administrator nie powinien specyfikować wymagań finansowych}
    \requ{tu tylko symbole wymagań wyspecyfikowanych w rozdziale 3}
\end{stakeholder}

\section{Wymagania wszelakie}
\label{sec:wymagania-wszelakie}

Na tabeli~\ref{tab:requirements:general} pokazano jak można definiować wymagania ogólne lub dziedzinowe.

\begin{requirementstab}[label={tab:requirements:general},caption={Przykładowe wymaganie ogólne lub dziedzinowe}]
    \id{WO1}
    \priority{M}
    \name{krótki opis}
    \descr{opis szczegółowy, należy dążyć do tego, żeby wszystkie znane na ten moment szczegóły wymagania zostały wydobyte i wyspecyfikowane}
    \sholder{nazwa udziałowca, który podał wymaganie}
    \reqrelated{wymagania zależne i uszczegóławiające – odesłanie poprzez identyfikator}
\end{requirementstab}

Teraz czas na wymagania funkcjonalne, na przykład~\ref{tab:requirements:func1}

\begin{requirementstab}[label={tab:requirements:func1},caption={Pryzkładowa tabela z wymaganiami na interfejs z otoczeniem}]
    \id{FO1}
    \priority{S}
    \name{krótki opis}
    \descr{opis szczegółowy, należy dążyć do tego, żeby wszystkie znane na ten moment szczegóły wymagania zostały wydobyte i wyspecyfikowane

    Można zastosować opis jak w User Story
        \begin{itemize}
            \item Jako.. (konkretny użytkownik systemu)
            \item chcę... (pożądana cecha lub problem, który trzeba rozwiązać)
            \item bo wtedy/ponieważ… (korzyść płynąca z ukończenia story)
        \end{itemize}
    }
    \acceptcrit{Warunki Satysfakcji (Szczegóły dodane na potrzeby  testów akceptacyjnych)}
    \inputdata{uzupełniane w trakcie sprintu – dane wejściowe, związane z wymaganiem}
    \preconditions{ uzupełniane w trakcie sprintu – warunki, które muszą być prawdziwe przed wywołaniem operacji}
    \postconditions{ uzupełniane w trakcie sprintu – warunki, które muszą być prawdziwe po wywołaniu operacji}
    \exceptions{ uzupełniane w trakcie sprintu – niepożądane sytuacje i sposoby ich obsługi}
    \implementation{ uzupełniane w trakcie sprintu – opis sposobu realizacji}
    \sholder{nazwa udziałowca, który podał wymaganie}
    \reqrelated{wymagania zależne i uszczegóławiające – odesłanie poprzez identyfikator}
\end{requirementstab}


Natomiast tabela~\ref{tab:requirements:env1} pokazuje wymagania na interfejs z otoczeniem.

\begin{requirementstab}[label={tab:requirements:env1},caption={Pryzkładowa tabela z wymaganiami na interfejs z otoczeniem}]
    \id{IO1}
    \priority{C}
    \name{krótki opis}
    \descr{opis szczegółowy, należy dążyć do tego, żeby wszystkie znane na ten moment szczegóły wymagania zostały wydobyte i wyspecyfikowane}
    \acceptcrit{Warunki Satysfakcji (Szczegóły dodane na potrzeby  testów akceptacyjnych)}
    \inputdata{uzupełniane w trakcie sprintu – dane wejściowe, związane z wymaganiem}
    \preconditions{ uzupełniane w trakcie sprintu – warunki, które muszą być prawdziwe przed wywołaniem operacji}
    \postconditions{ uzupełniane w trakcie sprintu – warunki, które muszą być prawdziwe po wywołaniu operacji}
    \exceptions{ uzupełniane w trakcie sprintu – niepożądane sytuacje i sposoby ich obsługi}
    \implementation{ uzupełniane w trakcie sprintu – opis sposobu realizacji}
    \sholder{nazwa udziałowca, który podał wymaganie}
    \reqrelated{wymagania zależne i uszczegóławiające – odesłanie poprzez identyfikator}
\end{requirementstab}

Tabela dotycząca wymagań pozafunkcjonalnych~\ref{tab:requirements:nonfunc1} jest także widoczne.

\begin{requirementstab}[label={tab:requirements:nonfunc1},caption={Pryzkładowa tabela z wymaganiami pozafunkcjonalnymi}]
    \id{NFO1}
    \priority{W}
    \name{krótki opis}
    \descr{opis szczegółowy, należy dążyć do tego, żeby wszystkie znane na ten moment szczegóły wymagania zostały wydobyte i wyspecyfikowane}
    \acceptcrit{Warunki Satysfakcji (Szczegóły dodane na potrzeby  testów akceptacyjnych)}
    \sholder{nazwa udziałowca, który podał wymaganie}
    \reqrelated{wymagania zależne i uszczegóławiające – odesłanie poprzez identyfikator}
\end{requirementstab}

\section{Przykładowy UML}
\label {sec:przykadowy-uml}

\begin{center}
    \begin{tikzpicture}
        \begin{umlsystem}[x=4, fill=red!10]{The system}
            \umlusecase{use case1}
            \umlusecase[y=-2]{use case2}
            \umlusecase[y=-4]{use case3}
            \umlusecase[x=4, y=-2, width=1.5cm]{use case4 on 2 lines}
            \umlusecase[x=6, fill=green!20]{use case5}
            \umlusecase[x=6, y=-4]{use case6}
        \end{umlsystem}

        \umlactor{user}
        \umlactor[y=-3]{subuser}
        \umlactor[x=14, y=-1.5]{admin}

        \umlinherit{subuser}{user}
        \umlassoc{user}{usecase-1}
        \umlassoc{subuser}{usecase-2}
        \umlassoc{subuser}{usecase-3}
        \umlassoc{admin}{usecase-5}
        \umlassoc{admin}{usecase-6}
        \umlinherit{usecase-2}{usecase-1}
        \umlVHextend{usecase-5}{usecase-4}
        \umlinput[name=incl]{usecase-3}{usecase-4}

        \umlnote[x=7, y=-7]{incl-1}{note on input dependency}
    \end{tikzpicture}
\end{center}
