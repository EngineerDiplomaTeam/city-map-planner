%! Author = mateuszbudzisz
%! Date = 08/11/2023
\usepackage{glossaries}

\section{Słownik pojęć}
\label{sec:slownik-pojec}

\newglossaryentry{PWA}name={pwa}, description={Progressive Web App}
\newglossaryentry{AWS}name={aws}, description={Amazon Web Services}
\newglossaryentry{GCP}name={gcp}, description={Google Cloud Platform}
\newglossaryentry{Azure}name={azure}, description={Microsoft Azure Cloud}
\newglossaryentry{On-demand}name={on-demand}, description={Rodzaj opgrogramowania charakteryzojącego się dynamicznym czasem pracy, uruchamiane na rządanie, gdy poda wynik program kończy pracę zamiast oczekiwać następnego zapytania}
\newglossaryentry{Refactoring}name={refactoring}, description={Znaczna zmiana konstrukcji programu mająca na celu usprawnienie oprogramowania bądź dostosowanie go do nowych wymogów}
\newglossaryentry{UI}name={ui}, description={Interfejs użytkownika}
\newglossaryentry{Frontend}name={frontend}, description={Oprogramowanie składające się z UI z którym docelowy użytkownik będzę wchodził w interakcję}
\newglossaryentry{Backend}name={backend}, description={Oprogramowanie pozbawione UI z którym docelowy użytkownik będzę wchodził w interakcję, potrzebne do prawidłowej pracy Frontendu}
\newglossaryentry{Job}name={job}, description={Oprogramowanie, które ma z góry określony cel, po jego uruchomieniu natychmiast zaczyna je wykonywać, nie wchodzi w interakcje z użytkownikiem docelowym, po zakończeniu kończy swoje życie}
\newglossaryentry{OSM}name={osm}, description={Open Street Map}
\newglossaryentry{POI}name={poi}, description={Point of interest}
\newglossaryentry{Renderowanie}name={rendering}, description={Stworzenie UI z postaci kodu do postaci konsumowalnej przez użytkownika docelowego}
\newglossaryentry{Hello world}name={hello-world}, description={Minimalny reprezentatywny program w danej technologii}
\newglossaryentry{HHermetyzacja}name={hermetyzacja}, description={Hermetyzacja opgorgramowania określa dobrą praktykę programistyczną polegającą na izolacji komponentów w aplikacji tak aby o sobie nie wiedziały gdy nie muszą o sobie wiedzieć}
\newglossaryentry{Infrastructure as a code}name={infra-as-code}, description={Sposób opisania architektury systemu po przez napisanie programu tworzącego docelową architekturę przy pomocy abstrakcji dostarczonych przez dostawcę mocy obliczeniowej}
\newglossaryentry{On-premise}name={on-prem}, description={Oprogramowanie hostowane na samodzielnie zarządzanej infrastrukturze}
\newglossaryentry{Wirtualizacja}name={virt}, description={Podział serwera na maszyny o mniejszej mocy obliczeniowej aby umożliwić podział podsespołów pomiedzy klientów tak aby nic o sobie na wzajem nie wiedzieli}
\newglossaryentry{Architektura ARM}name={ARM}, description={Architektura silnej ręki lol}

\makeglossaries


