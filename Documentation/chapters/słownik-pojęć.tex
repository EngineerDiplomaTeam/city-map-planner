%! Author = Mateusz Budzisz
%! Date = 08/11/2023

\makeglossaries

\newacronym{pwa}{PWA}{Progressive Web App}
\newacronym{aws}{AWS}{Amazon Web Services}
\newacronym{gcp}{GCP}{Google Cloud Platform}
\newacronym{azure}{Azure}{Google Cloud Platform}
\newacronym{osm}{OSM}{Open Street Map}
\newacronym{lsp}{LSP}{Language Service Protocol}
\newacronym{osw}{OSW}{Open Weather Map}
\newacronym{poi}{POI}{Point of interest}
\newacronym{http}{HTTP}{Hyper Text Transfer Protocol}

\newglossaryentry{On-demand}
{
    name={on-demand},
    description={Rodzaj opgrogramowania charakteryzojącego się dynamicznym czasem pracy, uruchamiane na rządanie, gdy poda wynik program kończy pracę zamiast oczekiwać następnego zapytania}
}

\newglossaryentry{Refactoring}
{
    name={refactoring},
    description={Znaczna zmiana konstrukcji programu mająca na celu usprawnienie oprogramowania bądź dostosowanie go do nowych wymogów}
}

\newglossaryentry{UI}
{
    name={ui},
    description={Interfejs użytkownika}
}

\newglossaryentry{Frontend}
{
    name={frontend},
    description={Oprogramowanie składające się z UI z którym docelowy użytkownik będzę wchodził w interakcję}
}

\newglossaryentry{Backend}
{
    name={backend},
    description={Oprogramowanie pozbawione UI z którym docelowy użytkownik będzę wchodził w interakcję, potrzebne do prawidłowej pracy Frontendu}
}

\newglossaryentry{Job}
{
    name={job},
    description={Oprogramowanie, które ma z góry określony cel, po jego uruchomieniu natychmiast zaczyna je wykonywać, nie wchodzi w interakcje z użytkownikiem docelowym, po zakończeniu kończy swoje życie}
}


\newglossaryentry{Wyrenderowanie}
{
    name={rendering},
    description={Stworzenie UI z postaci kodu do postaci konsumowalnej przez użytkownika docelowego}
}

\newglossaryentry{Hello world}
{
    name={hello-world},
    description={Minimalny reprezentatywny program w danej technologii}
}

\newglossaryentry{Hermertyczny}
{
    name={hermetyczny},
    description={Hermetyzacja opgorgramowania określa dobrą praktykę programistyczną polegającą na izolacji komponentów w aplikacji tak aby o sobie nie wiedziały gdy nie muszą o sobie wiedzieć}
}

\newglossaryentry{Infrastructure as a code}
{
    name={infra-as-code},
    description={Sposób opisania architektury systemu po przez napisanie programu tworzącego docelową architekturę przy pomocy abstrakcji dostarczonych przez dostawcę mocy obliczeniowej}
}

\newglossaryentry{On-premise}
{
    name={on-prem},
    description={Oprogramowanie hostowane na samodzielnie zarządzanej infrastrukturze}
}

\newglossaryentry{Wirtualizacja}
{
    name={virt},
    description={Podział serwera na maszyny o mniejszej mocy obliczeniowej aby umożliwić podział podsespołów pomiedzy klientów tak aby nic o sobie na wzajem nie wiedzieli}
}

\newglossaryentry{Architektura ARM}
{
    name={arm},
    description={Architektura silnej ręki lol}
}
