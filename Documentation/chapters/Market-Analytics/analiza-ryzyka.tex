%! Author = Wiktor Rostkowski
%! Date = 05/1/2024


\section{Analiza Ryzyka}
\label{subsec:analiza-ryzyka}

% \begin{table}[ht]
\begin{longtable}{|p{.15\textwidth}|p{.15\textwidth}|p{.2\textwidth}|p{.25\textwidth}|p{.2\textwidth}|}
% \begin {adjustbox}{width=\textwidth}

% \begin{tabular}{| p{2cm} | p{2cm} | p{3.5cm} | p{3cm} | p{3cm} |}
% \begin{tabular}{|c|c|c|c|c}
    \hline
    Ryzyko & Czynniki ryzyka & Charakterystyka ryzyka & Prawdopodobieństwo wystąpienia ryzyka & Planowane działania \\
    \hline
    \multirow{7}{*}{\parbox[t]{2mm}{\rotatebox[origin=c]{90}{Nieukończenie projektu w terminie}}} & & & & \\
    \cline{2-5}
     & Rozpad grupy & Odejście jednego z członków zespołu & niskie 
    & Przejrzysta komunikacja w zespole i ustalenie wersji MVP produktu, która może zostać ukończona w uszczuplonym składzie \\
    \hline
     & Czasowa niedostępność/niedyspozycja członka zespołu & Nagła lub zaplanowana nieobecność członka zespołu & średnie 
     & Możliwie wczesne sygnalizowanie planowanej nieobecności \\
    \hline
     & Zbyt ambitne założenia projektowe & Zaplanowane prace wykraczają poza moce przerobowe członków zespołu & średnie
    & Śledzenie postępu prac pod kątem wykonalności i zmieszczenia się w ograniczeniach czasowych, konsultacja z promotorem \\
    \hline
     & Pełzające wymagania & Ciągle powiększana pula wymagań, zwiększająca zakres prac do wykonania & średnie & \\
    \hline
     & Nauka nowych technologii/języków programowania & Zwiększenie ilości czasu potrzebnej na ukończenie zadania & 
     wysokie & \\
    \hline
     & Problem z komunikacją w zespole & Brak właściwego przepływu informacji między członkami zespołu & średnie
     & Cykliczne spotkania członków zespołu w celu omawiania postępu prac i planowania kolejnych działań \\
    \hline
     Produkt nie spełnia wymagań projektowych & & & & \\
    \hline
     & Błędna analiza rynku & Niewłaściwe rozpoznanie potrzeb potencjalnych użytkowników oraz rozwiązań konkurencyjnych & niskie
     & Konsultacje z promotorem, stworzenie profilu docelowego użytkownika \\
    \hline
     & Niespełnienie wymagań klientów & Finalny produkt nie zapewnia zakładanych korzyści dla użytkownika końcowego & średnie
     & Konsultacje z promotorem, cykliczna analiza kierunku rozwoju produktu \\
    \hline
     & Wydanie wersji zawierającej krytyczne błędy & Produkt zawiera błędy uniemożliwiające korzystanie z jego podstawowych funkcjonalności & niskie
     & Napisanie odpowiednich testów sprawdzających działanie funkcjonalności \\
    \hline
     & Zmiany w usługach zależnych & Wyłączenie lub zmiana warunków na jakich udostępniane są API wpływa na dostarczane funkcjonalności & średnie
     & Ustalenie alternatywnych dostawców potrzebnych usług \\
    \hline
     & Zły dobór narzędzi projektowych & Wybrane narzędzia są nieodpowiednie do zaplanowanych prac & niskie
     & Zaznajomienie się z dokumentacją używanych narzędzi, konsultacje w zespole \\
    \hline
     Niewłaściwe zaplanowanie przebiegu prac & & & & \\
    \hline
     & Przyjęcie niedpowiedniej strategii projektowej & Przyjęta strategia nie pozwala na uzyskanie wymaganych rezultatów przy dostępnych zasobach & wysokie
     & Stałe monitorowanie kierunku prac, konsultacje z promotorem \\
    \hline
     & Nierozpatrzenie potencjalnych przypadków brzegowych & Konieczność uwzględnienia nieprzewidzianych prac & wysokie
     & Staranne analizowanie planowanych funkcjonalności i ich wpływu na produkt \\
     \hline
     Zdarzenia nieprzewidziane  & & & & \\
     \hline
     & Zbyt gwałtowny rozrost liczby użytkowników & Liczba użytkowników przekracza zaplanowane możliwości techniczne produktu & niskie
     & Projektowanie produktu z możliwością jego skalowania \\



% \end{tabular}
% \end{adjustbox}
% \end{table}
\end{longtable}

Jakiś wstęp
TEST TEST

\paragraph{Paragraf}


