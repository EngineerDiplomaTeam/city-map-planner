%! Author = Wiktor Rostkowski
%! Date = 05/1/2024


\section{Analiza Ryzyka}
\label{subsec:analiza-ryzyka}

\pagebreak
\begin{longtable}{|p{.1\linewidth}|p{.15\textwidth}|p{.2\textwidth}|p{.25\textwidth}|p{.25\textwidth}|}
    \hline
    Ryzyko & Czynniki ryzyka & Charakterystyka ryzyka & Prawdopodobieństwo wystąpienia ryzyka & Planowane działania \\
    \hline
    \multirow{6}{=}{\parbox[c]{11cm}{\rotatebox{90}{\textbf{Nieukończenie projektu w terminie}}}}& Rozpad grupy & Odejście jednego z członków zespołu & niskie 
    & Przejrzysta komunikacja w zespole i ustalenie wersji MVP produktu, która może zostać ukończona w uszczuplonym składzie \\
    \cline{2-5}
     & Czasowa niedostępność/niedyspozycja członka zespołu & Nagła lub zaplanowana nieobecność członka zespołu & średnie 
     & Możliwie wczesne sygnalizowanie planowanej nieobecności \\
     \cline{2-5}
     & Zbyt ambitne założenia projektowe & Zaplanowane prace wykraczają poza moce przerobowe członków zespołu & średnie
    & Śledzenie postępu prac pod kątem wykonalności i zmieszczenia się w ograniczeniach czasowych, konsultacja z promotorem \\
    \cline{2-5}
     & Pełzające wymagania & Ciągle powiększana pula wymagań, zwiększająca zakres prac do wykonania & średnie & \\
     \cline{2-5}
     & Nauka nowych technologii/języków programowania & Zwiększenie ilości czasu potrzebnej na ukończenie zadania & 
     wysokie & \\
     \cline{2-5}
     & Problem z komunikacją w zespole & Brak właściwego przepływu informacji między członkami zespołu & średnie
     & Cykliczne spotkania członków zespołu w celu omawiania postępu prac i planowania kolejnych działań \\
    \hline
    \multirow{5}{=}{\parbox[c]{10cm}{\rotatebox[origin=c]{90}{\multirow{5}{=}{\textbf{Produkt nie spełnia wymagań projektowych}}}}} & Błędna analiza rynku & Niewłaściwe rozpoznanie potrzeb potencjalnych użytkowników oraz rozwiązań konkurencyjnych & niskie
     & Konsultacje z promotorem, stworzenie profilu docelowego użytkownika \\
     \cline{2-5}
     & Niespełnienie wymagań klientów & Finalny produkt nie zapewnia zakładanych korzyści dla użytkownika końcowego & średnie
     & Konsultacje z promotorem, cykliczna analiza kierunku rozwoju produktu \\
     \cline{2-5}
     & Wydanie wersji zawierającej krytyczne błędy & Produkt zawiera błędy uniemożliwiające korzystanie z jego podstawowych funkcjonalności & niskie
     & Napisanie odpowiednich testów sprawdzających działanie funkcjonalności \\
     \cline{2-5}
     & Zmiany w usługach zależnych & Wyłączenie lub zmiana warunków na jakich udostępniane są API wpływa na dostarczane funkcjonalności & średnie
     & Ustalenie alternatywnych dostawców potrzebnych usług \\
     \cline{2-5}
     & Zły dobór narzędzi projektowych & Wybrane narzędzia są nieodpowiednie do zaplanowanych prac & niskie
     & Zaznajomienie się z dokumentacją używanych narzędzi, konsultacje w zespole \\
    \hline
    \multirow{2}{=}{\parbox[c]{3.5cm}{\rotatebox[origin=c]{90}{\multirow{2}{=}{\textbf{Niewłaściwe zaplanowanie przebiegu prac}}}}} & Przyjęcie niedpowiedniej strategii projektowej & Przyjęta strategia nie pozwala na uzyskanie wymaganych rezultatów przy dostępnych zasobach & wysokie
     & Stałe monitorowanie kierunku prac, konsultacje z promotorem \\
     \cline{2-5}
     & Nierozpatrzenie potencjalnych przypadków brzegowych & Konieczność uwzględnienia nieprzewidzianych prac & wysokie
     & Staranne analizowanie planowanych funkcjonalności i ich wpływu na produkt \\
     \hline
     \multirow{1}{=}{\parbox[c]{2cm}{\rotatebox[origin=c]{90}{\multirow{1}{=}{\textbf{Zdarzenia nieprzewidziane}}}}}& Zbyt gwałtowny rozrost liczby użytkowników & Liczba użytkowników przekracza zaplanowane możliwości techniczne produktu & niskie
     & Projektowanie produktu z możliwością jego skalowania \\
    \hline

\end{longtable}
\pagebreak


