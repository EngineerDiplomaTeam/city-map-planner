%! Author = Wiktor Rostkowski, Mateusz Budzisz
%! Date = 05/1/2024

\chapter{Opis problemu}
\label{ch:opis-problemu}

\section{Przestawienie problemu}
\label{sec:przestawienie-problemu}

Analizując swoje potrzeby jako turystów, zespół projektowy zauważył liczne obszary, w których aplikacja może znacząco wspierać turystów.
W trakcie tej analizy, członkowie zespołu zidentyfikowali konkretne wyzwania i problemy, z jakimi turyści często się borykają.

Pierwszą kwestią jest natłok informacji, który może przytłoczyć turystów.
Podczas poszukiwania informacji o atrakcjach turystycznych, turyści są bombardowani ogromną ilością niepotrzebnych danych, takich jak punkty zainteresowań, które w rzeczywistości nie są atrakcjami turystycznymi.
Na przykład, korzystając z map Google, Bing lub Apple, użytkownicy często otrzymują wyniki, które oprócz rzeczywistych atrakcji turystycznych, zawierają również miejsca niezwiązane z turystyką, takie jak stacje benzynowe, szpitale i inne obiekty użyteczności publicznej.
Taka sytuacja utrudnia szybkie i efektywne znalezienie informacji o faktycznych atrakcjach, powodując frustrację i dezorientację użytkowników.
Dlatego niezwykle ważne jest, aby aplikacja skierowana do turystów była w stanie filtrować i precyzyjnie dostarczać informacje, które są istotne i wartościowe z punktu widzenia osób podróżujących.

Kolejnym problematycznym obszarem jest dostęp do godzin otwarcia i aktualność tych danych, które często są powielone w różnych wersjach w wielu miejscach.
Turystom zdarza się spotkać z rozbieżnościami w informacjach na temat godzin otwarcia muzeów, parków, restauracji i innych atrakcji turystycznych.
Na przykład, godziny otwarcia podane na oficjalnej stronie internetowej mogą różnić się od tych zamieszczonych na platformach społecznościowych, portalach recenzji lub w przewodnikach turystycznych.
Takie niespójności mogą prowadzić do nieporozumień i frustracji, kiedy turyści pojawiają się w miejscu, które miało być otwarte, ale okazuje się zamknięte.
Dlatego ważne jest, aby aplikacja dla turystów mogła zapewniać zaktualizowane, spójne i wiarygodne informacje o godzinach otwarcia, minimalizując ryzyko takich problemów i ułatwiając planowanie zwiedzania.

Trzecim z problematycznych obszarów jest skomplikowanie rozłożenia zwiedzania na poszczególne dni.
Ręczne tworzenie takiego planu wymaga zaawansowanych zdolności analitycznych oraz dużego nakładu pracy, a mimo to jest bardzo podatne na błędy.
Tworzony w ten sposób plan często szybko się dezaktualizuje, co wymusza konieczność jego ponownego przemyślenia i wykonania od nowa.
Chociaż istnieją narzędzia wspomagające planowanie dnia, konieczność ręcznego wprowadzania godzin otwarcia atrakcji turystycznych sprawia, że ich użycie może być bardziej czasochłonne niż przynoszące korzyści.
W efekcie, zamiast ułatwiać planowanie, narzędzia te często komplikują proces, zniechęcając użytkowników do ich stosowania.
Aby rzeczywiście wspomóc turystów w efektywnym planowaniu, aplikacja powinna automatycznie integrować aktualne informacje o godzinach otwarcia, dostarczając kompleksowe i łatwe w użyciu narzędzie do tworzenia planów zwiedzania.
Dzięki temu turyści będą mogli skupić się na cieszeniu się podróżą, a nie na logistyce jej organizacji.

Ostatnim obszarem wymagającym usprawnień jest ułożenie trasy z wykorzystaniem komunikacji miejskiej pomiędzy wieloma punktami.
Gdy użytkownik stworzy plan zwiedzania, powinien mieć automatyczną możliwość zobaczenia dostępnych połączeń między wybranymi atrakcjami bez konieczności ręcznego wyszukiwania informacji o dostępnych środkach transportu.
Wprowadzenie takiej funkcji w aplikacji turystycznej znacznie ułatwiłoby podróżowanie, eliminując potrzebę czasochłonnego i często skomplikowanego poszukiwania informacji o trasach, rozkładach jazdy i przesiadkach.
Automatyczna integracja danych o komunikacji miejskiej zapewniłaby turystom łatwy dostęp do aktualnych i precyzyjnych informacji, co przyczyniłoby się do bardziej efektywnego planowania czasu oraz zwiększenia komfortu podróżowania.
Dzięki temu turyści mogliby skupić się na zwiedzaniu i czerpaniu przyjemności z odkrywania nowych miejsc, mając pewność, że aplikacja zadba o logistyczne aspekty ich podróży.

\section{Rich picture}
\label{sec:rich-picture}

TODO

\section{Cele projektu}
\label{sec:cele-projektu}

TODO
