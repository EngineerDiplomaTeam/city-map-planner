%! Author = Mateusz Budzisz
%! Date = 26/05/2024

\chapter{Realizacja Projektu}
\label{ch:realizacja}
Po zakończeniu prac analitycznych i omówieniu przewidywanych funkcjonalności przeprowadzono analizę możliwych decyzji projektowych. Podjęto działania umożliwiające wybór odpowiednich rozwiązań projektowych. 
Następnie przystąpiono do realizacji poszczególnych komponentów systemu.

\section{Aplikacja City Map Planner}
\label{sec:aplikacja}

\subsection{Przyrost I - utworzenie szkieletu aplikacji oraz Potoki testów aplikacji}
\label{sec:przyrost1}

W ramach tego przyrostu pierwszego wykonano:
\begin{itemize}
    \item backend WebAPI z podstawowym testem działania funkcjonowania;
    \begin{figure}[H]
        \centering
        \includegraphics[width=1\textwidth]{attachments/1}
        \caption{Wykonane elementy w ramach pierwszego przyrostu}
        \label{fig:figure1}
        \end{figure}
    \item frontend z ustaleniem wyglady aplikacji o oparcie Material design + api example;
    \begin{figure}[H]
        \centering
        \includegraphics[width=1\textwidth]{attachments/2}
        \caption{Wykonane elementy w ramach pierwszego przyrostu}
        \label{fig:figure2}
        \end{figure}
    \item Potok ciągłego wdrożenia opisane w rozdziale 6.6;
\end{itemize}
 
Opis w czasie wykonanych zadań przestawia wykres Gantt'a
\begin{figure}[H]
    \centering
    \includegraphics[width=1\textwidth]{attachments/RALLY1}
    \caption{Wykonane elementy w ramach pierwszego przyrostu}
    \label{fig:figure3}
    \end{figure}

    \subsection{Przyrost II - utworzenie widoku mapy, w tym integracja  Overpass}
    \label{sec:przyrost2}

    W ramach tego przyrostu drugiego wykonano elementy, pomiędzy każdym przedstawiono fragmenty wybranych implementacji:
    \begin{itemize}
        \item integracja klienta Overpass API,
        \begin{figure}[H]
            \centering
            \includegraphics[width=1\textwidth]{attachments/overpassclient}
            \caption{Overpass klient}
            \label{fig:figure4}
            \end{figure}
        \item przygotowanie entities dla Entity Framework Core;
        \begin{figure}[H]
            \centering
            \includegraphics[width=1\textwidth]{attachments/Entity}
            \caption{Fragment klasy BusinessTimeEntity}
            \label{fig:figure5}
            \end{figure}
        \item wykonano kontrolery integrujące usługi
        \begin{figure}[H]
            \centering
            \includegraphics[width=1\textwidth]{attachments/OverpassIntegrationController}
            \caption{Fragment klasy OverpassIntegrationController}
            \label{fig:figure6}
            \end{figure}
        \item implementacja Domain
        \begin{figure}[H]
            \centering
            \includegraphics[width=1\textwidth]{attachments/PoisDomain}
            \caption{Fragment klasy Pois}
            \label{fig:figure7}
            \end{figure}
        \item implementacja Dto
        \begin{figure}[H]
            \centering
            \includegraphics[width=1\textwidth]{attachments/PoiMapDto}
            \caption{Fragment klasy PoiMapDto}
            \label{fig:figure}
            \end{figure}

        \item integracja LeaftModule;
        \begin{figure}[H]
            \centering
            \includegraphics[width=1\textwidth]{attachments/leaflet1}
            \caption{Fragment klasy color-scheme-based-tile-layer}
            \label{fig:figure}
            \end{figure}
            \begin{figure}[H]
                \centering
                \includegraphics[width=1\textwidth]{attachments/leaflet2}
                \caption{Fragment klasy PoiSelectorComponent}
                \label{fig:figure}
                \end{figure}
    \end{itemize}

    \subsection{Przyrost III - zarządzanie użytkownikiem}
    \label{sec:przyrost3}
    Backend/WebApi/Services/EmailSender.cs
    W ramach tego przyrostu trzeciego wykonano:
    \begin{itemize}
        \item widok logowania;
        \item widok rejestracji z automatycznym potwierdzeniem,
        \begin{figure}[H]
            \centering
            \includegraphics[width=1\textwidth]{attachments/emailsender}
            \caption{Fragment klasy EmailSender z widocznym serwisem potwierdzania rejestracji}
            \label{fig:figure}
            \end{figure}

        \item widok zarządzania kontem;
        \item zaimplementowano autoryzacje logowania po stronie aplikacji oraz bazy danych,
    \begin{figure}[H]
        \centering
        \includegraphics[width=1\textwidth]{attachments/userdb}
        \caption{Fragment klasy UserDataDbContext}
        \label{fig:figure}
        \end{figure}
    \end{itemize}

    %https://github.com/EngineerDiplomaTeam/city-map-planner/tree/b85c21829a74f57ce20414a7fa3ab3a398ad9833/Backend/WebApi/Extensions
    %https://github.com/EngineerDiplomaTeam/city-map-planner/commit/bbdbb4fdca805acd2438365551f4f361532bc34d


    \subsection{Przyrost IV - algorytm Trasy - \glslink{poidef}{POI}}
    \label{sec:przyrost4}

    W ramach tego przyrostu czwartego wykonano:
    \begin{itemize}
        \item dodanie \glslink{poidef}{POI} na mapie uzwgledniając wszystkie podstawowe informacje;
        \item dodanie algorytmu przeszukania trasy pomiędzy \glslink{poidef}{POI};
        \item dodanie ekranu koszyka z \glslink{poidef}{POI};
        %https://github.com/EngineerDiplomaTeam/city-map-planner/commit/88c32b690c76b113173d056ec0be1fa74c81709a#diff-ede64cf229dabd7c007a109c15db77bcacd008a0cf42e1eacca43c3e3da9af97
        \item dodanie indywidualnych zdjeć;
        %https://github.com/EngineerDiplomaTeam/city-map-planner/commit/354caede1081fe73ca98350bd1aceff95c7df8c9
        \item wykonano liste przykładowych atrakcji w Gdańsku
    \end{itemize}


    \subsection{Przyrost V - zarządzanie POI i kalendarz podróży}
    \label{sec:przyrost5}

    W ramach tego przyrostu piątego wykonano:
    \begin{itemize}
        \item integracja chatGPT do inportowania aktualnych o atrakcjach;
        % https://github.com/EngineerDiplomaTeam/city-map-planner/commit/989c85a11ceb0f7f9951f04838e4464c0a47d070
        \item dodanie ekranu kalendarza podróży
        \item Widok podsumowania podróży
        %https://github.com/EngineerDiplomaTeam/city-map-planner/commit/796bac2b6bc5db568a7aabc7dcdef299117c2da2
        \item integracja aplikacji oraz bazy danych z API pogodowym
        %https://github.com/EngineerDiplomaTeam/city-map-planner/commit/1f899291a34701bfa61bfc0b39fd558b628bb966
        \item poprawienie widoku kalendarza
       % https://github.com/EngineerDiplomaTeam/city-map-planner/commit/a3d894bcbf5c319bb2cd109a9b2c3e58db29ef56
    \end{itemize}

    \subsection{Przyrost VI - widok wszystkich atrakcji integracja pogody}
    \label{sec:przyrost5}

    W ramach tego przyrostu szóstego wykonano:
    \begin{itemize}
        \item integracja pogody widoku na przeglądarce internetowej;
        \item widok listy wszystkich POI
    \end{itemize}

