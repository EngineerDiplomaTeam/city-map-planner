%! Author = Wiktor Rostkowski
%! Date = 13/06/2024

\chapter{Realizacja Projektu}
\label{ch:realizacja}
Po zakończeniu prac analitycznych i omówieniu przewidywanych funkcjonalności przeprowadzono analizę możliwych decyzji projektowych. Podjęto działania umożliwiające wybór odpowiednich rozwiązań projektowych. 
Następnie przystąpiono do realizacji poszczególnych komponentów systemu.

\section{Aplikacja City Map Planner}
\label{sec:aplikacja}

\subsection{Przyrost I - utworzenie szkieletu aplikacji oraz Potoki testów aplikacji}
\label{sec:przyrost1}

W ramach pierwszego przyrostu utworzono szkielet aplikacji i potoki testów, obejmujące backend, frontend oraz ciągłe dostarczanie (CI/CD). \newline
\indent Przygotowano Backend WebAPI umożliwiający rozbudowę pod RESTAPI. Wdrożono podstawowe testy funkcjonalne, aby upewnić się, że WebAPI działa poprawnie.\newline
\indent Frontend został zaprojektowany zgodnie z zasadami Material Design i zintegrowany z backendem poprzez przykładowe zapytania do API, umożliwiając interakcję użytkownika z danymi na serwerze. \newline
\indent Wdrożono potok CI/CD, automatyzujący budowanie, testowanie i wdrażanie aplikacji przy użyciu narzędzia GitHub Actions. Potok ten uruchamiał testy integracyjne, weryfikując poprawność działania wszystkich komponentów aplikacji, w tym dokumentacji książki pisanej w Latex. \newline
\indent Wszystkie zadania przedstawiono na wykresie Gantta, monitorując postępy i identyfikując ewentualne opóźnienia. Stworzono solidny fundament aplikacji, umożliwiający dalszy rozwój funkcjonalności oraz zapewniający stabilną i skalowalną architekturę. \newline

\subsection{Przyrost II - utworzenie widoku mapy, w tym integracja  Overpass}
    \label{sec:przyrost2}

    W ramach drugiego przyrostu projektu zrealizowano kluczowe zadania, które rozszerzyły funkcjonalność i integrację systemu. \newline
    \indent Pierwszym krokiem była integracja klienta Overpass API, umożliwiająca pozyskiwanie danych z OpenStreetMap. Zbudowano klienta API do wysyłania zapytań i przetwarzania odpowiedzi, co pozwala na dynamiczne pobieranie i aktualizowanie danych mapowych.\newline
    \indent Następnie przygotowano encje dla Entity Framework Core, definiując model danych odpowiadający strukturze bazy danych.
    Wykonano kontrolery integrujące usługi, które obsługują żądania użytkowników i komunikację między frontendem a backendem. Kontrolery korzystają z usług i encji do realizacji funkcjonalności aplikacji.\newline
    \indent Implementacja warstwy domeny obejmowała logikę biznesową, definiując zasady i operacje na danych. Warstwa domeny oddziela logikę biznesową od technicznych szczegółów, co poprawia modularność i utrzymanie kodu.\newline
    \indent Przygotowano również Data Transfer Objects (Dto) do przenoszenia danych między warstwami aplikacji.    \newline
    \indent Ostatnim etapem była integracja modułu Leaflet do tworzenia interaktywnych map.\newline
    Te działania znacząco rozszerzyły funkcjonalność aplikacji, poprawiając jej elastyczność, skalowalność i użyteczność.\newline


    \subsection{Przyrost III - zarządzanie użytkownikiem}
    \label{sec:przyrost3}
    W ramach trzeciego przyrostu projektu zrealizowano funkcje związane z zarządzaniem kontem użytkownika, obejmujące logowanie, rejestrację, zarządzanie kontem oraz autoryzację i odzyskiwanie hasła. \newline
    \indent Uniwersalny widok logowania i rejestracji: Stworzono jeden widok, który umożliwia zarówno logowanie, jak i rejestrację. Rejestracja wymaga potwierdzenia poprzez link wysłany na e-mail użytkownika. \newline
    \indent Widok zarządzania kontem umożliwia zarządzaniem kontem poprzez edycję danych email, zmianę hasła jak i usuniecie konta. \newline 
    \indent Dodano mechanizm umożliwiający użytkownikom odzyskiwanie hasła poprzez e-mail z linkiem do jego resetowania.

    %W ramach tego przyrostu trzeciego wykonano:
    %\begin{itemize}
    %    \item widok logowania;
    %    \item widok rejestracji z automatycznym potwierdzeniem,


    %    \item widok zarządzania kontem;
    %    \item zaimplementowano autoryzacje logowania po stronie aplikacji oraz bazy danych,
    %\end{itemize}

    %https://github.com/EngineerDiplomaTeam/city-map-planner/tree/b85c21829a74f57ce20414a7fa3ab3a398ad9833/Backend/WebApi/Extensions
    %https://github.com/EngineerDiplomaTeam/city-map-planner/commit/bbdbb4fdca805acd2438365551f4f361532bc34d


    \subsection{Przyrost IV - algorytm Trasy - \glslink{poidef}{POI}}
    \label{sec:przyrost4}
    W ramach czwartego przyrostu projektu zrealizowano kluczowe funkcje zwiększające interaktywność i użyteczność aplikacji. \newline
    \indent Stworzono listę popularnych atrakcji turystycznych w Gdańsku, służącą jako inspiracja i pokaz możliwości aplikacji. \newline
    \indent Wprowadzono do bazy danych \glslink{poidef}{POI} z podstawowymi informacjami, takimi jak nazwa, opis, lokalizacja i godziny otwarcia.
    \indent Zaimplementowano algorytm wyznaczania optymalnych tras pomiędzy wybranymi \glslink{poidef}{POI}, uwzględniający odległość, czas podróży.
    \indent Dodano ekran koszyka, gdzie użytkownicy mogą zbierać i zarządzać listą wybranych \glslink{poidef}{POI}, co ułatwia planowanie wycieczek.
    \indent Wprowadzono wyświetlanie indywidualnych zdjęć dla każdego \glslink{poidef}{POI}, co wzbogaca wizualne doświadczenie użytkowników.



    %W ramach tego przyrostu czwartego wykonano:
    %\begin{itemize}
    %    \item dodanie \glslink{poidef}{POI} na mapie uzwgledniając wszystkie podstawowe informacje;
    %    \item dodanie algorytmu przeszukania trasy pomiędzy \glslink{poidef}{POI};
    %    \item dodanie ekranu koszyka z \glslink{poidef}{POI};
        %https://github.com/EngineerDiplomaTeam/city-map-planner/commit/88c32b690c76b113173d056ec0be1fa74c81709a#diff-ede64cf229dabd7c007a109c15db77bcacd008a0cf42e1eacca43c3e3da9af97
    %    \item dodanie indywidualnych zdjeć;
        %https://github.com/EngineerDiplomaTeam/city-map-planner/commit/354caede1081fe73ca98350bd1aceff95c7df8c9
    %    \item wykonano liste przykładowych atrakcji w Gdańsku
    %\end{itemize}


    \subsection{Przyrost V - zarządzanie POI i kalendarz podróży}
    \label{sec:przyrost5}

    W ramach tego przyrostu piątego wykonano:
    Wprowadzono integrację z ChatGPT, umożliwiającą importowanie aktualnych informacji o godzinach otwarcia atrakcjach turystycznych. 
    Dzięki temu użytkownicy mogą uzyskać najnowsze dane dotyczące miejsc, które planują odwiedzić.\newline
    \indent Integracja z ChatGPT przyniosła jednak pewne trudności. Za każdym razem, gdy ChatGPT dostarczał informacje, robił to w innym schemacie, 
    co wymagało dodatkowego przetwarzania danych. Było to wyzwaniem, ponieważ zmienność w strukturze danych utrudniała ich automatyczne parsowanie i integrację z naszą bazą danych. \newline 
    \indent Aby rozwiązać ten problem, zaimplementowano dodatkowe warstwy weryfikacji i normalizacji danych, które standaryzowały informacje dostarczane przez ChatGPT.\newline
    \indent Stworzono ekran kalendarza podróży, który pozwala użytkownikom na planowanie i organizowanie swoich wycieczek.
    Kalendarz umożliwia łatwe dodawanie, edytowanie i przeglądanie zaplanowanych wizyt w różnych miejscach.
    \indent Dodano widok podsumowania podróży, który prezentuje zorganizowane informacje o wszystkich zaplanowanych punktach zainteresowania. 
    Użytkownicy mogą przeglądać swoje trasy, czasy wizyt oraz dodatkowe szczegóły dotyczące każdej atrakcji. \newline
    \indent Aplikacja oraz baza danych zostały zintegrowane z API pogodowym, co umożliwia wyświetlanie aktualnych informacji o pogodzie dla zaplanowanych punktów zainteresowania. 
    Dzięki temu użytkownicy mogą lepiej planować swoje wizyty, uwzględniając warunki atmosferyczne. \newline
    Ulepszono widok kalendarza o dodatkowe kolory w celu lepszego wyróżnienia aplikacji. 


    %\begin{itemize}
    %    \item integracja chatGPT do inportowania aktualnych o atrakcjach;
        % https://github.com/EngineerDiplomaTeam/city-map-planner/commit/989c85a11ceb0f7f9951f04838e4464c0a47d070
    %    \item dodanie ekranu kalendarza podróży
    %    \item Widok podsumowania podróży
        %https://github.com/EngineerDiplomaTeam/city-map-planner/commit/796bac2b6bc5db568a7aabc7dcdef299117c2da2
    %    \item integracja aplikacji oraz bazy danych z API pogodowym
        %https://github.com/EngineerDiplomaTeam/city-map-planner/commit/1f899291a34701bfa61bfc0b39fd558b628bb966
    %    \item poprawienie widoku kalendarza
       % https://github.com/EngineerDiplomaTeam/city-map-planner/commit/a3d894bcbf5c319bb2cd109a9b2c3e58db29ef56
    % \end{itemize}

    \subsection{Przyrost VI - widok wszystkich atrakcji integracja pogody}
    \label{sec:przyrost6}

    W ramach szóstego przyrostu projektu zrealizowano następujące zadania:
  
    \indent  Dodano funkcjonalność wyświetlania informacji o pogodzie w widoku przeglądarki internetowej. 
        Użytkownicy mogą teraz na bieżąco sprawdzać warunki atmosferyczne dla wszystkich punktów zainteresowania bezpośrednio w aplikacji
    \indent Stworzono widok listy wszystkich punktów zainteresowania (POI), umożliwiający użytkownikom przeglądanie i wyszukiwanie interesujących miejsc. 
        Lista zawiera szczegółowe informacje o każdej atrakcji, co ułatwia planowanie wizyt i organizowanie wycieczek.

