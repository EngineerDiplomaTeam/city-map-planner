%! Author = Wiktor Rostkowski, Mateusz Budzisz
%! Date = 29/04/2024

\chapter{Podsumowanie}
\label{ch:podsumowanie}

\section{Napotkane wyzwania}
\label{sec:napotkane-wyzwania}

Największym wyzwaniem, z jakim zespół projektowy musiał się zmierzyć, był częściowy rozpad grupy, w wyniku którego zespół zmniejszył się o połowę.
To znaczące uszczuplenie liczby członków wpłynęło na zdolność zespołu do realizacji wszystkich zaplanowanych funkcjonalności.
W rezultacie projekt nie był w stanie w pełni zrealizować integracji z komunikacją miejską w Gdańsku, co było jedną z głównych funkcji przewidzianych w początkowym planie.
Ograniczenia te zmusiły zespół do priorytetyzacji innych elementów projektu oraz do poszukiwania alternatywnych rozwiązań, aby sprostać ograniczeniom zasobów i czasu.

Drugim największym wyzwaniem była implementacja kalendarza, który umożliwia planowanie zwiedzania w wybrane dni za pomocą funkcji przeciągnij i upuść.
Kalendarz ten uwzględnia godziny otwarcia atrakcji turystycznych oraz proponowany czas zwiedzania każdej atrakcji.
Algorytm kalendarza musi pilnować wszystkich tych ograniczeń, aby zaplanowany harmonogram był możliwy do zrealizowania w rzeczywistości.
Dodatkowo, możliwość zmiany czasu przeznaczonego na zwiedzanie przez użytkownika wprowadziła kolejny poziom skomplikowania, wymagając dynamicznej aktualizacji planu.
Ta funkcjonalność musiała również uwzględniać różne scenariusze i wyjątkowe przypadki, takie jak zmiany godzin otwarcia atrakcji w dni świąteczne czy nieprzewidziane zamknięcia, co dodatkowo komplikowało implementację.

\begin{figure}[H]
    \centering
    \includegraphics[width=1\textwidth]{attachments/t1}
    \caption{Screenshot przykładowego użycia kalendarza}
\end{figure}

Trzecim wyzwaniem, przed którym stanął zespół projektowy, była integracja menedżera punktów zainteresowania (POI) ze sztuczną inteligencją.
System co godzinę zapisuje snapshot stron internetowych atrakcji i porównuje je z wcześniejszymi wersjami w celu wykrycia zmian.
W przypadku wykrycia zmian, dane są przesyłane do modelu GPT-4 od OpenAI w celu ekstrakcji godzin otwarcia, co częściowo automatyzuje proces aktualizacji informacji.
Sztuczna inteligencja zwraca dane w formacie tabelki markdown z wartościami wymagającymi normalizacji.
Przetwarzanie tych zdenormalizowanych danych przed zapisaniem ich do finalnej tabeli w bazie danych stanowiło znaczące wyzwanie.
Ponadto konieczne było uwzględnienie i pokrycie wielu przypadków anomalii generowanych przez AI, co dodatkowo skomplikowało ten proces.
Wyzwanie to wymagało nie tylko solidnej wiedzy technicznej, ale także elastyczności w dostosowywaniu systemu do różnych formatów i źródeł danych.

\begin{figure}[H]
    \centering
    \includegraphics[width=1\textwidth]{attachments/t2}
    \caption{Przykład analizy AI}
\end{figure}

\section{Łączny nakład pracy}
\label{sec:laczny-naklad-pracy}

Łączny czas poświęcony przez członków zespołu na realizację tego projektu wyniósł 363 godziny.
Ten czas obejmował szeroki zakres działań, w tym analizę wymagań, projektowanie architektury systemu, implementację poszczególnych modułów, testowanie funkcjonalności oraz naprawianie błędów.
W ramach tych godzin zespół przeprowadzał także spotkania planistyczne i statusowe, konsultacje z interesariuszami, a także sesje programowania ekstremalnego, aby zapewnić, że wszystkie elementy projektu są zgodne z założeniami i spełniają oczekiwania użytkowników końcowych.

\section{Indywidualne wkłady pracy}
\label{sec:indywidualne-wklady-pracy}

\subsection{Mateusz Budzisz}
\label{subsec:mateusz-budzisz}

Bezpośrednio na projekt przeznaczyłem 253 godziny.
Jestem autorem pomysłu pracy dyplomowej oraz kierownikiem projektu.
Do moich obowiązków należało: organizacja cotygodniowych spotkań, przygotowanie infrastruktury docelowej, stworzenie i utrzymywanie repozytorium oraz ogólnie pojęta organizacja całości projektu.

W znacznej większości zaimplementowałem zarówno backend, jak i frontend systemu, co stanowiło kluczowy element techniczny projektu.
Dodatkowo, większość rozdziałów pracy dyplomowej jest mojego autorstwa, co odzwierciedla moje zaangażowanie i wkład merytoryczny w dokumentację projektu.

Pomimo moich wszelkich starań, aby wspierać innych członków zespołu, w tym poprzez propozycje wspólnych sesji programowania oraz udostępnienie prywatnych materiałów szkoleniowych, nie udało mi się przekonać ich do systematycznej pracy na rzecz projektu.
Starałem się aktywnie motywować i wspierać zespół, jednak spotkałem się z brakiem odpowiedniego zaangażowania z ich strony.
Moje wysiłki w zakresie organizacji i koordynacji prac projektowych, jak również znaczny wkład w rozwój techniczny i dokumentacyjny projektu, były kluczowe dla jego postępu i finalizacji.

\subsection{Wiktor Rostkowski}
\label{subsec:wiktor-rostkowski}

TODO
