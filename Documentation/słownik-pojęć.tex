%! Author = Mateusz Budzisz
%! Date = 08/11/2023

\newacronym{pwa}{PWA}{Progressive Web App}
\newacronym{aws}{AWS}{Amazon Web Services}
\newacronym{gcp}{GCP}{Google Cloud Platform}
\newacronym{azure}{Azure}{Google Cloud Platform}
\newacronym{osm}{OSM}{Open Street Map}
\newacronym{lsp}{LSP}{Language Service Protocol}
\newacronym{osw}{OSW}{Open Weather Map}
\newacronym{poi}{POI}{Point of interest}
\newacronym{http}{HTTP}{Hyper Text Transfer Protocol}

\newglossaryentry{on-demand}
{
    name={On-demand},
    description={Rodzaj opgrogramowania charakteryzojącego się dynamicznym czasem pracy, uruchamiane na rządanie, gdy poda wynik program kończy pracę zamiast oczekiwać następnego zapytania}
}

\newglossaryentry{refactoring}
{
    name={Refactoring},
    description={Znaczna zmiana konstrukcji programu mająca na celu usprawnienie oprogramowania bądź dostosowanie go do nowych wymogów}
}

\newglossaryentry{ui}
{
    name={UI},
    description={Interfejs użytkownika}
}

\newglossaryentry{frontend}
{
    name={Frontend},
    description={Oprogramowanie składające się z UI z którym docelowy użytkownik będzę wchodził w interakcję}
}

\newglossaryentry{backend}
{
    name={Backend},
    description={Oprogramowanie pozbawione UI z którym docelowy użytkownik będzę wchodził w interakcję, potrzebne do prawidłowej pracy Frontendu}
}

\newglossaryentry{job}
{
    name={Job},
    description={Oprogramowanie, które ma z góry określony cel, po jego uruchomieniu natychmiast zaczyna je wykonywać, nie wchodzi w interakcje z użytkownikiem docelowym, po zakończeniu kończy swoje życie}
}


\newglossaryentry{rendering}
{
    name={Wyrenderowanie},
    description={Stworzenie UI z postaci kodu do postaci konsumowalnej przez użytkownika docelowego}
}

\newglossaryentry{hello-world}
{
    name={Hello world},
    description={Minimalny reprezentatywny program w danej technologii}
}

\newglossaryentry{hermetyzacja}
{
    name={Hermetyzacja},
    description={Hermetyzacja opgorgramowania określa dobrą praktykę programistyczną polegającą na izolacji komponentów w aplikacji tak aby o sobie nie wiedziały gdy nie muszą o sobie wiedzieć}
}

\newglossaryentry{infra-as-code}
{
    name={Infrastructure as a code},
    description={Sposób opisania architektury systemu po przez napisanie programu tworzącego docelową architekturę przy pomocy abstrakcji dostarczonych przez dostawcę mocy obliczeniowej}
}

\newglossaryentry{on-prem}
{
    name={On-premise},
    description={Oprogramowanie hostowane na samodzielnie zarządzanej infrastrukturze}
}

\newglossaryentry{virt}
{
    name={Wirtualizacja},
    description={Podział serwera na maszyny o mniejszej mocy obliczeniowej aby umożliwić podział podsespołów pomiedzy klientów tak aby nic o sobie na wzajem nie wiedzieli}
}

\newglossaryentry{arm}
{
    name={Architektura ARM},
    description={Architektura silnej ręki lol}
}
\newglossaryentry{poidef}
{
    name={POI},
    description={Point of interest (w skrócie POI) to punkt w przestrzeni, najczęściej na powierzchni Ziemi.}
}
\newglossaryentry{openlayers}
{
    name={Openlayers},
    description={OpenLayers to biblioteka napisana w języku JavaScript, ułatwiająca dodawanie dynamicznych map na stronach internetowych..}
}
\newglossaryentry{reflink}
{
    name={Reflink},
    description={Reflink to specjalny link, który zawiera unikalny kod identyfikacyjny, pozwalający na monitorowanie ruchu i przekierowań z innych źródeł.}
}
\newglossaryentry{sla}
{
    name={SLA},
    description={Service Level Agreement, SLA (umowa o gwarantowanym poziomie świadczenia usług) to umowa utrzymania i systematycznego poprawiania ustalonego między usługodawcą a usługobiorcą poziomu jakości usług poprzez stały cykl obejmując.}
}
\newglossaryentry{sws}
{
    name={SWS},
    description={Specyfikacja Wymagań Systemowych }
}
\newglossaryentry{moscow}
{
    name={MoSCoW},
    description={Metoda MoSCoW to technika priorytetyzacji wykorzystywana w analizie biznesowej i przy tworzeniu oprogramowania w celu osiągnięcia wspólnego zrozumienia pomiędzy interesariuszami co do znaczenia, jakie ma dla nich dostarczenie każdego z wymagań. Inne nazwy metody to priorytetyzacja MoSCoW lub analiza MoSCoW.}
}