\documentclass{report}
\usepackage{tocloft}
\usepackage{hyperref}
\usepackage[english,polish]{babel}
\usepackage[T1]{fontenc}

\renewcommand{\cftchapfont}{\normalfont}
\renewcommand{\cftchappagefont}{\normalfont}

\begin{document}
	
	\tableofcontents{
		author = {Mateusz Budzisz Wiktor Rostkowski, Damian Kreft, Sebastian Kreft},
		title = {City Map planner},
		OPTurl = {url},
	}
	
	\clearpage
		\chapter{Wstęp}
	
	Celem projektu jest stworzenie interaktywnej Mapy z punktami atrakcji, która umożliwi kompleksowe zaplanowanie optymalnej trasy zwiedzania z uwzględnieniem środków komunikacji miejskiej oraz godzin otwarcia atrakcji turystycznych. W dzisiejszym świecie, gdzie podróże turystyczne stają się coraz popularniejsze, narzędzie to będzie niezwykle przydatne dla turystów, zwłaszcza w dużych miastach, gdzie dostępność środków komunikacji miejskiej i różnorodność atrakcji turystycznych stanowią kluczowe wyzwania.
	
	W miastach o dużej liczbie atrakcji turystycznych i rozbudowanych systemach transportu publicznego, planowanie efektywnej trasy zwiedzania może być skomplikowane. Warto więc opracować narzędzie, które ułatwi turystom eksplorację miasta i zapewni im możliwość zwiedzania miejscowych atrakcji w najbardziej optymalny sposób.
	
	Projekt ten zakłada integrację różnych źródeł danych, takich jak informacje o trasach komunikacji miejskiej, godziny otwarcia atrakcji, informacje turystyczne i dane geograficzne, aby dostarczyć spersonalizowane trasy zwiedzania, uwzględniając preferencje użytkownika, takie jak zainteresowania i dostępność czasu.
	
	W niniejszej pracy omówimy cele projektu, jego znaczenie w kontekście współczesnego ruchu turystycznego oraz zaprezentujemy plan działania i zakres projektu. Ponadto, przedstawimy również strukturę pracy oraz sposób, w jaki kolejne rozdziały będą się rozwijać, aby dokładnie opisać proces tworzenia interaktywnej Mapy z punktami atrakcji.
	
	\section{Cel Projektu}
	
	Głównym celem projektu jest:
	
	\begin{itemize}
		\item Utworzenie interaktywnej Mapy z punktami atrakcji, umożliwiającej kompleksowe zaplanowanie optymalnej trasy zwiedzania w mieście wojewódzkim Gdańsk.
	\end{itemize}
	
	W ramach projektu będziemy dążyć do osiągnięcia tego celu poprzez integrację danych geograficznych, informacji o komunikacji miejskiej oraz godzin otwarcia atrakcji turystycznych w jednym narzędziu, które będzie dostępne dla użytkowników na platformach mobilnych i internetowych. Dzięki temu turystowie będą mieli możliwość zaplanowania swojej podróży w sposób bardziej efektywny i dostosowany do swoich indywidualnych preferencji.
	
	\section{Znaczenie Projektu}
	
	Projekt ma ogromne znaczenie w kontekście współczesnego ruchu turystycznego. Ułatwienie zwiedzania miast poprzez zoptymalizowane trasy może przyciągnąć więcej turystów i przyczynić się do polepszenia ich doświadczenia podczas podróży. 
	
	Ponadto, projekt może również przyczynić się do zmniejszenia negatywnego wpływu masowego ruchu turystycznego na środowisko, poprzez zoptymalizowanie wykorzystania środków transportu publicznego i zmniejszenie konieczności korzystania z samochodów prywatnych.
	
	W kolejnych rozdziałach pracy zostaną dokładnie omówione kroki projektowe, technologie użyte do jego realizacji oraz wyniki uzyskane podczas tworzenia interaktywnej Mapy z punktami atrakcji.
	
	\chapter{Wprowadzenie}
	\section{Cel pracy}
	\section{Zakres pracy}
	\section{Struktura pracy}
	+++
	\chapter{Stan wiedzy}
	\section{Historia tematu}
	\section{Aktualny stan badań}
	
	\chapter{Metodologia}
	\section{Wybór metod badawczych}
	\section{Przygotowanie próbek}
	\section{Pomiary i eksperymenty}
	
	\chapter{Wyniki}
	\section{Analiza danych}
	\section{Prezentacja wyników}
	
	\chapter{Dyskusja}
	\section{Porównanie z wynikami innych badań}
	\section{Wnioski}
	
	\chapter{Podsumowanie}
	\section{Osiągnięte cele}
		\section{Perspektywy dalszych badań}
	
	
	

	
	
	\begin{thebibliography}{99}
		\bibitem{autor1} Nazwisko, Imię. Tytuł publikacji. Wydawnictwo, Rok.
		\bibitem{autor2} Nazwisko, Imię. Tytuł publikacji. Wydawnictwo, Rok.
	\end{thebibliography}
	
\end{document}
